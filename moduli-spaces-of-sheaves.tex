\IfFileExists{stacks-project.cls}{%
\documentclass{stacks-project}
}{%
\documentclass{amsart}
}

% For dealing with references we use the comment environment
\usepackage{verbatim}
\newenvironment{reference}{\comment}{\endcomment}
%\newenvironment{reference}{}{}
\newenvironment{slogan}{\comment}{\endcomment}
\newenvironment{history}{\comment}{\endcomment}

% For commutative diagrams we use Xy-pic
\usepackage[all]{xy}

% We use 2cell for 2-commutative diagrams.
\xyoption{2cell}
\UseAllTwocells

% We use multicol for the list of chapters between chapters
\usepackage{multicol}

% This is generally recommended for better output
\usepackage{lmodern}
\usepackage[T1]{fontenc}

% For cross-file-references
\usepackage{xr-hyper}

% Package for hypertext links:
\usepackage{hyperref}

% For any local file, say "hello.tex" you want to link to please
% use \externaldocument[hello-]{hello}
\externaldocument[introduction-]{introduction}
\externaldocument[conventions-]{conventions}
\externaldocument[sets-]{sets}
\externaldocument[categories-]{categories}
\externaldocument[topology-]{topology}
\externaldocument[sheaves-]{sheaves}
\externaldocument[sites-]{sites}
\externaldocument[stacks-]{stacks}
\externaldocument[fields-]{fields}
\externaldocument[algebra-]{algebra}
\externaldocument[brauer-]{brauer}
\externaldocument[homology-]{homology}
\externaldocument[derived-]{derived}
\externaldocument[simplicial-]{simplicial}
\externaldocument[more-algebra-]{more-algebra}
\externaldocument[smoothing-]{smoothing}
\externaldocument[modules-]{modules}
\externaldocument[sites-modules-]{sites-modules}
\externaldocument[injectives-]{injectives}
\externaldocument[cohomology-]{cohomology}
\externaldocument[sites-cohomology-]{sites-cohomology}
\externaldocument[dga-]{dga}
\externaldocument[dpa-]{dpa}
\externaldocument[sdga-]{sdga}
\externaldocument[hypercovering-]{hypercovering}
\externaldocument[schemes-]{schemes}
\externaldocument[constructions-]{constructions}
\externaldocument[properties-]{properties}
\externaldocument[morphisms-]{morphisms}
\externaldocument[coherent-]{coherent}
\externaldocument[divisors-]{divisors}
\externaldocument[limits-]{limits}
\externaldocument[varieties-]{varieties}
\externaldocument[topologies-]{topologies}
\externaldocument[descent-]{descent}
\externaldocument[perfect-]{perfect}
\externaldocument[more-morphisms-]{more-morphisms}
\externaldocument[flat-]{flat}
\externaldocument[groupoids-]{groupoids}
\externaldocument[more-groupoids-]{more-groupoids}
\externaldocument[etale-]{etale}
\externaldocument[chow-]{chow}
\externaldocument[intersection-]{intersection}
\externaldocument[pic-]{pic}
\externaldocument[weil-]{weil}
\externaldocument[adequate-]{adequate}
\externaldocument[dualizing-]{dualizing}
\externaldocument[duality-]{duality}
\externaldocument[discriminant-]{discriminant}
\externaldocument[derham-]{derham}
\externaldocument[local-cohomology-]{local-cohomology}
\externaldocument[algebraization-]{algebraization}
\externaldocument[curves-]{curves}
\externaldocument[resolve-]{resolve}
\externaldocument[models-]{models}
\externaldocument[functors-]{functors}
\externaldocument[equiv-]{equiv}
\externaldocument[pione-]{pione}
\externaldocument[etale-cohomology-]{etale-cohomology}
\externaldocument[proetale-]{proetale}
\externaldocument[relative-cycles-]{relative-cycles}
\externaldocument[more-etale-]{more-etale}
\externaldocument[trace-]{trace}
\externaldocument[crystalline-]{crystalline}
\externaldocument[spaces-]{spaces}
\externaldocument[spaces-properties-]{spaces-properties}
\externaldocument[spaces-morphisms-]{spaces-morphisms}
\externaldocument[decent-spaces-]{decent-spaces}
\externaldocument[spaces-cohomology-]{spaces-cohomology}
\externaldocument[spaces-limits-]{spaces-limits}
\externaldocument[spaces-divisors-]{spaces-divisors}
\externaldocument[spaces-over-fields-]{spaces-over-fields}
\externaldocument[spaces-topologies-]{spaces-topologies}
\externaldocument[spaces-descent-]{spaces-descent}
\externaldocument[spaces-perfect-]{spaces-perfect}
\externaldocument[spaces-more-morphisms-]{spaces-more-morphisms}
\externaldocument[spaces-flat-]{spaces-flat}
\externaldocument[spaces-groupoids-]{spaces-groupoids}
\externaldocument[spaces-more-groupoids-]{spaces-more-groupoids}
\externaldocument[bootstrap-]{bootstrap}
\externaldocument[spaces-pushouts-]{spaces-pushouts}
\externaldocument[spaces-chow-]{spaces-chow}
\externaldocument[groupoids-quotients-]{groupoids-quotients}
\externaldocument[spaces-more-cohomology-]{spaces-more-cohomology}
\externaldocument[spaces-simplicial-]{spaces-simplicial}
\externaldocument[spaces-duality-]{spaces-duality}
\externaldocument[formal-spaces-]{formal-spaces}
\externaldocument[restricted-]{restricted}
\externaldocument[spaces-resolve-]{spaces-resolve}
\externaldocument[formal-defos-]{formal-defos}
\externaldocument[defos-]{defos}
\externaldocument[cotangent-]{cotangent}
\externaldocument[examples-defos-]{examples-defos}
\externaldocument[algebraic-]{algebraic}
\externaldocument[examples-stacks-]{examples-stacks}
\externaldocument[stacks-sheaves-]{stacks-sheaves}
\externaldocument[criteria-]{criteria}
\externaldocument[artin-]{artin}
\externaldocument[quot-]{quot}
\externaldocument[stacks-properties-]{stacks-properties}
\externaldocument[stacks-morphisms-]{stacks-morphisms}
\externaldocument[stacks-limits-]{stacks-limits}
\externaldocument[stacks-cohomology-]{stacks-cohomology}
\externaldocument[stacks-perfect-]{stacks-perfect}
\externaldocument[stacks-introduction-]{stacks-introduction}
\externaldocument[stacks-more-morphisms-]{stacks-more-morphisms}
\externaldocument[stacks-geometry-]{stacks-geometry}
\externaldocument[moduli-]{moduli}
\externaldocument[moduli-curves-]{moduli-curves}
\externaldocument[examples-]{examples}
\externaldocument[exercises-]{exercises}
\externaldocument[guide-]{guide}
\externaldocument[desirables-]{desirables}
\externaldocument[coding-]{coding}
\externaldocument[obsolete-]{obsolete}
\externaldocument[fdl-]{fdl}
\externaldocument[index-]{index}

% Theorem environments.
%
\theoremstyle{plain}
\newtheorem{theorem}[subsection]{Theorem}
\newtheorem{proposition}[subsection]{Proposition}
\newtheorem{lemma}[subsection]{Lemma}

\theoremstyle{definition}
\newtheorem{definition}[subsection]{Definition}
\newtheorem{example}[subsection]{Example}
\newtheorem{exercise}[subsection]{Exercise}
\newtheorem{situation}[subsection]{Situation}

\theoremstyle{remark}
\newtheorem{remark}[subsection]{Remark}
\newtheorem{remarks}[subsection]{Remarks}

\numberwithin{equation}{subsection}

% Macros
%
\def\lim{\mathop{\mathrm{lim}}\nolimits}
\def\colim{\mathop{\mathrm{colim}}\nolimits}
\def\Spec{\mathop{\mathrm{Spec}}}
\def\Hom{\mathop{\mathrm{Hom}}\nolimits}
\def\Ext{\mathop{\mathrm{Ext}}\nolimits}
\def\SheafHom{\mathop{\mathcal{H}\!\mathit{om}}\nolimits}
\def\SheafExt{\mathop{\mathcal{E}\!\mathit{xt}}\nolimits}
\def\Sch{\mathit{Sch}}
\def\Mor{\mathop{\mathrm{Mor}}\nolimits}
\def\Ob{\mathop{\mathrm{Ob}}\nolimits}
\def\Sh{\mathop{\mathit{Sh}}\nolimits}
\def\NL{\mathop{N\!L}\nolimits}
\def\CH{\mathop{\mathrm{CH}}\nolimits}
\def\proetale{{pro\text{-}\acute{e}tale}}
\def\etale{{\acute{e}tale}}
\def\QCoh{\mathit{QCoh}}
\def\Ker{\mathop{\mathrm{Ker}}}
\def\Im{\mathop{\mathrm{Im}}}
\def\Coker{\mathop{\mathrm{Coker}}}
\def\Coim{\mathop{\mathrm{Coim}}}

% Boxtimes
%
\DeclareMathSymbol{\boxtimes}{\mathbin}{AMSa}{"02}

%
% Macros for moduli stacks/spaces
%
\def\QCohstack{\mathcal{QC}\!\mathit{oh}}
\def\Cohstack{\mathcal{C}\!\mathit{oh}}
\def\Spacesstack{\mathcal{S}\!\mathit{paces}}
\def\Quotfunctor{\mathrm{Quot}}
\def\Hilbfunctor{\mathrm{Hilb}}
\def\Curvesstack{\mathcal{C}\!\mathit{urves}}
\def\Polarizedstack{\mathcal{P}\!\mathit{olarized}}
\def\Complexesstack{\mathcal{C}\!\mathit{omplexes}}
% \Pic is the operator that assigns to X its picard group, usage \Pic(X)
% \Picardstack_{X/B} denotes the Picard stack of X over B
% \Picardfunctor_{X/B} denotes the Picard functor of X over B
\def\Pic{\mathop{\mathrm{Pic}}\nolimits}
\def\Picardstack{\mathcal{P}\!\mathit{ic}}
\def\Picardfunctor{\mathrm{Pic}}
\def\Deformationcategory{\mathcal{D}\!\mathit{ef}}

\begin{document}

\title{Moduli Spaces of Sheaves}
\maketitle

Minicourse by Cristina Manolache, CIMPA school Florianópolis 2025.

Notes at 
\href{http://github.com/danimalabares/cimpa-floripa}{github.com/danimalabares/cimpa-floripa}

\bigskip\noindent

{\bf Abstract.} This course will be an introduction to Moduli Spaces of vector
bundles. A moduli space of stable vector bundles on a smooth, algebraic variety
X is a scheme whose points are in “natural bijection” to isomorphic classes of
stable vector bundles on X. Using Geometric Invariant Theory the moduli space
can be constructed as a quotient of certain Quot-scheme by a natural group
action. We introduce the crucial concept of stability of vector bundles over
smooth projective varieties and we give a cohomological characterization of the
(semi)stability. The notion of (semi)stability is needed to ensure that the set
of vector bundles one wants to parameterize is small enough to be parameterized
by a scheme of fniite type. We introduce the formal defniition of moduli
functor, fnie moduli space and coarse moduli space and we recall some
generalities on moduli spaces of vector bundles. Then we focus on vector bundles
on algebraic surfaces. Quite a lot is known in this case and we will review the
main results, some of which will illustrate how the geometry of the surface is
refelcted in the geometry of the moduli space. Going beyond surfaces, we
introduce the notion of monad, which allow the classifciation of vector bundles
on, for instance, $\mathbb{P}^3$. Monads appeared in a wide variety of contexts
within algebraic geometry, and they are very useful when we want to construct
vector bundles with prescribed invariants like rank, determinant, Chern classes,
etc.  Finally, we will study moduli spaces of vector bundles on higher
dimensional varieties. As we will stress, the situation drastically difefrs and
results like the smoothness and irreducibility of moduli spaces of stable vector
bundles on algebraic surfaces turn to be false for moduli spaces of stable
vector bundles on higher dimensional algebraic varieties.  

\bigskip\noindent
\tableofcontents

\bigskip\noindent

\section{Introduction}
\label{section-introduction}

The goal is the following. $X$ smooth projective variety over $\mathbb{C}$. We
want to define a moduli space of sheaves up to some kind of isomorphism. Also
give some discrete invariants. And give structure of scheme which ``respects
limits''.

{\bf Functor of points.} For a given scheme $X$, the set
$\text{Hom}(\text{Spec}\mathbb{C},X)$ is the set of points of $X$. This motivates
defining a functor $h_X:\text{Sch}\to \text{Sets}$ by $h_X(S)=\text{Hom}(S,X)$,
and morphisms are just mapped via composition.

And there's more: given a morphism of schemes $\varphi:X\to Y$ we can produce
 $h_\varphi:h_X \to h_Y$.

\begin{slogan}
Schemes are functors.
\end{slogan}

\begin{definition}
\label{definition-representable-functor-by-scheme}
A functor $F$ is called {\it representable (by a scheme)} if there is a scheme
$X$ such that $F\cong h_X$ and we say $X$ is a {\it fine moduli space for $X$}.
\end{definition}

\begin{remark}
\label{remark-uniqueness-of-representable-functor}
If $X$ exists it is unique.
\end{remark}

\begin{example}
\label{example-quotient-sheaf}
$X$ scheme,
$$
\text{Quot}_X:(\text{Sch})\to (\text{Sets})
$$
$$
\text{Quot}_X(S)=\{
\mathcal{O}_{X \times S}^{\oplus n}\to\mathcal{F}
:\mathcal{F} \text{ is a flat sheaf}\}\Big/ \sim
$$
and morphisms are mapped by pullback.
\end{example}

Here, ``flat'' accounts for a continuously varying family of copies of $X$
over $S$, i.e. the product $X \times S$.

\begin{theorem}
\label{theorem-Quot-is-represented}
$\text{Quot}_X$ is represented by a projective scheme.
\end{theorem}

Almost all examples of moduli spaces are quotients.

\medskip\noindent

The one example of a moduli space everybody knows is…

\begin{example}[Projective space is a quotient]
\label{example-projective-space-is-a-quotient}
Take $X=$ point, and then
$$
F(S)=\{\mathcal{O}_S^{\oplus n+1}\to \mathcal{L}\}\cong h_{\mathbb{P}^n}
$$
where $\mathcal{L}$ is a line bundle on $S$.
\end{example}

\begin{definition}
\label{definition-universal-family}
If $F$ is represented by a scheme $X$ and $\eta:F(S)\to h_X(S)$ in part we have
a bijection
\begin{align*}
F(X) &\longrightarrow h_X(X):=\text{Hom}(X,X) \\
U &\longmapsto \text{id}
\end{align*}
\end{definition}

\bigskip\noindent

Again, flat means continuously varying. Consider first
$$
\mathcal{M}(S)=\{\mathcal{F}\text{a fheaf on $X\times S$ flat over }S\}/\sim
$$
and
$$
\mathcal{M}(f:T \to S)=f^*
$$
and equivalence is given by: $\mathcal{F}\sim \mathcal{G}$ iff there exists a
line bundle $\mathcal{L}$ on $S$ such that 
$\mathcal{F}\cong \mathcal{G}\otimes\pi^*S$.

\medskip\noindent
{\bf Problem.} $X:=\mathbb{P}^1$ and rank 2 sheaves on $\mathbb{P}^1$. In this
case […]

\medskip\noindent
{\bf Fact.} Fine moduli spaces are rare.

\begin{definition}
\label{definition-coarse-moduli-space}
A {\it coarse moduli space} of  $X$ is a scheme $X$ with a natural
transformation $\eta:F \to h_X$ such that
\begin{itemize}
\item $\eta_{\eta \times \mathbb{C}}:F(hec \mathbb{C}) \to h_X(hec \mathbb{C}$
is hij.
\item For all $Y$ and $F \to h_T$ we have [$F \to h_Y$, $F\to h_X$, $\exists !
h_X \to h_Y$.
\end{itemize}
\end{definition}

\medskip\noindent
{\bf Summary.}
\begin{enumerate}
\item Schemes are functors.
\item The best moduli spaces are represented functors, they are called fine.
These have universal families, and all are pullbacks of from the
universal family.
\end{enumerate}

\section{Discrete data}
\label{section-discrete-data}

$X$ projective and $H$ an ample divisor on $X$. The {\it Hilbert polynomial} is
$$
P(\mathcal{F},m):=\chi(\mathcal{F}\otimes \mathcal{O}(m))
$$

\medskip\noindent
{\bf Fact.} $P(\mathcal{F},m)=\sum_{i=0}^{\dim
X}\alpha_i(\mathcal{F})\frac{m^i}{i!}$.

The {\it rank of $\mathcal{F}$} is
$\text{rk}(\mathcal{F}):=\frac{\alpha_d(\mathcal{F})}{\alpha_d(\mathcal{O}_X)}$.
And $\alpha_d(\mathcal{O}_X)$ is the {\it degree} of $\mathfrak{X}$ w.r.t
$\mathcal{O}(1)$.

The {\it reduced Hilbert polynomial} is
$P(\mathcal{F},m):=\frac{P(\mathcal{F},m)}{\alpha_d(\mathcal{F})}$. The {\it
slope} of $\mathcal{F}$ is
$\mu_H(\mathcal{F}):=\frac{c_1(\mathcal{F})H^{d-1}}{\text{rk}\mathcal{F}}$.

\begin{example}[Curve of genus $g$ using Riemann-Roch]
\label{example-curve-of-genus-g-using-Riemann-Roch}
(Missing.)
\end{example}

\medskip\noindent
Here begin the contents of the second lecture.
\begin{definition}
\label{definition-Gieseker-and-mu-stability}
$\mathcal{F}$ is {\it Gieseker/$\mu$-semistable} w.r.t. $H$ if
\begin{itemize}
\item $P_\varepsilon(\mathfrak{m})\leq P_{\mathcal{F}}(\mathfrak{m})$ for all $0
\neq  \varepsilon \subset \mathcal{F}$ proper subsheaf.
\item $\mu_{M}(\varepsilon) \leq  \mu_{M}(\mathcal{F})$ of rank $0<r \leq r-1$.
\end{itemize}
If the inequality is strict then $\mathcal{F}$ is called {\it stable}.
\end{definition}

\begin{example}
\label{example-G-and-mu-stability-coincide-in-curve}
$X$ a curve, $d=\text{deg}\mathcal{F}$, $r=\text{rk}\mathcal{F}$,
$p=\text{deg}Hm+\frac{d}{r}+1-g$. Then $\mu$-stability is the same as
G-stability.
\end{example}
In general:
$$
\mu\text{-stability}\implies \text{ G-stability }\implies \text{G semi-stability
}\implies \mu \text{ semi-stability}
$$
\begin{remark}
\label{remark-stability-depends-on-H}
Stability depends on the choice of $H$.
\end{remark}

\begin{example}[Stability depends on choice of ample divisor]
\label{example-stability-depends-on-choice-of-ample-divisor}
$X=\mathbb{P}^1\times \mathbb{P}^1$, then $\text{Pic}(X)
=\mathbb{Z}[\ell]\oplus\mathbb{Z}[m]$ where $\ell^2=m^2=0$ and $\ell m=1$.
Consider $\mathcal{F}$ given by
\begin{equation}
\label{equation-extensions}
\xymatrix{
0\ar[r]&\mathcal{O}_X(\ell-3m)\ar[r]&\mathcal{F}\ar[r]&\mathcal{O}_X(3m)
\ar[r]&0
}
\end{equation}
Then $c_1(\mathcal{F})=\ell$ and $c_2(\mathcal{F})=(\ell-3m)(3m)=3$. Then
$L=\ell+sm$; $L'=\ell+7m$. We will show
that $\mathcal{F}$ is semistable w.r.t. $L$ but not w.r.t. $L'$. We do have
non-trivial extensions as in the sequence \ref{equation-extensions}. These are
parametrized by 
$$
\text{Ext}^1(\mathcal{O}_X(3m)), \mathcal{O}_X(-3m)\cong
H^{1}(X,\mathcal{O}_X(\ell-6m))\cong \ldots \cong \mathbb{C}^{10}
$$
So it is semi-stable w.r.t. $L$.

Suppose $\mathcal{O}(s) \subset \mathcal{F}$. We want to show that
$\mu(\mathcal{O}(s)) \leq  \mu(\mathcal{F})$. 

We claim that we have either $\mathcal{O}(s)\hookrightarrow \mathcal{O}(3m)$ or
$\mathcal{O}(s)\hookrightarrow \mathcal{O}(\ell-3m)$. Claim was proved and
further computations showed the required inequality, so $\mathcal{F}$ is
$\mu$-semistable w.r.t. $L$. On the other side, $\mathcal{F}$ is not semistable
w.r.t.  $L'$ since $\mathcal{O}(\ell-3m) \hookrightarrow  \mathcal{F}$ (so here
I realise that probably taking Chern class will be monotonous with respect to
this inclusion…) which gives
$\mu_{L'}(\mathcal{O}(\ell-3m))=(\ell-3m)(\ell+7m)=4\not \leq \frac{7}{2}$.
\end{example}

\begin{example}[Results on $\mu$-stability]
\label{example-results-on-mu-stability}
\begin{itemize}
\item All line bundles are stable.
\item If $0\to L_0 \to \mathcal{F} \to L_1 \to 0$ is a non-trivial extension
with degree $L_0=0$, $\text{deg}(L_1)=1$, $L_i$ line bundles, then
$\mathcal{F}$ is stable.
\item $E_i$ semi-stable shows $E_1\oplus E_2$ is semistable if and only if
$\mu(E_i)=\mu(E_2)$.
\item $E$ is $\mu$-semistable if and only if for all $L$ line bundles $E \otimes
L$ is semistable.
\item $E$ is semistable then $E^{\vee}$ is semi-stable.
\item $E_1,E_2$ semistable then $E_1 \otimes E_2$ is semistable.
\end{itemize}
\end{example}

Now we explain some {\bf easy criteria.} (We shall find a
condition on cohomology that is equivalent to stability.)

\begin{definition}
\label{definition-norm-sheaf}
$\mathcal{F}$ a reflexive sheaf on $\mathbb{P}^n$ of rank $r$. We define
$\mathcal{F}_{\text{norm}}:=\mathcal{F}(k)$ where $k$ is the unique integer such
that $c_1(\mathcal{F}(k))\in \{-r+1,\ldots,c\}$.
\end{definition}

\begin{proposition}[Cohomological characterization of stability]
\label{proposition-cohomological-characterization-of-stability}
$\mathcal{F}$ reflexive sheaf of rank  $2$, then $\mathcal{F}$ is $\mu$-stable
if and only if $H^{0}(\mathbb{P}^n,\mathcal{F}_{\text{norm}})=0$.
\end{proposition}

\begin{proof}
$\mathcal{F}$ vector bundle of rank $r$ on a smooth projective variety with
$\text{Pic}=\mathbb{Z}$.
\begin{itemize}
\item If $H^{0}(X,(\Lambda^{z}\mathcal{F})_{\text{norm}})=0$ for all $z \in
\{1,\ldots,r-1\}$ then $\mathcal{F}$ is $\mu$-stable.
\item If $H^{0}(X,\Lambda^{z}\mathcal{F})_{\text{norm}}(-1))=0$ for all $z \in
\{1,\ldots,r-1\}$, then $\mathcal{F}$ is $\mu$-semi-stable.
\end{itemize}
\end{proof}

\begin{example}
\label{example-Omega1-on-projective-space-is-stable}
$\Omega_{\mathbb{P}^n}$ is stable, and also $T_{\mathbb{P}^1}$. The cohomologies
vanish using Bott.
\end{example}

\begin{example}
\label{example-lots-of-Os}

\end{example}

{\bf Filtrations.} $H$ ample, $\mathcal{F}$ torsion free sheaf, then there
exists a unique filtration $0 \hookrightarrow  \mathcal{F}_0\subset
\mathcal{F}_1\subset\ldots\subset\mathcal{F}_m=\mathcal{F}$ with
$E_i=\mathcal{F}_i/\mathcal{F}_{i+1}$ $\mu$-semi-stable w.r.t. $H$ and
$\mu(E_1)>\mu(E_2)>\ldots>\mu(E_m)$. This is called {\it Harder-Narorimnan
filtration}.

Given any Chern classes, can we guarantee the existence of a sheaf with these
Chern classes? No.

\begin{theorem}[Bogomolov Inequality]
\label{theorem-Bogomolov-inequality}
$X$ smooth, projective variety of dimension $\geq 2$, $H$ ample. If
$\mathcal{F}$ is torsion-free, $\mu$ semi-stable sheaf w.r.t. all
$$
\Delta(\mathcal{F}):=(2rc_2(\mathcal{F})-
(r-1)c_1(\mathcal{F})^2)H^{n-2}\geq 0
$$
where $r=\text{rk}(\mathcal{F})$ and $\dim X=n$. We call $\Delta$ the {\it
discriminant}.
\end{theorem}

\section{Boundedness}
\label{section-boundedness}

We need further preparation before we can define moduli spaces.

\begin{definition}
\label{definition-bounded-family}
A family of isomorphism classes of coherent sheaves on $X$ is called {\it
bounded} if there exists $S$ of {\bf finite type} and an 
$\mathcal{O}_{S \times X}$ sheaf $F$ such that the family is contained in
 $\{ \mathcal{F}_{S \times X}|s  \text{ closed in }S\}$.
\end{definition}

\begin{definition}
\label{definition-m-regularity}
A coherent sheaf is called {\it $m$-regular} if $H^{i}(X,\mathcal{F}(m-i))=0$
for all $i>0$.
\end{definition}

\begin{proposition}
\label{proposition-m-regularity}
If $\mathcal{F}$ is $m$-regular,
\begin{itemize}
\item $F$ is $m'$ regular for all $m' \geq m$.
\item $F(m)$ is globally generated.
\end{itemize}
\end{proposition}

\begin{definition}
\label{definition-regular-infimum}
$\text{reg}\mathcal{F}:
=\text{inf}\{m \in \mathbb{Z}:\mathcal{F}\text{ is $m$-regular}\}$
\end{definition}

\begin{lemma}
\label{lemma-}
The following are equivalent:
\begin{itemize}
\item $\{\mathcal{F}_i\}_{i \in I}$ is bounded
\item The set of Hilbert polynomials $\{P(\mathcal{F}_i)\}_{i \in I}$ is finite
and there exists a bound $\rho$ such that $\text{reg}\mathcal{F}_i\leq \rho$ for
all $i \in I$.
\end{itemize}
\end{lemma}

The following is a difficult result that we shall only state and use.

\begin{theorem}
\label{theorem-}
$X$ projective, $H$ ample, $P$ a fixed polynomial of degree $d$. The family of
torsion-free sheaves with Hilbert polynomial $P$ is bounded.
\end{theorem}

\begin{proposition}
\label{proposition-semi-stable-sheaves-are-bounded-on-curves}
Let $X$ be a curve of genus $g$ with fixed Hilbert polynomial. The semi-stable
sheaves on $X$ are bounded.
\end{proposition}

\begin{proof}
Let $E$ be a semis-stable sheaf. We want $m$ such that $H^{1}(X,E(m-1))$. We
apply Serre Duality as follows:
$$
H^{1}(X,E(m-1))\overset{\text{S.D.}}{=}\text{Hom}(E,\omega(1-m))^\vee
$$Then $E \to \omega(m+1)$ $\implies$ $\omega^\vee(m-1)\to E^\vee$. $E$ is
semi-stable then $-\text{deg}(\omega^\vee)\leq \frac{\text{deg}(E^\vee)}{r}$
which implies that $-2g+2+m-1\leq -\frac{d}{r}$ for $d=\text{deg}E$, and then
$m\leq  2g-2-\frac{d}{r}+1$. If we take $m>2g-2\frac{d}{r}+1$.
\end{proof}

\begin{proposition}
\label{proposition-}
If  $E$ is a stable sheaf, then $\text{Hom}(E,E)=\mathbb{C}$.
\end{proposition}

\begin{proof}
If $f:E \to E$ factors through $E \to F \to F$ and is not surjective, then
$\mu(E)<\mu(F)<\mu(E)$ $\implies$ $f$ is injective. As an exercise, prove that
$E$ is stable if and only if for all $E \to F$ surjective we have that
$\mu(E)<\mu(F)$.
\end{proof}

\section{Construction of the moduli space of semi-stable sheaves}
\label{section-construction-of-the-moduli-space-of-semi-stable-sheaves}
$$
\mathcal{M}^p(s)=\left\{ \substack{\text{$\mathcal{F}$ on $X\times S$,
pet over $S$}
 \\ \text{with Hilbert polynomial $P$}} \right\} \Big/\sim
$$
where $\mathcal{F}\sim \mathcal{G}$ $\iff$ there exists $L$ on $S$ such that
$\mathcal{F}\otimes \pi^*L \cong \mathcal{G}$ where 
$X \times S\xrightarrow{\pi}\mathcal{G}$.

$\mathcal{F}$ is a semi-stable sheaf is  $\exists $ $m$ such that
$\mathcal{F}$ is $m$-regular, which implies that $\mathcal{F}(m)$ is globally
generated.

\medskip\noindent 
{\bf Warning.} Following are several formulas I don't understand and just copied
from the board as they were.
 $$
\dim H^{0}(X,\mathcal{F}(m))=P(m)
$$
$$
H^{i}(X,\mathcal{F}(m-1))=0\qquad \forall i>0
$$
$$
V=\mathcal{O}_X^{\oplus P(m)}\to \mathcal{F}(m)\to 0
$$
$$
\implies \left[ \mathcal{O}_X^{\oplus P(m)}(-m) \to \mathcal{F} \to 0 \right]
\in \text{Quot}(\mathcal{H},P)
$$
Let $R$ be open locus in $\text{Quot}(\mathcal{H},P)$ such that
\begin{itemize}
\item $\mathcal{F}$ is semi-stable.
\item $H^{0}(\mathcal{H}(m))\cong H^{0}(\mathcal{F}(m))$.
\end{itemize}
$\text{GL}(V)\mathbb{y} R$ by $g[\rho]=\pi \circ g$ for $\rho \in R$. 
$$
\xymatrix{
\mathcal{O}^{\oplus p(m)}\ar[r]^{\rho}\ar[d]_{g,\cong}&  \mathcal{F}\\
\mathcal{O}_X^{\oplus p(m)}\ar[ur]
}
$$
We want
``$R/\text{GL}(V)$''.

\begin{theorem}
\label{theorem-exists-coarse-moduli-space-of-semi-stable-sheaves-with-fixed
-Hilbert-polynomial}
There exists a coarse moduli space $M_{X,P}$ of semi-stable sheaves with fixed
Hilbert polynomial $P$, $\mathcal{M}_{X,P}$ is projective and
$\mathcal{M}_{X,P}^s$ of stable sheaves with Hilbert polynomial $P$.
\end{theorem}

\medskip\noindent
{\bf Fact.} The GIT (semi-)stability coincides with the Gieseker
(semi-)stability (see \cite{HL}).

\medskip\noindent
{\bf Remarks on working with stacks.}

\begin{itemize}
\item Instead of constructing sheaves which are coarse/fine moduli spaces we may
find more general geometric objects called ``stacks''.
\item Advantage: Quotients always exist: 
$G \mathbb{y} X \rightsquigarrow [X/G]$.
\item Stacks have universal families.
\item Disadvantage: we need to work with more general moduli functors. Instead
of $\mathcal{M}:(\text{Sch}) \to (\text{Sets})$, we substitute $(\text{Sets})$
with a 2-category.
\item Disadvantage: not all algebraic stacks have coarse moduli spaces but if
the number of automorphisms of points [is finite?].
\item Disadvantage: may not be projective (even if proper/compact).
\end{itemize}

In sum, we may not be able to produce a moduli space, and even if we can, it may
not be projective.

\medskip\noindent
Even in the easiest situation we can have, we don't have universal family:

\begin{example}[The line bundles of degree $d$ on $\mathbb{P}^1$ do not have
universal family]
\label{example-moduli-space-of-line-bundles-of-degree-d-over-P1-
do-not-have-universal-family}
Line bundles on $\mathbb{P}^1$ of degree $d$ are $\mathcal{O}(d)$. Them
$\mathcal{M}=\text{Spec}\mathbb{C}$. It does not have a universal family:
$$
\xymatrix{
\mathcal{O}_{\mathbb{P}^1}\ar[r]\ar[d]&\mathbb{C}\ar[d]\\
\mathbb{P}^1\ar[r]&\text{Spec}\mathbb{C}
}
$$
and $\mathcal{O}(d) \neq \mathcal{O}_{\mathbb{P}^1}$, $d\neq 0$.
\end{example}


\section{Remarks on stacks}
\label{section-remarks-on-stacks}

The following was not defined properly. As a stack,
$$
\text{Pic}_{\mathbb{P}^1,d}=\left[ \cdot/\mathbb{G}_m \right] = B\mathbb{G}_m
$$
parametrized by $\mathbb{G}_m$-bundles.
$$
\xymatrix{
\mathcal{L}\setminus\{\text{zero section}\}\ar[r]\ar[d]&\Gamma^A[?]\ar[d]\\
\mathbb{P}^1\ar[r]_\varphi&B\mathbb{G}_m
}
$$
\medskip\noindent
{\bf Tangent space.} The tangent space of $\mathcal{M}_{X,P}$, in $\ni E$, is
$\text{Ext}^1(E,E)$ if $\text{Ext}^2(E,E)=0 \implies  \mathcal{M}_{X,P}$ is
smooth. ($\text{Ext}$ groups are defined in Stacks Project, Algebra; this result
 also appears to be a theorem in deformation theory.)

\begin{example}
\label{example-moduli-space-of-curves-is-smooth}
Let $X$ be a curve. $E$ is a vector bundle of rank $n$, then
$$
\text{Ext}^2(E,E)=H^{2}(X,E \otimes E^{\vee})=0
$$
then the moduli space is smooth.
\end{example}

The dimension of the stack is $\dim T_{\mathcal{M},E}-\dim \text{Aut}(E)$, which
is just $\dim (\text{Ext}^1(E,E) - \dim \text{Ext}^0(E,E)$, which in turn says
that $\mathcal{X}(X,E \otimes E^\vee)=V^2(f-g)$.

\medskip\noindent
{\bf Question 2.} Once we have a moduli space we want so study its geometry. 
What can we say about $\mathcal{M}_{X,P,H}$?

Let $X$ be a projective surface and $E$ a vector bundle. 
Let $r = \text{rk} E$, $L = \Lambda^{n}E$. Consider
$\mathcal{M}_{X,P,H}(r,L,n)$.
$n=c_2(E)$ 

\begin{theorem}
\label{theorem-moduli-space-that-is-normal-generaically-smooth-irreducible-
quasi-projective-variety}
If $\Delta=2rm-(r-1)L^2 \gg 0$ then $\mathcal{M}_{X,H}(r,L,n)$ is a normal,
generically smooth irreducible quasi-projective variety. (If $\Delta$ is small
then it is not true.)
\end{theorem}

\begin{theorem}
\label{theorem-birrational-moduli-spaces}
Fix $H,H'$ ample. For $\Delta \gg 0$, $\mathcal{M}_{X,H}(r,L,n)$ and
$\mathcal{M}_{X,H}(r,L,n)$ are birrational.
\end{theorem}

\medskip\noindent
Intuitive idea of what follows: there are ample cones, and walls between them, 
and when we cross the
wall, we get to an object birational to the one we were previously at.

\begin{definition}
\label{definition-wall}
Let $C_X$ be the ample cone $\mathbb{R} \otimes \text{Num}(X)$. For 
$\xi \in\text{Num}(X)$, 
$W^\xi:=C_X \cap \{x \in \text{Num}(X) \otimes\mathbb{R}:x\cdot\xi=0\}$. 
$W^\xi$ is called a {\it wall} of type 
$(c_1,c_2) \iff \exists G \in\text{Pic}(X)$ with $\xi = G$ such that 
\begin{itemize}
\item $G+c_1$ is divisible by $2$ in $\text{Pic}(X)$
\item $c_1^2-4c_2 \leq G^2 <0$.
\end{itemize}
\end{definition}

\begin{remark}
\label{remark-wall-not-zero-if-exists-ample-with-condition}
$W^\xi \neq 0$ if there exists an ample line bundle $L$ with $L\xi =0$,
$W(c_1,c_2)=\bigcup_{\xi}W^1$.
\end{remark}

\begin{definition}
\label{definition-chamber}
A {\it chamber} is a connected component of $C_X\setminus W(c_1,c_2)$. (Perhaps
this is the same as $C_X \setminus \bigcup_{\xi}W\xi$.)
\end{definition}

\begin{theorem}[Qin]
\label{theorem-Qin}
The moduli space $M_{X,H}(2c_1,c_2)$ of rank 2 locally free sheaves only depends
on the chamber of $H$.
\end{theorem}

\begin{remark}
\label{remark-remarks}
\begin{itemize}
\item $c_1^2-4c_2<0$ is Bogomolov inequality.
\item  $G^2<0$ is Hodge index theorem
\item $F \subset E$ destabilizing line bundle $F$ $\iff$ $H$  
$c_1(F)>\frac{c_1(E)H}{2}$ $\iff$ $(\underbrace{2c_1(F)-c_1(E)}_{=G})H>0$
$\implies$ $2c_1(F)=c_1(G)+c_1(E) \implies c_1(G)+c_1(E)$ is divisible by 2.
\item Suppose we have $0 \to F \to E \to M \otimes |_{Z}\to 0$ with $M$ a line
bundle, $Z$ codimension 2 in $X$.
\end{itemize}
\end{remark}

[Lots of computations I missed.]

\begin{exercise}
\label{exercise-P1timesP1}
$X=\mathbb{P}^1\times \mathbb{P}^1$, $E$ of rank 2, $c_1=\ell$, $c_2=3$.
$G=a\ell+bm$, $G^2=2ab \implies ab<0$, $\Delta=c_1^2-4c_2\leq G^2=ab\implies
-12\leq 2ab<0$, $(a+1)\ell+bm$ divisible by $2$, then $a+1$ is even.

I understand the walls and chambers were computed:
$L_0=(\ell+7m)\in C_0$,
$L_1=(\ell+5m) \in C_1$ and $L_2=(\ell+3m) \in C_2$.

We have
 \begin{itemize}
\item For $L$ in  $C_0$, $\mathcal{M}(2,\ell,3)=\emptyset$. (Proved.)
\item For $L$ in $C_1$, $\mathcal{M}(2,\ell,3)\cong \mathbb{P}^5$.
\item For $L$ in $C_2$, $\mathcal{M}(2,\ell,3)$ is an open in $\mathbb{P}^3$.
\end{itemize}
\end{exercise}

For more general rank:

\begin{theorem}
\label{theorem-general-rank}
$F \subset E$ a subsheaf of rank $r'<r$, $\mu_H(F)=\mu_H(E)$, then 
$\xi=rc_1(F)-r'c_1(E)$ satisfies $\xi H=0$ and 
$\frac{r^2}{4}\Delta \leq \xi^2<0$, with $e_2$ $\iff$ $\xi=0$.
\end{theorem}

\begin{theorem}
\label{theorem-smooth-irreducible-projective-surface-moduli-are-birrational}
$X$ smooth irreducible projective surface, $H,H'$ ample and $\Delta\gg 0$. Then
$\mathcal{M}_{X,H}$ and $\mathcal{M}_{X,H'}$ are birational (any rank).
\end{theorem}

For dimension $\geq 3$, the moduli space of sheaves is in general smooth or
irreducible.

\begin{theorem}[Ein]
\label{theorem-Ein}
Let $C$ be a smooth projective 3-fold, $c_1,H \in \text{Pic}(X)$, $H$ ample.
Assume there exists $a,b$ such that $ac_1=bH$. $\mathcal{M}_{X,H}$ rank 2 vector
bundle on $X$, $\mu$ stable with respect to $d:=c_2H$. $\mu(d)=F$ irreducible
component of $\mathcal{M}_{X,H}(c_1,c_2)$, $\implies $ 
$\lim_{d \to \infty} m(d)=\infty$.
\end{theorem}

\begin{example}
\label{example-}
In DT (Donaldson-Thomas) theory, $X$ 3-fold,
$$
I(X,P)=\left\{Z \subset X: \substack{Z \text{ a curve with} \\ 
\text{fixed Hilbert polynomial}}\right\}
$$
$P(u)=3u+1$ twisted cubics have  $P(u)=3u+1$. But genus 1 curves with a
point also have  $P(u)=3u+1$. $I(\mathbb{P}^3,3n+1)$ has several components.
\end{example}

\bibliography{my}
\bibliographystyle{amsalpha}

\end{document}
