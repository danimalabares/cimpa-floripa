\input{preamble}
\begin{document}

\title{Kummer Type Hyperkähler Varieties}
\maketitle

Minicourse by Xiaolei Zhao, CIMPA school Florianópolis 2025.

Notes at 
\href{http://github.com/danimalabares/cimpa-floripa}
{github.com/danimalabares/cimpa-floripa}

\bigskip\noindent

{\bf Abstract.} Examples of noncommutative K3 surfaces arise from semiorthogonal
decompositions of the bounded derived category of certain Fano varieties. The
most interesting cases are those of cubic fourfolds and Gushel-Mukai varieties
of even dimension. Using the deep theory of families of stability conditions,
locally complete families of hyperkähler manifolds deformation equivalent to
Hilbert schemes of points on a K3 surface have been constructed from moduli
spaces of stable objects in these noncommutative K3 surfaces. On the other hand,
an explicit description of a locally complete family of hyperkähler manifolds
deformation equivalent to a generalized Kummer variety is not available from
classical geometry. In this lecture series, we will construct families of
noncommutative abelian surfaces as equivariant categories of the derived
category of K3 surfaces which specialize to Kummer K3 surfaces. Then we will
explain how to induce stability conditions on them and produce examples of
locally complete families of hyperkähler manifolds of Kummer type. Based on
joint work with Arend Bayer, Alex Perry and Laura Pertusi. 

\bigskip\noindent
\tableofcontents
\bigskip\noindent

\noindent

\section{K3 surfaces and Hyperkähler varieties}
\label{section-K3-surfaces-and-hyperkahler-varieties}

\begin{definition}
\label{definition-K3-surface}
A {\it K3 surface} (over $\mathbb{C}$) is a smooth projective surface $S$ such
that $\omega_S \cong \mathcal{O}_S$ and $\pi_1(S)=1$.
\end{definition}

\begin{example}[K3 surfaces]
\label{example-K3-surfaces}
\begin{enumerate}
\item $S\subset \mathbb{P}^3$ of degree 4. A straightforward computation shows
the conditions of the definition are verified.
\item $S \to \mathbb{P}^2$ double cover ramified along a degree 6 curve.
\end{enumerate}
\end{example}

\begin{definition}
\label{definition-hyperkahler-variety}
A {\it hyperkähler} variety is a smooth projective variety $X$ such that
\begin{enumerate}
\item $\pi_{1}(X)=1$ 
\item $H^{0}(X,\Omega_X^2)=\mathbb{C} \omega$ (i.e. the space of global
holomorphic 2-forms is one dimensional) where $\omega$ is holomorphically
symplectic.
\end{enumerate}
\end{definition}

\begin{remark}
\label{remark-hyperkahler}
If $X$ is hyperkähler, it must be even dimensional and $\omega_X=\mathcal{O}_X$.
\end{remark}

\begin{remark}
\label{remark-Beauville-Bogomolov}
We just mention the name of Beauville-Bogomolov theorem.
\end{remark}

For a while people were looking for examples of hyperkähler varieties.

\begin{example}[Hyperkähler varieties]
\label{example-hyperkahler}
\begin{enumerate}
\item K3 surface $S$.
\item $X=S^{[n]}$ Hilbert scheme of $n$ points on K3 surface. (Moduli space of
stable sheaves on K3 $S$ of rank 1,  $c_1=0$, $c_2=0$.)
\item $\mathcal{M}_H(s,v)=$ moduli of $H$-stable sheaves on $S$ of class $v$.
($v$ primitive, $H$ is $v$-generic.) (Recall the example in Cristina's course
where we studied how the moduli changes under changes in the polarization $H$.)
Let $[E] \in \mathcal{M}_H(s,v)$. We have the Yoneda map
 $$
T_{[E]}\mathcal{M}\cong \text{Ext}^1(E,E)\times \text{Ext}^1(E,E)
\to \text{Ext}^2(E,E)\overset{\substack{S \text{ is}\\\text{ a K3}}}{\cong}\Hom(E,E)^*\cong \mathbb{C}
$$
where the first factor product of $\text{Ext}^1(E,E)\times \text{Ext}^1(E,E)$ is
associated to first arrow, the second factor to the second arrow, of the
following diagram:
$$
E \to E[1] \to E[2]
$$
This moduli space is always of Picard rank 2.
\item $Y \subset \mathbb{P}^5$ a smooth cubic fourfold.
$$
F(Y)=\{[\ell] \in \text{Gr}(2,6)|\ell \subseteq Y\}
$$
is a hyperkähler 4-fold. For a specific choice $Y_0$ of hyperkähler 4-fold in
the moduli of smooth cubic 4-folds (which is 20-dimensional) we find $F(Y_0)$ is
deformation equivalent to $S^{[2]}$ where $S$ is a K3 surface.
\end{enumerate}
\end{example}

\medskip\noindent
Here is a general picture:
\iffalse$$
\xymatrix{
\substack{\text{cubic 4-fold $Y$} \\ D^b(Y)}\ar[dr]&  &  
\substack{\text{moduli of} \\ \text{stable objects}\\ \text{in $\text{Ku}_Y$ as}
\\ \text{HK variety}}\\
&\substack{\text{sod K3} \\ \text{cat conditions} \\ \text{Ku}_Y }
\ar[ur]_{\substack{\text{stability} \\ \text{conditions}}}
$$\fi

\medskip\noindent
Let $X,Y$ be projective smooth varieties and $f:X\to Y$. Consider the functors
$$
Rf_*:D^b(X) \to D^b(Y)\qquad Lf^*:D^b(Y) \to D^b(X)
$$
for $F,G \in D^b(X)$, $F\overset{L}{\otimes}G \in D^b(X)$. {\bf Convention:} we
drop the $R$ and $L$.

\begin{definition}
\label{definition-}
Let $K \in D^b(X \times Y)$. The {\it Fourier-Mulai functor} is
\begin{align*}
\Phi_K: D^b(X) &\longrightarrow D^b(Y) \\
F &\longmapsto \text{pr}_{Y,*}(\text{pr}_X^*F \otimes K)
\end{align*}
where we are using our convention --- all functors here are derived. Here the
maps are
$$
\xymatrix{
& X \times Y\ar[dl]_{\text{pr}_X}\ar[dr]^{\text{pr}_Y}\\
X& & Y
}
$$
\end{definition}

\begin{example}
\label{example-Fourier-Mukai}
$f:X \to Y$ and $\Gamma_f \equiv X \times Y$ graph. Then
$\Phi_{\mathcal{O}_{p_f}}=f_*$.
\end{example}

\begin{theorem}[Orlov]
\label{theorem-Orlov}
If $F:D^b(X) \to D^b(Y)$ is an equivalence, then $\exists ! K \in D^b(X \times
Y)$ such that $F \cong \Phi_K$.
\end{theorem}

\begin{theorem}
\label{theorem-preserved-by-derived-equivalence}
The following are preserved by derived equivalence:
\begin{enumerate}
\item (Bondal-Orlov.) $\dim X$.
\item (Bondal-Orlov.) $\bigoplus_{m \geq 0}H^{0}(X,\pm m K_X)$.
\item $H^{*}(X,\mathbb{Q})$, and $\bigoplus_{p-q=i}H^{p,q}(X)$ for any $i \in
\mathbb{Z}$.
\end{enumerate}
\end{theorem}

\medskip\noindent
{\bf Non-trivial equivalences.} $S$ a K3 surface. $H^{2}(S,\mathbb{Z})$.
\begin{itemize}
\item (Hodge decomposition.) 
$H^{2}(S,\mathbb{C})\cong H^{2,0} \oplus H^{1,1} \oplus H^{0,2}$.
\item Pairing on $H^{2}(S,\mathbb{Z})$ given by cup product.
\end{itemize}

\begin{theorem}[Torelli for K3]
\label{theorem-Torelli-for-K3}
Two K3 surfaces $S,S'$ are isomorphic if and only if there exists a Hodge
isometry $\varphi:H^{2}(S,\mathbb{Z})\to H^{2}(S',\mathbb{Z})$ 
(i.e. an isomorphism that preserves the Hodge decomposition and the pairing).
\end{theorem}

This was the classical content.

\subsection{Mukai lattice}
\label{subsection-Mukai-lattice}
$$
\tilde{H}(S,\mathbb{Z})
=H^{0}(S,\mathbb{Z})\oplus H^{2}(S,\mathbb{Z})\oplus H^{4}(S,\mathbb{Z})
$$
\begin{itemize}
\item Weight 2 Hodge structure
$$
\tilde{H}^{2,0}=H^{2,0}\qquad \tilde{H}^{1,1}=H^0\oplus H^{1,1}\oplus H^2
\qquad \tilde{H}^{0,2}\cong H^{0,2}.
$$
\item Pairing:
$$
\left<(a,b,c),(a',b',c')\right>=bb'-ac'-a'c
$$
\end{itemize}

\begin{theorem}[Mukai, Orlov]
\label{theorem-Mukai-Orlov}
If $S,S'$ are K3 surfaces, then $D^b(S) \cong D^b(S')\iff
\tilde{H}(S,\mathbb{Z})\underset{\varphi}{\cong}\tilde{H}(S',\mathbb{Z})$ Hodge
isometry.
\end{theorem}

\subsection{Mukai vector}
\label{subsection-Mukai-vector}

Let $K_0(S)$ be the free abelian group generated by $\Ob(D^b(S))$. Consider
$$
v:K_0(S)\to \tilde{H}(S;\mathbb{Z})
$$
Where $[F]=[E]+[G]$ if $E \to F \to G \xrightarrow{+1}$ is a distinguished
triangle.
$$
v:[E] \to \text{ch}(E)\cdot \sqrt{\det(S)}
$$
where $\text{ch}(E)$ is the Chern character.
$$
\left<v(E),v(F)\right>=-\chi(E,F),
$$
where $\chi(E,F):=\sum(-1)^i\dim\text{Ext}^i(E,F):=\Hom_{D^b(S)}(E,F[i])$.

\begin{proof}[Proof of the backward implication of Mukai-Orlov theorem]
Let 
\begin{align*}
\varphi: \tilde{H}(S,\mathbb{Z}) &\longrightarrow \tilde{H}(S',\mathbb{Z}) \\
(0,0,1) &\longmapsto v\\
(-1,0,0) &\longmapsto v'
\end{align*}
The intersection matrix of $v$ and $v'$ is
$$
U=\begin{pmatrix}
0&1\\ 
1&0
\end{pmatrix}
$$
For simplicity assume that $v$ is of positive rank. ($V=(a,b,c)$, where $a\in
H^0$ is positive.)

The heart of the proof uses the following result by Mukai. There exists a
nonempty moduli space $\mathcal{M}$ of stable sheaves on $S'$ of class $v$
$\implies$ $S'$ is a K3 surface!
$$
0=\chi(v,v)=\underbrace{\Hom(E,E)}_{=1}-\underbrace{\text{ext}^1(E,E)}_{=2}
+\underbrace{\text{ext}^2(E,E)}_{=1}
$$
$v\cdot v'=1\implies \mathcal{M}$ is a fine moduli space. There exists a
universal family $D^b(S \times M)\ni \mathcal{E} \to S' \times M$.

{\bf Claim.} $\Phi_{\mathcal{E}}:D^b(S')\xrightarrow{\cong} D^b(\mathcal{M})$
(general criteria).

Now $\tilde{H}(S,\mathbb{Z})\xrightarrow{\varphi,\cong} \tilde{H}(S',\mathbb{Z}$
$$
\xymatrixrowsep{1em}
\xymatrix{
\tilde{H}(S,\mathbb{Z})\ar[r]^{\varphi,\cong}&\tilde{H}(S',\mathbb{Z})
\ar[r]^{\Phi_{\mathcal{E}}}&\overset{\vee}{H}(M,\mathbb{Z})\\
(0,0,1)\ar@{|->}[r]&  v \ar@{|->}[r]& (0,0,1)\\
(1,0,0)\ar@{|->}[rr]& & (1,0,0)
}
$$
\end{proof}

\end{document}
