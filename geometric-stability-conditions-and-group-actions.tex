\IfFileExists{stacks-project.cls}{%
\documentclass{stacks-project}
}{%
\documentclass{amsart}
}

% For dealing with references we use the comment environment
\usepackage{verbatim}
\newenvironment{reference}{\comment}{\endcomment}
%\newenvironment{reference}{}{}
\newenvironment{slogan}{\comment}{\endcomment}
\newenvironment{history}{\comment}{\endcomment}

% For commutative diagrams we use Xy-pic
\usepackage[all]{xy}

% We use 2cell for 2-commutative diagrams.
\xyoption{2cell}
\UseAllTwocells

% We use multicol for the list of chapters between chapters
\usepackage{multicol}

% This is generally recommended for better output
\usepackage{lmodern}
\usepackage[T1]{fontenc}

% For cross-file-references
\usepackage{xr-hyper}

% Package for hypertext links:
\usepackage{hyperref}

% For any local file, say "hello.tex" you want to link to please
% use \externaldocument[hello-]{hello}
\externaldocument[introduction-]{introduction}
\externaldocument[conventions-]{conventions}
\externaldocument[sets-]{sets}
\externaldocument[categories-]{categories}
\externaldocument[topology-]{topology}
\externaldocument[sheaves-]{sheaves}
\externaldocument[sites-]{sites}
\externaldocument[stacks-]{stacks}
\externaldocument[fields-]{fields}
\externaldocument[algebra-]{algebra}
\externaldocument[brauer-]{brauer}
\externaldocument[homology-]{homology}
\externaldocument[derived-]{derived}
\externaldocument[simplicial-]{simplicial}
\externaldocument[more-algebra-]{more-algebra}
\externaldocument[smoothing-]{smoothing}
\externaldocument[modules-]{modules}
\externaldocument[sites-modules-]{sites-modules}
\externaldocument[injectives-]{injectives}
\externaldocument[cohomology-]{cohomology}
\externaldocument[sites-cohomology-]{sites-cohomology}
\externaldocument[dga-]{dga}
\externaldocument[dpa-]{dpa}
\externaldocument[sdga-]{sdga}
\externaldocument[hypercovering-]{hypercovering}
\externaldocument[schemes-]{schemes}
\externaldocument[constructions-]{constructions}
\externaldocument[properties-]{properties}
\externaldocument[morphisms-]{morphisms}
\externaldocument[coherent-]{coherent}
\externaldocument[divisors-]{divisors}
\externaldocument[limits-]{limits}
\externaldocument[varieties-]{varieties}
\externaldocument[topologies-]{topologies}
\externaldocument[descent-]{descent}
\externaldocument[perfect-]{perfect}
\externaldocument[more-morphisms-]{more-morphisms}
\externaldocument[flat-]{flat}
\externaldocument[groupoids-]{groupoids}
\externaldocument[more-groupoids-]{more-groupoids}
\externaldocument[etale-]{etale}
\externaldocument[chow-]{chow}
\externaldocument[intersection-]{intersection}
\externaldocument[pic-]{pic}
\externaldocument[weil-]{weil}
\externaldocument[adequate-]{adequate}
\externaldocument[dualizing-]{dualizing}
\externaldocument[duality-]{duality}
\externaldocument[discriminant-]{discriminant}
\externaldocument[derham-]{derham}
\externaldocument[local-cohomology-]{local-cohomology}
\externaldocument[algebraization-]{algebraization}
\externaldocument[curves-]{curves}
\externaldocument[resolve-]{resolve}
\externaldocument[models-]{models}
\externaldocument[functors-]{functors}
\externaldocument[equiv-]{equiv}
\externaldocument[pione-]{pione}
\externaldocument[etale-cohomology-]{etale-cohomology}
\externaldocument[proetale-]{proetale}
\externaldocument[relative-cycles-]{relative-cycles}
\externaldocument[more-etale-]{more-etale}
\externaldocument[trace-]{trace}
\externaldocument[crystalline-]{crystalline}
\externaldocument[spaces-]{spaces}
\externaldocument[spaces-properties-]{spaces-properties}
\externaldocument[spaces-morphisms-]{spaces-morphisms}
\externaldocument[decent-spaces-]{decent-spaces}
\externaldocument[spaces-cohomology-]{spaces-cohomology}
\externaldocument[spaces-limits-]{spaces-limits}
\externaldocument[spaces-divisors-]{spaces-divisors}
\externaldocument[spaces-over-fields-]{spaces-over-fields}
\externaldocument[spaces-topologies-]{spaces-topologies}
\externaldocument[spaces-descent-]{spaces-descent}
\externaldocument[spaces-perfect-]{spaces-perfect}
\externaldocument[spaces-more-morphisms-]{spaces-more-morphisms}
\externaldocument[spaces-flat-]{spaces-flat}
\externaldocument[spaces-groupoids-]{spaces-groupoids}
\externaldocument[spaces-more-groupoids-]{spaces-more-groupoids}
\externaldocument[bootstrap-]{bootstrap}
\externaldocument[spaces-pushouts-]{spaces-pushouts}
\externaldocument[spaces-chow-]{spaces-chow}
\externaldocument[groupoids-quotients-]{groupoids-quotients}
\externaldocument[spaces-more-cohomology-]{spaces-more-cohomology}
\externaldocument[spaces-simplicial-]{spaces-simplicial}
\externaldocument[spaces-duality-]{spaces-duality}
\externaldocument[formal-spaces-]{formal-spaces}
\externaldocument[restricted-]{restricted}
\externaldocument[spaces-resolve-]{spaces-resolve}
\externaldocument[formal-defos-]{formal-defos}
\externaldocument[defos-]{defos}
\externaldocument[cotangent-]{cotangent}
\externaldocument[examples-defos-]{examples-defos}
\externaldocument[algebraic-]{algebraic}
\externaldocument[examples-stacks-]{examples-stacks}
\externaldocument[stacks-sheaves-]{stacks-sheaves}
\externaldocument[criteria-]{criteria}
\externaldocument[artin-]{artin}
\externaldocument[quot-]{quot}
\externaldocument[stacks-properties-]{stacks-properties}
\externaldocument[stacks-morphisms-]{stacks-morphisms}
\externaldocument[stacks-limits-]{stacks-limits}
\externaldocument[stacks-cohomology-]{stacks-cohomology}
\externaldocument[stacks-perfect-]{stacks-perfect}
\externaldocument[stacks-introduction-]{stacks-introduction}
\externaldocument[stacks-more-morphisms-]{stacks-more-morphisms}
\externaldocument[stacks-geometry-]{stacks-geometry}
\externaldocument[moduli-]{moduli}
\externaldocument[moduli-curves-]{moduli-curves}
\externaldocument[examples-]{examples}
\externaldocument[exercises-]{exercises}
\externaldocument[guide-]{guide}
\externaldocument[desirables-]{desirables}
\externaldocument[coding-]{coding}
\externaldocument[obsolete-]{obsolete}
\externaldocument[fdl-]{fdl}
\externaldocument[index-]{index}

% Theorem environments.
%
\theoremstyle{plain}
\newtheorem{theorem}[subsection]{Theorem}
\newtheorem{proposition}[subsection]{Proposition}
\newtheorem{lemma}[subsection]{Lemma}

\theoremstyle{definition}
\newtheorem{definition}[subsection]{Definition}
\newtheorem{example}[subsection]{Example}
\newtheorem{exercise}[subsection]{Exercise}
\newtheorem{situation}[subsection]{Situation}

\theoremstyle{remark}
\newtheorem{remark}[subsection]{Remark}
\newtheorem{remarks}[subsection]{Remarks}

\numberwithin{equation}{subsection}

% Macros
%
\def\lim{\mathop{\mathrm{lim}}\nolimits}
\def\colim{\mathop{\mathrm{colim}}\nolimits}
\def\Spec{\mathop{\mathrm{Spec}}}
\def\Hom{\mathop{\mathrm{Hom}}\nolimits}
\def\Ext{\mathop{\mathrm{Ext}}\nolimits}
\def\SheafHom{\mathop{\mathcal{H}\!\mathit{om}}\nolimits}
\def\SheafExt{\mathop{\mathcal{E}\!\mathit{xt}}\nolimits}
\def\Sch{\mathit{Sch}}
\def\Mor{\mathop{\mathrm{Mor}}\nolimits}
\def\Ob{\mathop{\mathrm{Ob}}\nolimits}
\def\Sh{\mathop{\mathit{Sh}}\nolimits}
\def\NL{\mathop{N\!L}\nolimits}
\def\CH{\mathop{\mathrm{CH}}\nolimits}
\def\proetale{{pro\text{-}\acute{e}tale}}
\def\etale{{\acute{e}tale}}
\def\QCoh{\mathit{QCoh}}
\def\Ker{\mathop{\mathrm{Ker}}}
\def\Im{\mathop{\mathrm{Im}}}
\def\Coker{\mathop{\mathrm{Coker}}}
\def\Coim{\mathop{\mathrm{Coim}}}

% Boxtimes
%
\DeclareMathSymbol{\boxtimes}{\mathbin}{AMSa}{"02}

%
% Macros for moduli stacks/spaces
%
\def\QCohstack{\mathcal{QC}\!\mathit{oh}}
\def\Cohstack{\mathcal{C}\!\mathit{oh}}
\def\Spacesstack{\mathcal{S}\!\mathit{paces}}
\def\Quotfunctor{\mathrm{Quot}}
\def\Hilbfunctor{\mathrm{Hilb}}
\def\Curvesstack{\mathcal{C}\!\mathit{urves}}
\def\Polarizedstack{\mathcal{P}\!\mathit{olarized}}
\def\Complexesstack{\mathcal{C}\!\mathit{omplexes}}
% \Pic is the operator that assigns to X its picard group, usage \Pic(X)
% \Picardstack_{X/B} denotes the Picard stack of X over B
% \Picardfunctor_{X/B} denotes the Picard functor of X over B
\def\Pic{\mathop{\mathrm{Pic}}\nolimits}
\def\Picardstack{\mathcal{P}\!\mathit{ic}}
\def\Picardfunctor{\mathrm{Pic}}
\def\Deformationcategory{\mathcal{D}\!\mathit{ef}}

\begin{document}

\title{Geometric Stability Conditions and Group Actions}
\maketitle

Minicourse by Hannah Dell (University of Bonn, Germany), CIMPA school
Florianópolis 2025.

Notes at 
\href{http://github.com/danimalabares/cimpa-floripa}
{github.com/danimalabares/cimpa-floripa}

\bigskip\noindent

{\bf Abstract.} Group actions on categories arise naturally from symmetries of
varieties and quivers, but how does this interact with Bridgeland stability? In
the frist half of this course we will introduce equivariant categories, which
generalises the category of equivariant sheaves. Then we will show there is a
correspondence between stability conditions on a category with a fniite group
action, and stability conditions on the equivariant category – this will also
play a role in Xiaolei Zhao’s course. We will use this to produce stability
conditions on quotient varieties (and stacks).

In the second half of the course, we will apply this to study open questions
about the geometry of the stability manifold: in particular, we will discuss
"geometric stability conditions" – those for which all skyscraper sheaves of
points are stable. In practice, these are constructed using slope stability for
sheaves. Some varieties have only geometric stability conditions, whereas in
other cases, there are more (for example if there is an equivalence with quiver
representations). Lie Fu, Chunyi Li, and Xiaolei Zhao were the frist to provide
a general result explaining this phenomenon. In particular, they showed that if
a variety has a fniite map to an abelian variety, then all stability conditions
are geometric. We will test the converse on free quotients of abelian varieties
by fniite groups, including Beauville-type and bielliptic surfaces. This is
based on joint work with Edmund Heng and Anthony Licata. 

\bigskip\noindent
\tableofcontents
\bigskip\noindent

\section*{Motivation}
\label{section-motivation}

The Bridgeland stability machine:

$$
[diagram]
$$
\noindent
{\bf Question 1.} How does the geometry of $\text{Stab}(X)$ relate to $X$?
\noindent
{\bf Question 2.} How to invariants (of $X$ or $\ddot\cup $) behave under (finite) 
group actions?

\medskip\noindent
{\bf Goal of these lectures.} Answer both by studying ``free quotients'' (and
see lots of examples along the way!)

\section{Geometric stability conditions on surfaces}
\label{section-geometric-stability-conditions-on-surfaces}
\noindent
{\bf Setup.} $X$ smooth projective surface over $\mathbb{C}$.
\begin{align*}
\text{NS}(X)&:=\text{Pic}(X)/\text{Pic}^0(X)\qquad \text{Neron-Severi group}\\
\text{NS}_{\mathbb{R}}(X)& :=\text{NS}(X) \otimes \mathbb{R}\\
\text{Amp}_{\mathbb{R}}(X)&:=\text{ ample cone, i.e. $H \in
\text{NS}_{\mathbb{R}}(X)$ such that $H^2>0$}
\end{align*}

\begin{definition}
\label{definition-slope-and-stability}
Let $0 \neq  E \in \text{Coh}(X)$ and $H \in \text{Amp}_{\mathbb{R}}(X)$,
\begin{enumerate}
\item The {\it $H$-slope} of $E$ is
$$
\mu_H(E):=
\begin{cases}
+\infty\qquad &\text{ch}_0(E)=0 \\
\frac{H\cdot \text{ch}_1(E)}{H^2 \text{ch}_0(E)}\qquad &\text{otherwise}
\end{cases}
$$
\item $E$ is {\it $H$-(semi)stable} if $0 \neq  F \subset E$ implies
$$
\mu_H(F) < \mu_H(E/F)\qquad (\mu_H\leq \mu_H(E/F)
$$
\end{enumerate}
\end{definition}

\begin{example}
\label{example-semistable-condition}

\end{example}

\noindent
{\bf Problem.} [missing]

\begin{definition}
\label{definition-numerical-Bridgeland-stability-condition}
A path $\sigma=(\mathcal{A},Z)$ is a {\it (numerical) Bridgeland stability
condition} for $D^b(X)$ if
\begin{enumerate}
\item $\mathcal{A} \subset D^b(X)$ is the heart of an bounded $t$-structure.
\item $Z:K_0(X) \to \mathbb{C}$ homomorphism (the {\it central charge}) such
that
\begin{itemize}
\item $0 \neq E \in \mathcal{A} \implies  Z(E) \in \mathbb{H}$, where
$\mathbb{H}$ is the upper half plane in $\mathbb{C}$.
\item HN filtrations exist.
\item Factors via $\text{Knum}(X)$.
\end{itemize}
\item Support property.
\end{enumerate}
\end{definition}

Let $X$ be a surface, and fix  $H \in \text{Amp}_\mathbb{R}(X)$ and $\beta \in
\mathbb{R}$. Define
\begin{align*}
\tau_{M,\beta}&:=\{E \in \text{Coh}(X):E \overset{\text{surj.}}{\to} Q \neq 0 
\implies \mu_M(Q)>\beta\}\\
\mathcal{F}_{H,\beta}&:=\{E \in \text{Coh}(X):0 \neq  F \subset E
\implies \mu_M(F) \leq \beta\}
\end{align*}

\begin{exercise}
\label{exercise-torsion-pair}
$(\tau_{M,\beta},\mathcal{F}_{H,\beta})$ is a torsion pair in $\text{Coh}(X)$.
\end{exercise}

\begin{definition}
\label{definition-Coh-H-beta}
$$
\text{Coh}^{H,\beta}(X):=\left\{ E \in D^b(X):
\substack{H^0(E) \in \tau_{H,\beta} \\ 
H^{-1}(E) \in \mathcal{F}_{H,\beta}\\
H^i(E)=0} \right\} 
$$
\end{definition}

\begin{exercise}
\label{exercise-new-slope}

\end{exercise}

There are notes available about the following:

\begin{theorem}[Bridgeland '08, Arcara-Bertam, Macri-Schmidt]
\label{theorem-Bridgeland}
For $\alpha\gg 0$,
$$
\mathbb{C}\cdot \{\sigma_{H,D,\alpha,\beta}=
(\text{Coh}^{H,B}(X),Z_{H,D,\alpha,\beta})\}
$$
is a continuous family of Bridgeland stability conditions.
\end{theorem}

\noindent
{\bf Question 3.} Is this all of $\text{Stab}(X)$?

\begin{exercise}
\label{exercise-O-is-stable}
$\mathcal{O}_x$ is $\sigma_{H,D,\alpha,\beta}$-stable for all $x\in X$.
\end{exercise}

\begin{definition}
\label{definition-geometric-projective-variety}
$X$ smooth projective variety, $\sigma \in \text{Stab}(X)$ is {\it geometric} if
for all $x \in X$, $\mathcal{O}_x$ is $\sigma$-stable. Write
$\text{Stab}^{\text{geo}}(X)=$ all geometric stability conditions.
\end{definition}

\medskip\noindent
Now we can refine Question 3:

\medskip\noindent
{\bf Question 3.1.} Does Theorem \ref{theorem-Bridgeland} describe all geometric
stability conditions?

\medskip\noindent
{\bf Question 3.2.} Do there exist nongeometric stability conditions?

\medskip\noindent
{\bf Answer 3.2.}
\begin{itemize}
\item $\dim X=1$ no! Unless $X=\mathbb{P}^1$.
\item $\dim X \leq 3$? Yes if $X$ has a full exceptional collection.
\item $\dim X=2$?
\begin{enumerate}
\item If $X$ is an abelian surface, no!
\item Yes for $\mathbb{P}^2$, and more generally for rational surfaces.
\item Yes for K3 surfaces.
\item $X \supset C \cong \mathbb{P}^1$ such that $C^2 < 0$.
\end{enumerate}
\end{itemize}

\medskip\noindent
{\bf Takeaway.} $\text{Stab}(X)$ sees something about how (1) is different to
(2)-(4) but what is it?

\medskip\noindent
The following theorem is valid in any dimension:

\begin{theorem}[Lie Fu-Chunyi Li-Xiaolei Zhao '21]
\label{theorem-Albanese-variety}
$X$ has a finite Albanese morphism ($\iff$ there exist a finite morphism to an
abelian variety), then
$$
\text{Stab}(X) = \text{Stab}^{\text{geo}}(X)
$$
\end{theorem}

\begin{proof}[Sketch of proof]
\begin{itemize}
\item $\sigma \in \text{Stab}(X)$.
\item Let $E_1,\ldots,E_k$ be the {\it Jordan-Hölder factors} of $\mathcal{O}_X$
w.r.t. $\sigma$ (i.e. $E_i$ $\sigma$-stable connected component containing the
identity).
\item $\mathcal{L} \in \text{Pic}^0(X)$.
\end{itemize}
By [Polishchuk '07], $\otimes \mathcal{L}$ does not change stability. Also
$\mathcal{O}_x \otimes \mathcal{L}\cong \mathcal{O}_x$ so HN filtration and
Jordan-Hölder factors are preserved $\implies $ $E_i \otimes \mathcal{L} \cong
E_i$.
\begin{itemize}
\item $\mathcal{L} \in \text{Pic}^0(\text{Alb}(X))$, $E_i \otimes
\text{alb}^\vee(\mathcal{L})\cong E_i$.
\end{itemize}
\begin{align*}
\implies & R\text{alb}_*(E_i)\otimes \mathcal{L} \cong R\text{alb}_*(E_i)\qquad 
\text{(projection formula)}\\
\implies &  R\text{alb}_*(E_i)\text{ has finite support}\qquad 
\text{([Polishucuk '03])}\\
\implies &E_i\text{ has finite support}\qquad 
\text{($\text{alb}_*$ finite)}\\
\overset{\text{claim}}{\implies }&\mathcal{O}_x\text{ is $\sigma$-stable}
\end{align*}
\end{proof}

\medskip\noindent
{\bf Question 4.} Albanese morphism of $X$ is not finite implies that there
exist nongeometric stability conditions?

\medskip\noindent
The idea is to investigate this for examples arising as free quotients. If 
$G \mathbb{y} X$ is a free action of a finite group, then $\text{alb}_X$ is
finite, and for $Y:=X/G$ we have $\text{alb}_Y$ is not finite. In sections 2 and
3 we will compate $\text{Stab}(X)$ and $\text{Stab}(X/G)$.

\section{Equivariant categories}
\label{section-equivariant-categories}

Let $G$ be a finite group and $\mathcal{D}$ a $k$-linear category (i.e. all
$\Hom$ are $k$-vector spaces, in particular it is additive), where $k=\bar{k}$
and $(|G|,\text{char}k)=1$.

The following definition is by [Deligne '97].

\begin{definition}[Group action]
\label{definition-group-action}
A {\it (right) action} of $G$ on $\mathcal{D}$ is the data of
\begin{enumerate}
\item $\forall g \in G$, $\phi_g:\mathcal{D}\xrightarrow{\sim}\mathcal{D}$.
\item $\forall g,h\in G$, a natural isomorphism 
$\mathcal{E}_{g,h}:\phi_g\circ \phi_h \xrightarrow{\sim}\phi_{hg}$ such that
$$
\xymatrix{
\phi_f \circ\phi_g \circ\phi_h
\ar[r]^{\mathcal{E}_{g,h}}\ar[d]_{\mathcal{E}_{f,g}}&\phi_f \circ \phi_{hg}
\ar[d]^{\mathcal{E}_{f \circ hg}}\\
\phi_{gf}\circ \phi_h\ar[r]^{\mathcal{E}_{gf,h}}&\phi_{hgf}
}
$$
\end{enumerate}
\end{definition}

\begin{remark}
\label{remark-more-than-a-homomorphism}
This is more than a group a homomorphism $G \to \text{Aut}(\mathcal{D})$. 
Indeed, given this, $\forall E \in \mathcal{D}$, 
$$
\phi_g \circ \phi_h(E)\cong \phi_{gh}(E)
$$
but may not come from a natural isomorphism of functors
$\phi_g\circ\phi_h\cong\phi_{hg}$.
\end{remark}

\begin{example}
\label{example-group-actions}
\begin{enumerate}
\item $X$: scheme, $G \leq \text{Aut}(X)$. For every $g \in G$ define
$\phi_g:=g^* :\text{Coh}(X)\xrightarrow{\sim}\text{Coh}(X)$. Then for every
$g,h$ there are canonical isomorphisms
$$
\phi_g \circ\phi_h=g^*\circ h^*\xrightarrow{\sim}(hg)^* =\phi_{hg}.
$$
$G \mathbb{y} \text{Coh}(X)$ lifts to $G\mathbb{y}D^b(X)$.

Eg. an Enriques surface, i.e. $Y=X/\mathbb{Z}/2\mathbb{Z}$, $X$ a K3
surface, $\mathbb{Z}/2\mathbb{Z}=\left<i\right>$, for $i:X\to X$ involution, $x
\in X$, $i^* (\mathcal{O}_x)=\mathcal{O}_{i^{-1}(x)}$.

\item $Q$: acyclic quiver, $G\leq \text{Aut}^+(Q)=$ automorphisms of the
graph, orientation preserving. Then $G\mathbb{y}\text{Rep}Q$. Via
$(M_i,\varphi_\alpha)\mapsto (M_{g(i)},\varphi_{g(x)})$.

Eg. [Missing]
\end{enumerate}
\end{example}

\medskip\noindent
{\bf Equivariantization.}

\begin{definition}
\label{definition-G-equivariant-object}
Let $G \mathbb{y}\mathcal{D}$. A {\it $G$-equivariant} object is a pair
$(E,\{\lambda_g\}_G$,
\begin{itemize}
\item $E \in \mathcal{D}$,
\item $\lambda_g:E \xrightarrow{\sim}\phi_g(E)$, a {\bf choice} of isomorphism
for all $g\in G$, such that
$$
\xymatrix{
E\ar[r]^{\lambda_g}\ar[d]_{\lambda_hg,\sim}&\phi_g(E)\\
\phi_{hg}(E)&\phi_g(\phi_n(E))\ar[l]_{\mathcal{E}_{gh}}
}
$$
A {\it morphism of $G$-equivariant objects} is 
$$
(E,\{\lambda_g\}_G)\to(E'_g,\{\lambda'_g\}_G)
$$
is $F \in \Hom_\mathcal{D}(E,E')$ such that for all $g \in G$,
$$
\xymatrix{
E\ar[d]_F\ar[r]^{\lambda_g}&\phi_g(E)\ar[d]^{\phi_g(F)}\\
E'\ar[r]^{\lambda'_g}&\phi_g(E')
}
$$
Together these form a category, $\mathcal{D}_G$, called {\it $G$-equivariant
category}.
\end{itemize}
\end{definition}

\begin{remark}
\label{remark-uses-isomorphisms-Egh}
This uses the isomorphisms $\mathcal{E}_{g,h}$.
\end{remark}

\begin{example}
\label{example-G-equivariant-objects}
\begin{enumerate}
\item $G\mathbb{y}X$, $G$-equivariant objects are $G$-equivariant coherent
sheaves.
$$
(\text{Coh}(X))_G=:\text{Coh}_G(X)
$$
In fact, $\text{Coh}_G(X)\cong \text{Coh}([X/G])$.

If $G$ acts freely, then
$$
\xymatrix{
X\ar[dr]_\pi\ar[rr]^g &  &  X\ar[dl]_\pi\\
X/G
}
$$
and $\text{Coh}(X/G)\cong \text{Coh}_G(X)$ via $E \mapsto (\pi^*
E,\{\lambda_g\}_{g \in G}$ where
$$
\xymatrix{
\lambda_g:\pi^* E\ar[dr]_{\lambda_g}\ar[r]& (\pi \circ g)^* E\ar[d]^=\\
&g^*(\pi^*E)
}
$$
Eg. $Y=X/(\mathbb{Z}/2\mathbb{Z})$ Enriques surface, $y \in Y$,
$\text{supp}\pi^{-1}(y)=\{x,x'=i^{-1}(x)\}$ 
$$
\xymatrixrowsep{.2em}
\xymatrix{
&\mathcal{O}_x\ar[r]^{\text{id}}&\mathcal{O}_x\\
\pi^* (\mathcal{O}_x)=&\oplus&\oplus&=\text{id}^*(\pi^*(\mathcal{O}_y))\\
&\mathcal{O}_{x'}\ar[r]^{\text{id}}&\mathcal{O}_{x'}
}
$$
$$
\xymatrixrowsep{.2em}
\xymatrix{
&\mathcal{O}_x\ar[r]^{i^*}\ar[ddr]&\mathcal{O}_x\\
\pi^* (\mathcal{O}_y)=&\oplus&\oplus&=i^*(\pi^*(\mathcal{O}_y))\\
&\mathcal{O}_{x'}\ar[r]^{i^*}\ar[uur]&\mathcal{O}_{x'}
}
$$
In general,
$$
(D^b(X))_G\cong D^b(\text{Coh}_G(X))=:D^b_G(X)
$$
\item {\bf Theorem (Demonet '10).} $G\mathbb{y} Q$ then $\exists Q_G$ such that
$\text{Rep}(Q_G)\cong\text{Rep}(Q)_G$. Eg. [Missing.]

\item $G \mathbb{y} \mathcal{D}$ for $G$ abelian,
$\hat{G}:=\Hom(G,K^*)\mathbb{y}\mathcal{D}_G$ by 
$$
\phi_{\underbrace{x}_{\in G}}((E,\{\lambda_g\}_G))
:=(E,\{\lambda_g\}_G)\otimes x=(E,\{\lambda_g\cdot x(g)\}_{g \in G})
$$
Eg. $Y$ Enriques surface … [Missing]
\end{enumerate}
\end{example}
 
\begin{theorem}[Elagin '15]
\label{theorem-Elagin}
$G$ abelian, $G\mathbb{y}\mathcal{D}$
$$
(\mathcal{D}_G)_{\hat{G}}\cong \mathcal{D}
$$
\end{theorem}

\end{document}
