\IfFileExists{stacks-project.cls}{%
\documentclass{stacks-project}
}{%
\documentclass{amsart}
}

% For dealing with references we use the comment environment
\usepackage{verbatim}
\newenvironment{reference}{\comment}{\endcomment}
%\newenvironment{reference}{}{}
\newenvironment{slogan}{\comment}{\endcomment}
\newenvironment{history}{\comment}{\endcomment}

% For commutative diagrams we use Xy-pic
\usepackage[all]{xy}

% We use 2cell for 2-commutative diagrams.
\xyoption{2cell}
\UseAllTwocells

% We use multicol for the list of chapters between chapters
\usepackage{multicol}

% This is generally recommended for better output
\usepackage{lmodern}
\usepackage[T1]{fontenc}

% For cross-file-references
\usepackage{xr-hyper}

% Package for hypertext links:
\usepackage{hyperref}

% For any local file, say "hello.tex" you want to link to please
% use \externaldocument[hello-]{hello}
\externaldocument[introduction-]{introduction}
\externaldocument[conventions-]{conventions}
\externaldocument[sets-]{sets}
\externaldocument[categories-]{categories}
\externaldocument[topology-]{topology}
\externaldocument[sheaves-]{sheaves}
\externaldocument[sites-]{sites}
\externaldocument[stacks-]{stacks}
\externaldocument[fields-]{fields}
\externaldocument[algebra-]{algebra}
\externaldocument[brauer-]{brauer}
\externaldocument[homology-]{homology}
\externaldocument[derived-]{derived}
\externaldocument[simplicial-]{simplicial}
\externaldocument[more-algebra-]{more-algebra}
\externaldocument[smoothing-]{smoothing}
\externaldocument[modules-]{modules}
\externaldocument[sites-modules-]{sites-modules}
\externaldocument[injectives-]{injectives}
\externaldocument[cohomology-]{cohomology}
\externaldocument[sites-cohomology-]{sites-cohomology}
\externaldocument[dga-]{dga}
\externaldocument[dpa-]{dpa}
\externaldocument[sdga-]{sdga}
\externaldocument[hypercovering-]{hypercovering}
\externaldocument[schemes-]{schemes}
\externaldocument[constructions-]{constructions}
\externaldocument[properties-]{properties}
\externaldocument[morphisms-]{morphisms}
\externaldocument[coherent-]{coherent}
\externaldocument[divisors-]{divisors}
\externaldocument[limits-]{limits}
\externaldocument[varieties-]{varieties}
\externaldocument[topologies-]{topologies}
\externaldocument[descent-]{descent}
\externaldocument[perfect-]{perfect}
\externaldocument[more-morphisms-]{more-morphisms}
\externaldocument[flat-]{flat}
\externaldocument[groupoids-]{groupoids}
\externaldocument[more-groupoids-]{more-groupoids}
\externaldocument[etale-]{etale}
\externaldocument[chow-]{chow}
\externaldocument[intersection-]{intersection}
\externaldocument[pic-]{pic}
\externaldocument[weil-]{weil}
\externaldocument[adequate-]{adequate}
\externaldocument[dualizing-]{dualizing}
\externaldocument[duality-]{duality}
\externaldocument[discriminant-]{discriminant}
\externaldocument[derham-]{derham}
\externaldocument[local-cohomology-]{local-cohomology}
\externaldocument[algebraization-]{algebraization}
\externaldocument[curves-]{curves}
\externaldocument[resolve-]{resolve}
\externaldocument[models-]{models}
\externaldocument[functors-]{functors}
\externaldocument[equiv-]{equiv}
\externaldocument[pione-]{pione}
\externaldocument[etale-cohomology-]{etale-cohomology}
\externaldocument[proetale-]{proetale}
\externaldocument[relative-cycles-]{relative-cycles}
\externaldocument[more-etale-]{more-etale}
\externaldocument[trace-]{trace}
\externaldocument[crystalline-]{crystalline}
\externaldocument[spaces-]{spaces}
\externaldocument[spaces-properties-]{spaces-properties}
\externaldocument[spaces-morphisms-]{spaces-morphisms}
\externaldocument[decent-spaces-]{decent-spaces}
\externaldocument[spaces-cohomology-]{spaces-cohomology}
\externaldocument[spaces-limits-]{spaces-limits}
\externaldocument[spaces-divisors-]{spaces-divisors}
\externaldocument[spaces-over-fields-]{spaces-over-fields}
\externaldocument[spaces-topologies-]{spaces-topologies}
\externaldocument[spaces-descent-]{spaces-descent}
\externaldocument[spaces-perfect-]{spaces-perfect}
\externaldocument[spaces-more-morphisms-]{spaces-more-morphisms}
\externaldocument[spaces-flat-]{spaces-flat}
\externaldocument[spaces-groupoids-]{spaces-groupoids}
\externaldocument[spaces-more-groupoids-]{spaces-more-groupoids}
\externaldocument[bootstrap-]{bootstrap}
\externaldocument[spaces-pushouts-]{spaces-pushouts}
\externaldocument[spaces-chow-]{spaces-chow}
\externaldocument[groupoids-quotients-]{groupoids-quotients}
\externaldocument[spaces-more-cohomology-]{spaces-more-cohomology}
\externaldocument[spaces-simplicial-]{spaces-simplicial}
\externaldocument[spaces-duality-]{spaces-duality}
\externaldocument[formal-spaces-]{formal-spaces}
\externaldocument[restricted-]{restricted}
\externaldocument[spaces-resolve-]{spaces-resolve}
\externaldocument[formal-defos-]{formal-defos}
\externaldocument[defos-]{defos}
\externaldocument[cotangent-]{cotangent}
\externaldocument[examples-defos-]{examples-defos}
\externaldocument[algebraic-]{algebraic}
\externaldocument[examples-stacks-]{examples-stacks}
\externaldocument[stacks-sheaves-]{stacks-sheaves}
\externaldocument[criteria-]{criteria}
\externaldocument[artin-]{artin}
\externaldocument[quot-]{quot}
\externaldocument[stacks-properties-]{stacks-properties}
\externaldocument[stacks-morphisms-]{stacks-morphisms}
\externaldocument[stacks-limits-]{stacks-limits}
\externaldocument[stacks-cohomology-]{stacks-cohomology}
\externaldocument[stacks-perfect-]{stacks-perfect}
\externaldocument[stacks-introduction-]{stacks-introduction}
\externaldocument[stacks-more-morphisms-]{stacks-more-morphisms}
\externaldocument[stacks-geometry-]{stacks-geometry}
\externaldocument[moduli-]{moduli}
\externaldocument[moduli-curves-]{moduli-curves}
\externaldocument[examples-]{examples}
\externaldocument[exercises-]{exercises}
\externaldocument[guide-]{guide}
\externaldocument[desirables-]{desirables}
\externaldocument[coding-]{coding}
\externaldocument[obsolete-]{obsolete}
\externaldocument[fdl-]{fdl}
\externaldocument[index-]{index}

% Theorem environments.
%
\theoremstyle{plain}
\newtheorem{theorem}[subsection]{Theorem}
\newtheorem{proposition}[subsection]{Proposition}
\newtheorem{lemma}[subsection]{Lemma}

\theoremstyle{definition}
\newtheorem{definition}[subsection]{Definition}
\newtheorem{example}[subsection]{Example}
\newtheorem{exercise}[subsection]{Exercise}
\newtheorem{situation}[subsection]{Situation}

\theoremstyle{remark}
\newtheorem{remark}[subsection]{Remark}
\newtheorem{remarks}[subsection]{Remarks}

\numberwithin{equation}{subsection}

% Macros
%
\def\lim{\mathop{\mathrm{lim}}\nolimits}
\def\colim{\mathop{\mathrm{colim}}\nolimits}
\def\Spec{\mathop{\mathrm{Spec}}}
\def\Hom{\mathop{\mathrm{Hom}}\nolimits}
\def\Ext{\mathop{\mathrm{Ext}}\nolimits}
\def\SheafHom{\mathop{\mathcal{H}\!\mathit{om}}\nolimits}
\def\SheafExt{\mathop{\mathcal{E}\!\mathit{xt}}\nolimits}
\def\Sch{\mathit{Sch}}
\def\Mor{\mathop{\mathrm{Mor}}\nolimits}
\def\Ob{\mathop{\mathrm{Ob}}\nolimits}
\def\Sh{\mathop{\mathit{Sh}}\nolimits}
\def\NL{\mathop{N\!L}\nolimits}
\def\CH{\mathop{\mathrm{CH}}\nolimits}
\def\proetale{{pro\text{-}\acute{e}tale}}
\def\etale{{\acute{e}tale}}
\def\QCoh{\mathit{QCoh}}
\def\Ker{\mathop{\mathrm{Ker}}}
\def\Im{\mathop{\mathrm{Im}}}
\def\Coker{\mathop{\mathrm{Coker}}}
\def\Coim{\mathop{\mathrm{Coim}}}

% Boxtimes
%
\DeclareMathSymbol{\boxtimes}{\mathbin}{AMSa}{"02}

%
% Macros for moduli stacks/spaces
%
\def\QCohstack{\mathcal{QC}\!\mathit{oh}}
\def\Cohstack{\mathcal{C}\!\mathit{oh}}
\def\Spacesstack{\mathcal{S}\!\mathit{paces}}
\def\Quotfunctor{\mathrm{Quot}}
\def\Hilbfunctor{\mathrm{Hilb}}
\def\Curvesstack{\mathcal{C}\!\mathit{urves}}
\def\Polarizedstack{\mathcal{P}\!\mathit{olarized}}
\def\Complexesstack{\mathcal{C}\!\mathit{omplexes}}
% \Pic is the operator that assigns to X its picard group, usage \Pic(X)
% \Picardstack_{X/B} denotes the Picard stack of X over B
% \Picardfunctor_{X/B} denotes the Picard functor of X over B
\def\Pic{\mathop{\mathrm{Pic}}\nolimits}
\def\Picardstack{\mathcal{P}\!\mathit{ic}}
\def\Picardfunctor{\mathrm{Pic}}
\def\Deformationcategory{\mathcal{D}\!\mathit{ef}}

\begin{document}

\title{Noncommutative Abelian Surfaces and Kummer Type Hyperkähler Varieties}
\maketitle

\noindent
Minicourse by Xiaolei Zhao (University of California, Santa Barbara (CA), USA), 
CIMPA school Florianópolis 2025.

\medskip\noindent
Notes at 
\href{http://github.com/danimalabares/cimpa-floripa}
{github.com/danimalabares/cimpa-floripa}

\bigskip\noindent

{\bf Abstract.} Examples of noncommutative K3 surfaces arise from semiorthogonal
decompositions of the bounded derived category of certain Fano varieties. The
most interesting cases are those of cubic fourfolds and Gushel-Mukai varieties
of even dimension. Using the deep theory of families of stability conditions,
locally complete families of hyperkähler manifolds deformation equivalent to
Hilbert schemes of points on a K3 surface have been constructed from moduli
spaces of stable objects in these noncommutative K3 surfaces. On the other hand,
an explicit description of a locally complete family of hyperkähler manifolds
deformation equivalent to a generalized Kummer variety is not available from
classical geometry. In this lecture series, we will construct families of
noncommutative abelian surfaces as equivariant categories of the derived
category of K3 surfaces which specialize to Kummer K3 surfaces. Then we will
explain how to induce stability conditions on them and produce examples of
locally complete families of hyperkähler manifolds of Kummer type. Based on
joint work with Arend Bayer, Alex Perry and Laura Pertusi. 

\bigskip\noindent
\tableofcontents
\bigskip\noindent

\noindent

\section{K3 surfaces and Hyperkähler varieties}
\label{section-K3-surfaces-and-hyperkahler-varieties}

\begin{definition}
\label{definition-K3-surface}
A {\it K3 surface} (over $\mathbb{C}$) is a smooth projective surface $S$ such
that $\omega_S \cong \mathcal{O}_S$ and $\pi_1(S)=1$.
\end{definition}

\begin{example}[K3 surfaces]
\label{example-K3-surfaces}
\begin{enumerate}
\item $S\subset \mathbb{P}^3$ of degree 4. A straightforward computation shows
the conditions of the definition are verified.
\item $S \to \mathbb{P}^2$ double cover ramified along a degree 6 curve.
\end{enumerate}
\end{example}

\begin{definition}
\label{definition-hyperkahler-variety}
A {\it hyperkähler} variety is a smooth projective variety $X$ such that
\begin{enumerate}
\item $\pi_{1}(X)=1$ 
\item $H^{0}(X,\Omega_X^2)=\mathbb{C} \omega$ (i.e. the space of global
holomorphic 2-forms is one dimensional) where $\omega$ is holomorphically
symplectic.
\end{enumerate}
\end{definition}

\begin{remark}
\label{remark-hyperkahler}
If $X$ is hyperkähler, it must be even dimensional and $\omega_X=\mathcal{O}_X$.
\end{remark}

\begin{remark}
\label{remark-Beauville-Bogomolov}
We just mention the name of Beauville-Bogomolov theorem.
\end{remark}

For a while people were looking for examples of hyperkähler varieties.

\begin{example}[Hyperkähler varieties]
\label{example-hyperkahler}
\begin{enumerate}
\item K3 surface $S$.
\item $X=S^{[n]}$ Hilbert scheme of $n$ points on K3 surface. (Moduli space of
stable sheaves on K3 $S$ of rank 1,  $c_1=0$, $c_2=0$.)
\item $\mathcal{M}_H(s,v)=$ moduli of $H$-stable sheaves on $S$ of class $v$.
($v$ primitive, $H$ is $v$-generic.) (Recall the example in Cristina's course
where we studied how the moduli changes under changes in the polarization $H$.)
Let $[E] \in \mathcal{M}_H(s,v)$. We have the Yoneda map
 $$
T_{[E]}\mathcal{M}\cong \text{Ext}^1(E,E)\times \text{Ext}^1(E,E)
\to \text{Ext}^2(E,E)\overset{\substack{S \text{ is}\\\text{ a K3}}}{\cong}\Hom(E,E)^*\cong \mathbb{C}
$$
where the first factor product of $\text{Ext}^1(E,E)\times \text{Ext}^1(E,E)$ is
associated to first arrow, the second factor to the second arrow, of the
following diagram:
$$
E \to E[1] \to E[2]
$$
This moduli space is always of Picard rank 2.
\item $Y \subset \mathbb{P}^5$ a smooth cubic fourfold.
$$
F(Y)=\{[\ell] \in \text{Gr}(2,6)|\ell \subseteq Y\}
$$
is a hyperkähler 4-fold. For a specific choice $Y_0$ of hyperkähler 4-fold in
the moduli of smooth cubic 4-folds (which is 20-dimensional) we find $F(Y_0)$ is
deformation equivalent to $S^{[2]}$ where $S$ is a K3 surface.
\end{enumerate}
\end{example}

\medskip\noindent
Here is a general picture:
\iffalse$$
\xymatrix{
\substack{\text{cubic 4-fold $Y$} \\ D^b(Y)}\ar[dr]&  &  
\substack{\text{moduli of} \\ \text{stable objects}\\ \text{in $\text{Ku}_Y$ as}
\\ \text{HK variety}}\\
&\substack{\text{sod K3} \\ \text{cat conditions} \\ \text{Ku}_Y }
\ar[ur]_{\substack{\text{stability} \\ \text{conditions}}}
$$\fi

\medskip\noindent
Let $X,Y$ be projective smooth varieties and $f:X\to Y$. Consider the functors
$$
Rf_*:D^b(X) \to D^b(Y)\qquad Lf^*:D^b(Y) \to D^b(X)
$$
for $F,G \in D^b(X)$, $F\overset{L}{\otimes}G \in D^b(X)$. {\bf Convention:} we
drop the $R$ and $L$.

\begin{definition}
\label{definition-}
Let $K \in D^b(X \times Y)$. The {\it Fourier-Mulai functor} is
\begin{align*}
\Phi_K: D^b(X) &\longrightarrow D^b(Y) \\
F &\longmapsto \text{pr}_{Y,*}(\text{pr}_X^*F \otimes K)
\end{align*}
where we are using our convention --- all functors here are derived. Here the
maps are
$$
\xymatrix{
& X \times Y\ar[dl]_{\text{pr}_X}\ar[dr]^{\text{pr}_Y}\\
X& & Y
}
$$
\end{definition}

\begin{example}
\label{example-Fourier-Mukai}
$f:X \to Y$ and $\Gamma_f \equiv X \times Y$ graph. Then
$\Phi_{\mathcal{O}_{p_f}}=f_*$.
\end{example}

\begin{theorem}[Orlov]
\label{theorem-Orlov}
If $F:D^b(X) \to D^b(Y)$ is an equivalence, then $\exists ! K \in D^b(X \times
Y)$ such that $F \cong \Phi_K$.
\end{theorem}

\begin{theorem}
\label{theorem-preserved-by-derived-equivalence}
The following are preserved by derived equivalence:
\begin{enumerate}
\item (Bondal-Orlov.) $\dim X$.
\item (Bondal-Orlov.) $\bigoplus_{m \geq 0}H^{0}(X,\pm m K_X)$.
\item $H^{*}(X,\mathbb{Q})$, and $\bigoplus_{p-q=i}H^{p,q}(X)$ for any $i \in
\mathbb{Z}$.
\end{enumerate}
\end{theorem}

\medskip\noindent
{\bf Non-trivial equivalences.} $S$ a K3 surface. $H^{2}(S,\mathbb{Z})$.
\begin{itemize}
\item (Hodge decomposition.) 
$H^{2}(S,\mathbb{C})\cong H^{2,0} \oplus H^{1,1} \oplus H^{0,2}$.
\item Pairing on $H^{2}(S,\mathbb{Z})$ given by cup product.
\end{itemize}

\begin{theorem}[Torelli for K3]
\label{theorem-Torelli-for-K3}
Two K3 surfaces $S,S'$ are isomorphic if and only if there exists a Hodge
isometry $\varphi:H^{2}(S,\mathbb{Z})\to H^{2}(S',\mathbb{Z})$ 
(i.e. an isomorphism that preserves the Hodge decomposition and the pairing).
\end{theorem}

This was the classical content.

\subsection{Mukai lattice}
\label{subsection-Mukai-lattice}

``The Mukai lattice is some sort of topological invariant that behaves better 
with derived category.''

$$
\tilde{H}(S,\mathbb{Z})
=H^{0}(S,\mathbb{Z})\oplus H^{2}(S,\mathbb{Z})\oplus H^{4}(S,\mathbb{Z})
$$
\begin{itemize}
\item Weight 2 Hodge structure
$$
\tilde{H}^{2,0}=H^{2,0}\qquad \tilde{H}^{1,1}=H^0\oplus H^{1,1}\oplus H^2
\qquad \tilde{H}^{0,2}\cong H^{0,2}.
$$
\item Pairing:
$$
\left<(a,b,c),(a',b',c')\right>=bb'-ac'-a'c
$$
\end{itemize}

\begin{theorem}[Mukai, Orlov]
\label{theorem-Mukai-Orlov}
If $S,S'$ are K3 surfaces, then $D^b(S) \cong D^b(S')\iff
\tilde{H}(S,\mathbb{Z})\underset{\varphi}{\cong}\tilde{H}(S',\mathbb{Z})$ Hodge
isometry.
\end{theorem}

\subsection{Mukai vector}
\label{subsection-Mukai-vector}

Let $K_0(S)$ be the free abelian group generated by $\Ob(D^b(S))$. Consider
$$
v:K_0(S)\to \tilde{H}(S;\mathbb{Z})
$$
Where $[F]=[E]+[G]$ if $E \to F \to G \xrightarrow{+1}$ is a distinguished
triangle.
$$
v:[E] \to \text{ch}(E)\cdot \sqrt{\det(S)}
$$
where $\text{ch}(E)$ is the Chern character.
$$
\left<v(E),v(F)\right>=-\chi(E,F),
$$
where $\chi(E,F):=\sum(-1)^i\dim\text{Ext}^i(E,F):=\Hom_{D^b(S)}(E,F[i])$.

\begin{proof}[Proof of the backward implication of Mukai-Orlov theorem]
Let 
\begin{align*}
\varphi: \tilde{H}(S,\mathbb{Z}) &\longrightarrow \tilde{H}(S',\mathbb{Z}) \\
(0,0,1) &\longmapsto v\\
(-1,0,0) &\longmapsto v'
\end{align*}
The intersection matrix of $v$ and $v'$ is
$$
U=\begin{pmatrix}
0&1\\ 
1&0
\end{pmatrix}
$$
For simplicity assume that $v$ is of positive rank. ($V=(a,b,c)$, where $a\in
H^0$ is positive.)

The heart of the proof uses the following result by Mukai. There exists a
nonempty moduli space $\mathcal{M}$ of stable sheaves on $S'$ of class $v$
$\implies$ $S'$ is a K3 surface!
$$
0=\chi(v,v)=\underbrace{\Hom(E,E)}_{=1}-\underbrace{\text{ext}^1(E,E)}_{=2}
+\underbrace{\text{ext}^2(E,E)}_{=1}
$$
$v\cdot v'=1\implies \mathcal{M}$ is a fine moduli space. There exists a
universal family $D^b(S \times M)\ni \mathcal{E} \to S' \times M$.

{\bf Claim.} $\Phi_{\mathcal{E}}:D^b(S')\xrightarrow{\cong} D^b(\mathcal{M})$
(general criteria).

Now $\tilde{H}(S,\mathbb{Z})\xrightarrow{\varphi,\cong} \tilde{H}(S',\mathbb{Z}$
$$
\xymatrixrowsep{1em}
\xymatrix{
\tilde{H}(S,\mathbb{Z})\ar[r]^{\varphi,\cong}&\tilde{H}(S',\mathbb{Z})
\ar[r]^{\Phi_{\mathcal{E}}}&\overset{\vee}{H}(M,\mathbb{Z})\\
(0,0,1)\ar@{|->}[r]&  v \ar@{|->}[r]& (0,0,1)\\
(1,0,0)\ar@{|->}[rr]& & (1,0,0)
}
$$
\end{proof}

\section{Semiorthogonal decomposition and Calabi-Yau categories}
\label{section-SOD-and-CY-categories}

\begin{situation}
\label{situation-SOD-and-SC}
$$
\xymatrix{
D^b(Y),\substack{Y\text{ a cubic} \\
\text{4-fold}}\ar[dr]_{\substack{\text{semiorthogonal} \\
\text{decomposition}}}
&  &\substack{\text{moduli}\\ \text{on K3} \\ \text{as HK}}\\
&\text{Ku}_Y\substack{\text{K3} \\ \text{category}}
\ar[ur]_{\text{stability conditions}}
}
$$
\end{situation}

\begin{definition}
\label{definition-SOD}
Let $X$ be a projective smooth variety over $\mathbb{C}$. 
A {\it SOD} $D^b(X)=\left<A_1,\ldots,A_n\right>$ is a sequence $A_i$ full
triangulated subcategory such that
\begin{enumerate}
\item $\Hom(F,G)=0$ for all $F \in A_i$, $G \in A_j$.
\item $\forall F \in D^b(X)$ $\exists 0=F_n\to F_{n-1}\to\ldots\to F_0=F$ 
such that
$\text{Cone}(F_i \to F_{i-1})\in A_i$.
\end{enumerate}
\end{definition}

\begin{exercise}
\label{exercise-}
\begin{enumerate}
\item $D^b(\text{Spec}\mathbb{C})\ni V \simeq \bigoplus_{i}H^{i}(V)[-i]$.
\item For $E \in D^b(X)$ define
\begin{align*}
\phi_E: D^b(\text{Spec}\mathbb{C}) &\longrightarrow D^b(X) \\
V &\longmapsto V\otimes E=\bigoplus_{i}H^i(V)\otimes E[-i]
\end{align*}
Then $\phi_E$ fully faithful $\iff$ $E$ is exceptional
$$
\left(\text{Ext}^p(E,E)=
\begin{cases}
\mathbb{C}\qquad &p=0 \\
0\qquad &p\neq 0
\end{cases}\right)
$$
\begin{definition}
\label{definition-}
$\left<E\right>:=\phi_E(D^b(\text{Spec}\mathbb{C}))\subset D^b(X)$ for
exceptional $E$.

\end{definition}
\item {\bf Example.} $\mathcal{O}(i)$ exceptional on $\mathbb{P}^n$.
(Beilinson.) $D^b(\mathbb{P}^n)
=\left<\mathcal{O},\mathcal{O}(1),\ldots,\mathcal{O}(n)\right>$
\end{enumerate}
\end{exercise}

\begin{definition}
\label{definition-right-admissible}
A triangulated category $A \subset D^b(X)$ is {\it right admissible} if
$\alpha:A\to D^b(X)$ admits a right adjoint. $\alpha^!:D^b(X)\to A$. (I.e.,
$\Hom_{D^b(X)}(\alpha(E),F)\simeq \Hom_A(E,\alpha^!(F))$
\end{definition}

\begin{exercise}
\label{exercise-right-admissible}
Let $E$ be exceptional. $\left<E\right>\subset D^b(X)$ is right admissible.
$$
\alpha^!=R \Hom(E,F) \in D^b(\text{Spec} \mathbb{C})
$$
\end{exercise}

Let
$$
A^\perp:=\{F \in D^b(X):\Hom(E,F)=0, \forall E \in ?\}
$$
\begin{lemma}
\label{lemma-right-admissible}
$A \subset D^b(X)$ is right admissible $\implies$
$D^b(X)=\left<A^\perp,A\right>$.
\end{lemma}

\begin{proof}
\begin{enumerate}
\item Is clear.
\item $0=F_2 \to F_1 \to F_0=F$, $F_1 \in A$, $\text{Cone}(F_1 \to F) \in
A^\perp$. We use adjunction as follows:
$$
\text{id} \in \Hom(\alpha^!(F),\alpha^!(F))=\Hom(\alpha \alpha^!F,F)
$$
We have the following exact triangle:
$$
\xymatrix{
\alpha \alpha^!F\ar[r]^{\text{counit}}&  F\ar[r]& B
}
$$
{\bf Claim.} $B \in A^\perp$. Let $E \in A$ 
$$
\xymatrixcolsep{1em}
\xymatrix{
\cdots\ar[r]&\Hom(\alpha(E),\alpha \alpha^!(F))\ar[dr]^{\cong}\ar[rr]&&
\Hom(\alpha(E),F)\ar[dl]_{\cong}\ar[r]&\Hom(\alpha(E),B)\\
& & \Hom(E,\alpha^!(F))
}
$$
which implies that $\Hom(\alpha(E),B)=0$.

\end{enumerate}
\end{proof}

\begin{definition}
\label{definition-left-mutation}
{\it Left mutation} $\mathbb{L}_A$ is defined as
$$
\alpha \alpha^!F \to F \to \mathbb{L}_AF.
$$
\end{definition}

\begin{exercise}
\label{exercise-left-mutation}
$A=\left<E\right>$.
$$
R \Hom(E,F)\otimes E \xrightarrow{\text{ev}}F \to \mathbb{L}_{\left<E\right>}F
$$
\end{exercise}

{\bf Corollary.} $E_1,\ldots,E_n \in D^b(X)$ exceptional objects with
$\text{Ext}^\bullet(E_i,E_j)=0$ for $i>j$ (we say this is an  {\it exceptional
collection}). Then
$$
D^b(X)=\left<R_x,E_1,\ldots,E_n\right>,\qquad 
R_X=\left<E_1,\ldots,E_n\right>^\perp
$$

\begin{example}
\label{example-Fano}
$X$ Fano, $-K_X=rH$, $H$ ample, $r>0$,
 $\mathcal{O}_X,\mathcal{O}_X(H),\ldots,\mathcal{O}_X((r-1)H)$
an exceptional collection.

($\text{Ext}^\bullet(\mathcal{O}(iH),\mathcal{O}(jH))=
H^{\bullet}(X,\mathcal{O}(j-i)H)=0$) if $-r<j-i<0$. $\implies$
$$
D^b(X)=\left<R_X,\mathcal{O}_X\ldots\mathcal{O}_X(r-i)H\right>
$$
\end{example}

\begin{definition}
\label{definition-Serre-functor}
A {\it Serre functor} for a $\Delta$-category $\mathcal{D}$ is an autoeq
$S_{\mathcal{D}}$ such that
$$
\Hom_{\mathcal{D}}(E,F)^\vee \cong \Hom(F,S_{\mathcal{D}}(E)).
$$
functorially in $E,F \in \mathcal{D}$.
\end{definition}

\begin{remark}
\label{remark-uniqueness-of-Serre-functor}
It is unique if it exists.
\end{remark}

The following example explains why this is called the Serre functor --- it is a
generalization of Serre duality.

\begin{example}
\label{example-Serre-functor-is-Serre-duality}
$X$ smooth projective variety of dimension $n$. Consider
$$
S_{D^b(X)}=(-\otimes \omega_X)[n]
$$
eg. $E$ locally free on $X$.
\begin{align*}
H^{i}(X,E)&=\Hom_{D^b(X)}(\mathcal{O}_X,E[i])\\
& =\Hom_{D^b(X)}(E[i],\omega_X(n])^* \\
&=\Hom(\mathcal{O},E^\vee \otimes \omega_X[n-i])\\
&=H^{n-i}(E^{\vee}\otimes \omega_X)^*
\end{align*}
\end{example}

\begin{definition}
\label{definition-Calabi-Yau-category}
$\mathcal{D}$ is
\begin{itemize}
\item {\it Calabi-Yau category} of dimension $n$ if $S_{\mathcal{D}}\cong[n]$.
(I.e. the Serre functor is just shifting by $n$.)
\item {\it Fractional Calabi-Yau category} of dimension $p/q$ if
$S^q_{\mathcal{D}}\cong[p]$ (where the exponent $q$ just means composing the
functor $S_{\mathcal{D}}$ $q \in \mathbb{Z}$ times).
\end{itemize}
\end{definition}

\begin{theorem}[Kuznetsov]
\label{theorem-Kuznetsov}
$X \subset \mathbb{P}^n$ smooth Fano hypersurface of degree $d \leq n$.
$$
D^b(X)=\left<\text{Ku}_X,\mathcal{O}_X,\ldots,\mathcal{O}_X(n-d)\right>
$$
Then $\text{Ku}_X$ has a Serre functor $S$ with
$S^{d/c}\cong\left[\frac{(n+1)(d-2)}{c}\right]$ where 
 $c=\text{gcd}(d,n+1)$.
\end{theorem}

\begin{example}
\label{example-cubic-3-and-4-folds}
$d=3$, a cubic 3-fold,  $n=4$,  $S^3=[5]$, cubic 4-fold $n=5$,  $S=[2]$.
\end{example}

\begin{theorem}[Kuznetsov]
\label{theorem-Kuznetsov-2}
$X$ cubic 4-fold,
\begin{enumerate}
\item $\exists $ special $X$ such that $\text{Ku}_X \simeq D^b(S)$, where $S$ is
a K3 surface.
\item $X$ very general $\implies$ $\text{Ku}_X \not \simeq D^b(\text{var})$.
\end{enumerate}
\end{theorem}

We shall not prove this theorems. However, by the fourth lecture we may give the
idea of their proofs.

\medskip\noindent
{\bf Conjecture (Kuznetsov).} Cubic 4-fold $X$ is rational if and only if
$\text{Ku}_X \simeq D^b(\text{K3 surface})$.

\medskip\noindent
[KKPY] very general $X$ is not rational.

\medskip\noindent
Let's go back and show how to prove the statement in Example
\ref{example-cubic-3-and-4-folds} about
the cubic 4-fold.

\begin{lemma}
\label{lemma-cubic-4-fold}
Assume that $\mathcal{D}=\left<A^{\perp},A\right>$, $S_{\mathcal{D}}\exists $.
Then 
$S^{-1}_{A^\perp}=\mathbb{L}_A\circ S^{-1}_{\mathcal{D}}|_{A^\perp}$.
\end{lemma}

\begin{proof}
$F,G \in A^\perp$,
\begin{align*}
\Hom_{A^\perp}(F,G)
&\cong\Hom_{\mathcal{D}}(G,S_{\mathcal{D}}(F))^\vee\\
&\cong\Hom_{\mathcal{D}}(S^{-1}_{\mathcal{D}}(G),F)^\vee\\
&\cong \Hom_{\mathcal{D}}(\mathbb{L}_A(\underbrace{S^{-1}_{\mathcal{D}}(G))}
_{\in A^\perp},F)^\vee
\end{align*}
\end{proof}

Now: cubic 4-fold $X \subset \mathbb{P}^5$,
$$
D^b(X)=\left<\text{Ku}_X,\mathcal{O},\mathcal{O}(1),\mathcal{O}(2)\right>
$$
The lemma tells us that
$$
S_{\text{Ku}_X}^{-1}
=\mathbb{L}_{\left<\mathcal{O},\mathcal{O}(1),\mathcal{O}(2)\right>}
\circ (- \otimes \mathcal{O}(3)[-4])
$$

\medskip\noindent
{\bf Key observation 1.} 
$X\overset{i}{\hookrightarrow}\mathbb{P}^5$, for $F \in D^b(X)$
there exists an exact $\Delta$:
$$
i^*i_*F \to F \to F \otimes \mathcal{O}_X(-3)[2]
$$
Assume $F \in \text{Ku}_X$, apply $S^{-1}_{\text{Ku}}$.

$$
\mathbb{L}_{\left<\mathcal{O},\mathcal{O}(1),\mathcal{O}(2)\right>}
\circ (i^* i_* F \otimes \mathcal{O}(3)[-4]) \to S_{\text{Ku}}^{-1}(F)
\to \underbrace{\mathbb{L}_{\left<\ldots\right>}\circ(F[-2])}
_{\cong F[-2]\text{, as }F\in \text{Ku}_X}
$$

\medskip\noindent
{\bf Key observation 2.} $i^*i_*F \otimes \mathcal{O}(3) \in 
\left<\mathcal{O},\mathcal{O}(1),\mathcal{O}(2)\right>$, so the first term
vanishes. (It's enough to show that $i_*F \otimes \mathcal{O}(3) \in
\left<\mathcal{O}_{\mathbb{P}^5},\mathcal{O}_{\mathbb{P}^5}(1),
\mathcal{O}_{\mathbb{P}^5}(2)\right> 
\iff i_*F \in 
\underbrace{\left<\mathcal{O}(-3),\mathcal{O}(-2),\mathcal{O}(-1)\right>}
_{=\left<\mathcal{O},\mathcal{O}(1),\mathcal{O}(2)\right>^\perp} \subset 
D^b(\mathbb{P}^5)$.)

\begin{align*}
R\Hom\left(
\begin{matrix}
\mathcal{O}\\ \mathcal{O}(1)\\ \mathcal{O}(2)
\end{matrix},i_*F
\right)&=0
\\
R\Hom\left(
i^*\begin{matrix}
\mathcal{O}\\ \mathcal{O}(1)\\ \mathcal{O}(2)
\end{matrix},F
\right)&=0\qquad \text{since $F \in \text{Ku}_X$}
\end{align*}
and the left-hand-sides on the last two equations are $\cong$.

\begin{proof}[Proof of Key observation 1 assuming $F$ is a sheaf.]
Take $i^*i_*F \to F \to G$ and consider its pushforward
$$
i_*i^*i_*F \to i_* F \to i_*G
$$
Note that
\begin{align*}
i_*i^*i_*F \simeq i_*F\overset{L}{\otimes}_{\mathcal{O}_{\mathbb{P}^5}}
&\cong \left[ 
\underbrace{i_* F \otimes\mathcal{O}
_{\mathbb{P}^5}(-x)}_{\text{deg}-1} \xrightarrow{\vee}
i_*F\otimes \mathcal{O}_{\mathbb{P}^5}\right] \\
&  \simeq i_*F \oplus i_*F\otimes \mathcal{O}_{\mathbb{P}^5}(-x)[1]\\
&\xrightarrow{\substack{ \text{proj. to}\\\text{$1^{\text{st}}$ factor}}}i_*F\\
&  \implies i_*G \simeq i_* F \otimes \mathbb{P}_{\mathbb{P}^5(-x)[2]}\\
& \implies G \simeq F \otimes \mathcal{O}_X(-x)[2]\qquad i_*\text{ f.f.}
\end{align*}

\end{proof}

\section{Inducing $t$-structure and stability conditions}
\label{section-inducing-t-structure-and-stability-conditions}

\begin{example}
\label{example-cubic-3-fold}
$X$ cubic 3-fold, $D^b(X)=\left<\text{Ku}_X,\mathcal{O},\mathcal{O}(1)\right>$.
\end{example}

\noindent
{\bf Goal.} Use $t$-structure/(weak) stability conditions on $D^b(X)$ to induce
a $t$-structure/stability conditions on $Ku_X$.

\medskip\noindent
\begin{proposition}[Key proposition]
\label{proposition-key}
Let $\mathcal{D}=\left<R,E_1,\ldots,E_n\right>$ be a semiorthogonal
decomposition. Assume that $\mathcal{A} \subset \mathcal{D}$ is the heart of a
bonded $t$-structure such that
\begin{enumerate}
\item $\forall i,$ $E_i \in \mathcal{A}$.
\item $\forall i$, $S_{\mathcal{D}}(E_i) \in \mathcal{A}[i]$.
\end{enumerate}
Then $\mathcal{A}:=\mathcal{A}\cap R$ is the heart of a bounded $t$-structure on
$R$. (Concretely, this means for $F \in R$, view it as an object in
$\mathcal{D}$ and take cohomology with respect to the $t$-structure, we need
$H_{\mathcal{A}}^q(F) \in R$ $\forall q$.)
\end{proposition}

[BLMS] using spectral sequence.

\begin{example}
\label{example-cubic-3-fold-continued}
For $X$ cubic 3-fold and
$D^b(X)=\left<\text{Ku}_X,\mathcal{O},\mathcal{O}(1)\right>$,
\begin{align*}
\left(\textit{Coh}^{-\frac{1}{2}}(X),Z_{\alpha,-\frac{1}{2}}
=\frac{1}{2}\alpha^2H^2 ch_0 -ch_2^{-\frac{1}{2}}+iHch_1^{-\frac{1}{2}}\right)
\end{align*}
\end{example}

\begin{exercise}
\label{exercise-}
\begin{itemize}
\item For all $\alpha>0$, $\mathcal{O}$, $\mathcal{O}(1)$, $\mathcal{O}(-2)[1]$, 
$\mathcal{O}(-1)[1]\in \textit{Coh}^{-\frac{1}{2}}$.
\item For all $\alpha \sim 0$,
\begin{align*}
\mu_{\alpha,-\frac{1}{2}}(\mathcal{O}(-2)[1])
&<\mu_{\alpha,-\frac{1}{2}}(\mathcal{O}(-1)[1])<0\\
&<\mu_{\alpha,-\frac{1}{2}}(\mathcal{O})
<\mu_{\alpha,-\frac{1}{2}}(\mathcal{O}(1)).
\end{align*}
So tilt $\textit{Coh}^{-\frac{1}{2}}(X)$ one more time w.r.t.
$\mu_{\alpha,-\frac{1}{2}}=0$. We can apply Key Proposition 
\ref{proposition-key}.
\end{itemize}
\end{exercise}

\begin{example}
\label{example-cubic-4-fold}
How about cubic 4-fold $Y$? Idea: embed 
$Ku_X\hookrightarrow \underbrace{D^b(\mathbb{P}^3,\mathcal{D}_0)}
_{\text{conic fibration}}$.
\end{example}

\begin{theorem}
\label{theorem-exists-SC-for-cubic-4-fold}
There exist stability conditions on $Ku_Y$  for a cubic 4-fold $Y$.
\end{theorem}

\medskip\noindent
Now we shall give the idea of why Key Proposition \ref{proposition-key} holds.

General question modeling Key Proposition \ref{proposition-key}: 
$F: \mathcal{C} \to \mathcal{D}$ exact functor between triangulated categories.
Say $\mathcal{D}^{heart}$ is the heart of a $t$-structure on $\mathcal{D}$. 
Under what condition is
$$
\mathcal{C}^{heart}=\{E \in \mathcal{C}|F(E) \in \mathcal{D}^{heart}\}
$$
the heart of a $t$-structure on $\mathcal{C}$?

\begin{example}
\label{example-}
Let $X,Y$ be smooth projective and $f:X \to Y$ a finite a morphism.
$$
\xymatrix{
D^b(X)\ar[r]_{f_*} \ar@{^{(}->}[d]
&D^b(Y) \ar@/_{1.5pc}/[l]_{f^*}\ar@{^{(}->}[d]\\
D(Q\textit{Coh}(X))=D_{qc}(X)\ar[r]_{f_*}
&\underbrace{D_{qc}(Y)}_{\substack{\text{admits small} \\ \text{direct sum}}}
\ar@/_{1.5pc}/[l]_{f^*}
}
$$
\end{example}

\begin{situation}
\label{situation-}
$\tilde{\mathcal{C}},\tilde{\mathcal{D}}$ 
triangulated categories admitting direct sums. 
$\mathcal{C}\subset \tilde{\mathcal{C}}$, 
$\mathcal{D} \subset\tilde{\mathcal{D}}$ 
full essentially small triangulated subcategories. 
$F:\tilde{\mathcal{C}}\to \tilde{\mathcal{D}}$ exact functor such that
\begin{itemize}
\item $F$ commutes with direct sums.
\item $F$ has left adjoint 
$\mathcal{G}:\tilde{\mathcal{D}}\to \tilde{\mathcal{C}}$.
\item $F(\mathcal{C}) \subseteq \mathcal{D}$ and if $E \in \tilde{\mathcal{C}}$
is such that $F(E) \in \mathcal{D}$ then $E \in \mathcal{C}$.
\item $G(\mathcal{D}) \subseteq \mathcal{ C}$.
\item For $E \in \mathcal{C}$, if $F(E)=0 \implies E=0$.
\end{itemize}
\end{situation}

\begin{definition}
\label{definition-vanishing-cohomologies-subcategories}
Let $\mathcal{D}^{heart}$ be the heart of a $t$-structure on $\mathcal{D}$.
Define
\begin{align*}
\mathcal{D}^{\leq 0}&=\{E \in \mathcal{D}|H^{i}(E)=0, i>0\}
\mathcal{D}^{\geq 0}&=\{E \in \mathcal{D}|H^{i}(E)=0, i<0\}
\end{align*}
\end{definition}

\begin{theorem}[Polischuk]
\label{theorem-Polischuk}
Let $\mathcal{D}^{heart}$ be the heart of a bounded $t$-structure on
$\mathcal{D}$. If $FG:\mathcal{D} \to \mathcal{D}$ is right $t$-exact 
(i.e. $FG(\mathcal{D}^{\leq 0})\subseteq \mathcal{D}^{\leq 0}$) 
then 
$$
\mathcal{C}^{heart}=\{E \in \mathcal{C}|F(E) \in \mathcal{D}^{heart}\}
$$
is the heart of a bounded $t$-structure on $\mathcal{C}$.
\end{theorem}

\begin{example}
\label{example-right}
$f_*:D^b(X) \to D^b(Y)$, $f_*f^*(E)=E\otimes f_*\mathcal{O}_x$. 
If this is right $t$-exact, then can induce $t$-structures, e.g. 
$f:H \hookrightarrow Y$ hypersurface.
$$
E \to E \otimes f_*\mathcal{O}_H\to E\otimes \mathcal{O}_T(-H)[1]
$$
See recent paper by Chengi Li ``Real reduction …''
\end{example}

\begin{example}
\label{example-Kuznetsov}
$D^b(X)=\left<R,E\right>$, $i:R \to D^b(X)$. See [Kuznetsov]. There exists a
semiorthogonal decomposition of $D_{qc}(X) \supset \tilde{R}$, $i^* \to i$,
$$
\mathbb{L}_{\left<E\right>}=ii^* :D^b(X) \to D^b(X)\qquad \text{right exact?}
$$
\end{example}

The reader is really encouraged to solve the following exercise.

\begin{exercise}
\label{exercise-right-exact}
Let $E \in \mathcal{D}^{heart}$, $S(E) \in \mathcal{D}^{heart}[1]$, then
$\mathbb{L}_E$ is right exact.
\end{exercise}

\noindent
{\bf Key homological algebra ingredient.} Let $\tilde{\mathcal{C}}^{\leq 0}$ 
be the smallest full subcategory of $\tilde{\mathcal{C}}$ that
\begin{itemize}
\item contains $G(\mathcal{D}^{\leq 0})$,
\item is closed under direct sum, extension, positive shift $[n]$, $n>0$.
\end{itemize}
$$
\tilde{\mathcal{C}}^{\geq 0}:=\{E \in \tilde{\mathcal{C}}|
\Hom(\tilde{\mathcal{C}}^{\leq 0}[1],E)=0\}.
$$
Then this is a $t$-structure on $\tilde{\mathcal{C}}$.

\begin{example}
\label{example-curve}
Let $X$ be a smooth projective curve and $p \in X$ a point. Then
$D_{qc}(X)^{\leq 0}$ contains $\mathcal{O}_p$ and is closed under ….
$$
D_{qc}(X)^{\geq 0}j_* \mathcal{O}_U/\mathcal{O}_X[-1] \to 
\mathcal{O}_X \to j_*\mathcal{O}_U
$$
\end{example}

\section{Moduli of objects in derived category}
\label{section-moduli-of-obejects-in-derived-category}

\subsection{Moduli on a variety}
\label{subsection-moduli-on-a-variety}

Let $X$ be a scheme of finite type over $\mathbb{C}$.

\begin{definition}
\label{definition-perfect-complex}
An object $E \in D^b(X)$ is a {\it perfect complex} if locally on $X$, $E$ is
isomorphic to a bounded cx of locally free sheaves
$D_{\text{perf}}(X)\subset D^b(X)$ subcategory of perfect cx. [?]
\end{definition}

\begin{remark}
\label{remark-perfect}
\begin{enumerate}
\item $D_{perf}(-)$ is preserved under derived pullback.
\item $X$ smooth $\implies$ $D_{\text{perf}}(X)=D^b(X)$.
\end{enumerate}
\end{remark}

\begin{definition}
\label{definition-universally-gluable}
Given $T$ scheme of finite type over $\mathbb{C}$ and $E \in
D_{\text{perf}}(X\times T)$, say $E$ is {\it universally gluable} over $T$ if
for all $t \in T(\mathbb{C})$, $\text{Ext}^i(E_t,E_t)=0$ if $i<0$, where  $E_t$
is the derived pullback.
\end{definition}

\begin{example}
\label{example-locally-free}
Locally free on $X \times T$.
\end{example}

We will consider the following functor to formulate the moduli problem:

\begin{definition}
\label{definition-moduli-stack-of-perfectly-univerasally-gluable-objects}
Let $X$ be a smooth projective variety over $\mathbb{C}$. The {\it moduli stack
of perfectly universally gluable objects} is the functor
\begin{align*}
\mathcal{M}_{\text{proj}}(X): (\Sch^{ft}/\mathbb{C})^{op} 
&\longrightarrow \mathit{Gpds} \\
T &\longmapsto  \{E \in D_{\text{perf}}(X\times T)\text{ uniersally gluable}/T\}
\end{align*}

\end{definition}

\begin{theorem}[Lieblich]
\label{theorem-Lieblich}
$\mathcal{M}_{\text{proj}}(X)$ is an algebraic stack locally of finite type over
$\mathbb{C}$.
\end{theorem}

\subsection{Moduli of $\sigma$-semistable objects}
\label{subsection-moduli-of-sigma-semistable-objects}

Let $X$ be a smooth projective variety over $\mathbb{C}$. Fix
$\sigma=(A,Z)\in\text{Stab}_{\Lambda}(X)$, $v:K_0(X) \to \Lambda$. Fix $w \in
\Lambda$, define 
\begin{align*}
\mathcal{M}_{\sigma}(w):(\Sch^{ft}/\mathbb{C})^{op} 
 &\longrightarrow  \mathit{Gpds} \\
T &\longmapsto \left\{ \substack{\text{perfect complex $E$ on $X\times T$ s.t.} 
\\ \text{$\forall t \in T(\mathbb{C})$, $E_t$ is in $\mathcal{A}$}\\
\text{$\sigma$-ss, $V(E_t)=w$}} \right\} 
\end{align*}

\medskip\noindent
{\bf Hard problems.}
\begin{enumerate}
\item Are $\mathcal{M}_{\sigma}^s(w) \subset \mathcal{M}_{\sigma}(w)
\subset \mathcal{M}_{\text{proj}}(X)$.
\begin{enumerate}
\item Open.
\item Finite type?
\end{enumerate}
Known for $X$ surface, $\sigma$ via tilting (Toda). 
\begin{itemize}
\item \text{[}Halper-Leistner-Robotis '25].
\item Proper SC.
\end{itemize}
\item Do there exist maps to schemes
$$
\xymatrix{
\mathcal{M}_{\sigma}(w)&M_\sigma(w)\\
\mathcal{M}_{\sigma}^s(w)\ar[r]\ar@{^{(}->}[u]& M_{\sigma}^s(w)\ar@{^{(}->}[u] 
}
$$
where $M_\sigma(w)$ is projective and paramatrizes s-equivalent classes.
\end{enumerate}

\begin{definition}
\label{definition-good-moduli-space}
Let $\mathcal{M}$ be an algebraic stack, $M$ algebraic space. 
$\pi:\mathcal{M} \to M$ is a {\it good moduli space} if
\begin{enumerate}
\item $\pi_* :\text{QCoh}(\mathcal{M}) \to \text{QCoh}(M)$ is exact.
\item $\mathcal{O}_M \xrightarrow{\simeq }\pi_*\mathcal{O}_{\mathcal{M}}$
\end{enumerate}
\end{definition}

\begin{example}
\label{example-good-moduli-space}
$X$ affine, $G$ linear reductive $G \mathbb{y} X$, $[X/G] \to X/\!/G$ is a good
moduli space. {\bf Exercise:} think this through!
\end{example}

\begin{theorem}[Alper-Halpern-Leistor-Heisloth]
\label{theorem-AHLH}
Good moduli space exists if the stack satisfies $\Theta$-reductivity and
S-completeness.
\end{theorem}

Where, roughly,
\begin{itemize}
\item $\Theta$-reductivity means ``HN filtration specializes'',
\item S-completeness means: two specializations of a family of objects differ by
 ``elementary modification''.
\end{itemize}

\begin{example}[Elementary modification I]
\label{example-elementary-modification}
$\mathbb{P}_0=\mathbb{P}^1 \times \{0\} = \mathbb{P}^1 \times \mathbb{A}^1$,
$$
\xymatrix{
0\ar[r]&  K \ar[r]&  \mathcal{O}_{\mathbb{P}^1\times\mathbb{A}^1}^{\oplus 2}
\ar[rr]\ar@{->>}[rd]&  &  \mathcal{O}_{\mathbb{P}^0}(1) \ar[r]&  0\\
&  &  &  \mathcal{O}_{\mathbb{P}_0^{\oplus 2}}\ar@{->>}[ur]
}
$$
\end{example}

\begin{example}[Elementary modification II]
\label{example-elementary-modification-2}
$C \times \mathbb{A}^1$, $E$ rank 2 vector bundle such that
$$
\xymatrix{
0\ar[r]&\mathcal{L}_1\ar[r]&E|_{co}\ar[r]&\mathcal{L}\ar[r]&0
}
$$
for $\text{deg}\mathcal{L}_1=\text{deg}\mathcal{L}^2$.
$$
\xymatrix{
0\ar[r]&  K \ar[r]& E\ar[rr]\ar@{->>}[rd]
&  &  \mathcal{L}_2 \ar[r]&  0\\
&  &  &  E|_{c_0}\ar@{->>}[ur]
}
$$
\end{example}

\medskip\noindent
{\bf Upshot.} $\mathcal{M}_{\sigma}(w) \to M_{\sigma}(w)$ good moduli exist.

\begin{itemize}
\item Bayer-Macri: there exists a (numerical) divisor class on $M_{\sigma}(w)$
that intersects every curve positively.
\end{itemize}

\subsection{Mukai's theorem for K3 category}
\label{subsection-Mukais-theorem-for-K3-category}

\begin{proposition}
\label{proposition-4-fold-Mukai-lattice}
$X$ cubic 4-fold. Then there exists a Mukai lattice (cf Subsection
\ref{subsection-Mukai-lattice}) $\tilde{H}(Ku_X';\mathbb{Z})$ with
\begin{itemize}
\item Weight 2 Hodge structure,
\item Pairing compatible with $-\chi$.
\end{itemize}
\end{proposition}

\begin{theorem}[BLMNPS]
\label{theorem-BLMNPS}
Let $0 \neq  v \in \tilde{H}_{\text{alg}}(Ku_X;\mathbb{Z})$ primitive, $\sigma$
generic w.r.t. $v$. Then $\mathcal{M}_{\sigma}(v)$ has a good moduli space
$M_{\sigma}(v)\neq \emptyset$ 
smooth projective hyperkähler variety of dimension $v^2+2$,
deformation equivalent to $K3^{\left[ \frac{v+1}{2 } \right] }$.
\end{theorem}

(The deepest part of this theorem is showing that
$M_{\sigma}(v)$ is not empty!)

\medskip\noindent
There exists a rank 2 lattice contained in $\tilde{H}_{alg}(Ku_X;\mathbb{Z})$
for any $X$ $\rightsquigarrow $ $M_\sigma(v)$ deforms in $20$-dimensional
family.

\section{Noncommutative abelian surfaces and Hyperkähler of Kummer type}
\label{section-noncommutative-abelian-surfaces-and-HK-of-Kummer-type}

Joint work with Boyer, Perry, Petrusi ('25?)

\subsection{Polarized hyperkähler}
\label{subsection-polarized-hyperkahler}

Recall Definition \ref{definition-hyperkahler-variety}. In the past lectures we
have seen that the spaces $K3^{[n]}$ are examples.

Let $A$ be an abelian surface 
($A \cong \mathbb{C}^2/\Lambda \hookrightarrow\mathbb{P}^N$). 
The fact that abelian surfaces are not abelian makes $M_H(v)$, the moduli of
$H$-semistable sheaves of topological class $V$ on $A$, not abelian. So we
consider
$$
\xymatrix{
M_H(v)\ar[d]_{\text{alb}}\\
\hat{A}\times A
}
$$
where we must assume that $v$ is primitive, $v^2>0$ and $\sigma/H$ is $v$-gen.
Then $M:=\text{alb}^{-1}(0)$ is a hyperkähler variety called {\it generalized
Kummer}.

Now consider the even cohomology $H^{ev}(Z,\mathbb{Z})$. Let
$$
H^{ev}(A;\mathbb{Z})\supset V^\perp \xrightarrow{\theta,\simeq }H^{2}(M)
$$
(Torelly theorem, […, Verbitsky])

\begin{remark}
\label{remark-positive-definte-rank-2-lattice}
$$
\Lambda=\left<v,\theta^{-1}(H)\right>\subset H^{ev}_{alg}(A,\mathbb{Z})
$$
positive definite rank 2 lattice.
\end{remark}

{\bf Goal.} Fix $\Lambda$. Look for: $\Lambda$-polarized deformation of $D^b(A)$
(i.e. $\Lambda$ remains alg.).

\subsection{Deformations via equivariant categories}
\label{subsection-deformations-via-equivariant-categories}

Consider an involution on an abelian surface, $\mathbb{Z}/2 \mathbb{y} A$. 
This induces an action $ \mathbb{Z}/2 \mathbb{y}D^b(A)$.

There are 16 singular points in the quotient $A/(\mathbb{Z}/2)$. The resolution
$\widetilde{A/(\mathbb{Z}/2)=S}$ is the so-called {\it Kummer K3}. (Another way
to see this is considering the quotient stack $A$ by  $\mathbb{Z}/2$, the result
is ``basically'' equivalent to $S$.)

[BKR] $D^b(S)\cong D^b(A)^{\mathbb{Z}/2}$.

[Elagin] $D^b(A) \cong D^b(S)^{\widehat{\mathbb{Z}/2}}$ where
$\widehat{\mathbb{Z}/2}$ is an involution acting on the derived category of $S$,
and we explain it next: 
there are 16 exceptional divisors $E_1,\ldots,E_{16}\subset S$. The involution 
is
$$
(- \otimes \mathcal{L})\cdot \prod_{i=1}^{16}ST_{\mathcal{O}_{E_i}(-1)}.
$$

\medskip\noindent
{\bf Key 1.} In order to deform $D^b(A)$, we only need to deform
$D^b(S)$ together with its involution.

\medskip\noindent
{\bf Definition/Proposition.} 
\begin{itemize}
\item The derived Kummer lattice $\tilde{K}\subset\tilde{H}^*(S;\mathbb{Z})$ 
is the $(-1)$-eigenspace of $\widehat{\mathbb{Z}/2}$ action on
$\tilde{H}(S;\mathbb{Z})$.
\item Spanned by $(0,2E_i,-1)$.
\end{itemize}

\medskip\noindent
{\bf General theory (for deforming K3 surfaces).} Due to Huybrechts, Toda,
Addington-Thomas. To deform $D^b(S)$ and its $\widehat{\mathbb{Z}/2}$-action is
equivalent to deforming $D^b(S)$ such that $\tilde{K}$ remains algebraic.

What need to remain algebraic under deformation?
\begin{itemize}
\item $\tilde{K}^{16}$ (so that the involution deforms).
\item $\Lambda^2$.
\end{itemize}
[The dimension of the moduli space must then be $22-16=4$.]

\begin{lemma}
\label{lemma-}
$$
\left<\tilde{K},\Lambda\right>\supset U=\begin{pmatrix}
0&1\\ 
1&0
\end{pmatrix}
$$
$$
\underbrace{\left<\tilde{K},\Lambda\right>}_{(2,16)}
=\underbrace{U}_{(1,1)}\oplus\underbrace{\Lambda_T}_{(1,15)}
$$
\end{lemma}

\medskip\noindent
{\bf Key 2.} There exists
\begin{align*}
\Phi: D^b(S) &\longrightarrow D^b(T) \\
\tilde{H}(S) & \tilde{H}(T)
\left<\tilde{K}, U\right>\quad U &\longmapsto H^0 \oplus H^4\\
U \oplus \Lambda_T\quad \Lambda_T &\longmapsto \Lambda'_T \subset H^2
\end{align*}
where $ T$ is a K3 surface and $\tilde{H}(S) \supset \left<
\tilde{K},\Lambda\right>=U\oplus \Lambda_T$. Deform $T$ as a
$\Lambda_T$-polarized K3 surface.

Cine $\tilde{K}$ remains algebraic, invariant deformations w/ $D^b(T)$ take
the equiv. category.

\begin{theorem}
\label{theorem-exists-quasi-finit-dominant-map}
There exists a quasi-finite dominant map
\begin{align*}
U  &\longrightarrow \text{Mod}_{\Lambda_T}^{K3}
\end{align*}
such that $u \in U$. $D^b(T_u)^{\widehat{\mathbb{Z}/2}}=$ 4-dimensional family
of categories deforming.
\end{theorem}

\begin{theorem}
\label{theorem-}
For each $u \in U$, there exists an involution $\widehat{\mathbb{Z}/2}$
stability condition on $D^b(T_u)$.
\end{theorem}

From these two theorems we obtain
$$
\xymatrix{
\substack{\text{Kummer} \\\text{type}\\ \text{HK}}\to 
\mathcal{M}_{\sigma}(D^b(T_u)^{\widehat{\mathbb{Z}/2}},v)\ar[d]^{alb}\\
\text{abelian 4-fold}
}
$$

\begin{definition}
\label{definition-Weil-type}
An abelian 4-fold $W$ is of {\it Weil type} if 
$\mathbb{R}(\sqrt{-d}) \hookrightarrow \text{End}_{\mathbb{Q}}(W)$ for
$d\in\mathbb{Z}_+$ along with some mild condition.
\end{definition}

Let $\Lambda=\left<v,w\right>$. 
Consider the abelian category $\mathcal{A}$ nc abelian surface. [In fact, 
$\text{Aut}^0(\mathcal{A})$ is an abelian 4-fold.]
$$
\xymatrix{
M_\sigma(v)\ar[dr]_{alb_v}&  \text{Aut}^0(\mathcal{A})\ar[l]\ar[r]
\ar@/^{.5pc}/[d]^{c_w}\ar@/^{-0.5pc}/[d]_{c_v}
& M_\sigma(w)\ar[dl]^{alb_w}\\
&\text{Alb}(\mathcal{A})
}
$$
\begin{theorem}
\label{theorem-}
$c_w^{-1} \circ c_v$ is an endomorphism of Weil type.
\end{theorem}
\end{document}
