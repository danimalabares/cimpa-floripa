\input{preamble}
\begin{document}

\title{Derived categories of sheaves}
\maketitle

Minicourse by Lyalya Guseva, CIMPA school Florianópolis 2025.

Notes at 
\href{http://github.com/danimalabares/cimpa-floripa}
{github.com/danimalabares/cimpa-floripa}

\bigskip\noindent

{\bf Abstract.} Derived category of coherent sheaves is a convenient environment
of investigating algebraic geometry of a variety. It provides useful techniques
and gives a perspective point of view. On the level of derived categories one
can see unexpected connections which are not visible on the classical level. I
will try to give an introduction into the techniques of derived categories and
semiorthogonal decomposition and (hopefully) will present many examples of
applications of derived categories in algebraic geometry. 

\bigskip\noindent
\tableofcontents
\bigskip\noindent

{\bf Upshot.} The derived category of an abelian category is the localization of
its homotopy category with respect to the class of quasi-isomorphisms.
\begin{itemize}
\item The objects of $D(\mathcal{A})$ are complexes in $\mathcal{A}$ and the
arrows are equivalence classes of roofs.
\item $D(\mathcal{A})$ is additive but not abelian unless $\mathcal{A}$ is
semisimple.
\item $D(\text{Vect}_k)=$ graded vector spaces.
\item For all $A,B\in \mathcal{A}$,
$$
\text{Hom}_{D(\mathcal{A})}(A,B[i])=
\begin{cases}
0\qquad &i<0 \\
\text{Hom}_{\mathcal{A}}(A,B)\qquad &i=0\\
\text{Ext}^i(A,B)\qquad &i>0
\end{cases}
$$
\item A sequence
$$
\xymatrix{
A\ar[r]^f&B\ar[r]^b&\text{Cone}(f)\ar[r]^a&A[I]\ar[r]&0
}
$$
gives long exact sequences $\text{Hom}^\bullet_{D(\mathcal{A})}(C,\cdot)$ and 
$\text{Hom}^\bullet_{D(\mathcal{A})}(\cdot,C)$, which are the distinguished
triangles.
\end{itemize}

\section{Basic concepts}

\begin{definition}
\label{definition-abelian-category}
An {\it abelian category} $\mathcal{A}$ is a category such that
\begin{enumerate}
\item $\forall X,Y \in \mathcal{A}$, $\text{Hom}{\mathcal{A}}(X,Y)$ is an
abelian group such that $\text{Hom}_{\mathcal{A}}(X,Y)\times\text{Hom}(Y,Z)\to
\text{Hom}(X,Y)$ is bilinear.
\item There is a zero object $0_{\mathcal{A}}$ such that for every $A \in
\mathcal{A}$, $\text{Hom}(0_{\mathcal{A}},A)=\text{Hom}(A,0)=0$.
\item There exist finite products.
\item For every  $f \in \text{Hom}(A,B)$ there exist kernel and cokernel such
that
$$
\xymatrix{
\Ker f\ar[r]&A\ar[r]^{f}&B\ar[r]&\Coker f
}
$$
\end{enumerate}
\end{definition}

\begin{definition}
\label{definition-additive-category}
$\mathcal{A}$ satisfying 1-3 is called  {\it additive category}.
\end{definition}

Examples: abelian groups, $R$-modules, $\text{Coh}(X)$. Non-example: vector
bundles over $X$, it is an additive category but not abelian.

\medskip\noindent

Now consider the category $\text{Com}(\mathcal{A})$, the category of complexes
in $\mathcal{A}$, with quasi-isomorphisms:

\begin{definition}
\label{definition-quasi-isomorphism}
$f\in \text{Hom}_{\text{Com}(\mathcal{A})}(A',B')$ is a {\it quasi-isomorphism}
if it induces isomorphisms $H^i(f):H^i(A^{\bullet})\cong H^i(B^{\bullet})$ for
all $i$.
\end{definition}

\begin{definition}
\label{definition-derived-category}
Let $\mathcal{A}$ be an abelian category. The {\it derived category} of
$\mathcal{A}$ is the localization $\text{Com}(\mathcal{A})[QIS]^{-1}$, where
$QIS$ is the class of quasi-isomorphisms.
\end{definition}

But what is localization? Given a class of morphisms $S$ inside the class of
morphisms of a given category, the localization will allow us to invert these
morphisms.

\begin{definition}
\label{definition-localization-of-category}
Let $\mathcal{C}$ be a category, $S$ a class of morphisms in
$\mathcal{C}$ and $\text{Iso}(\mathcal{C})$ the class of all isomorphisms of
$\mathcal{C}$. The {\it localization} of $\mathcal{C}$ with respect to $S$ is
the category $\mathcal{C}[S]^{-1}$ with an isomorphism 
$$
Q:\mathcal{C}\to \mathcal{C}[S]^{i-\mathcal{A}}
$$
that satisfy
\begin{enumerate}
\item $Q(S) \subset \text{Iso}(\mathcal{C}[S]^{-1})$
\item (Universality.) For any $F:\mathcal{C}\to \tilde{\mathcal{C}}$ and
$Q:\mathcal{C}\to \mathcal{C}[S^{-1}]$ there exists a unique
$G:\mathcal{C}[S^{-1}]\to \tilde{ \mathcal{C}}$ making the diagram commute, i.e.
$F=GQ$.
\end{enumerate}
\end{definition}

\begin{exercise}
\label{exercise-uniqueness-of-localization}
Show that if $\mathcal{C}[S^{-1}]$ exists, it must be unique up to isomorphism.
\end{exercise}

\begin{definition}
\label{definition-bounded-derived-category}
If instead of $\text{Com}(\mathcal{A})$ we consider
$$
\text{Com}^b(\mathcal{A}):=\{A^{\bullet} \in \text{Com}(\mathcal{A}):
A^i=0\forall  |i|\gg 0\}
$$
Then we obtain the {\it bounded derived category} $D^b(\mathcal{A})$.
\end{definition}

\section{Motivation}
\label{section-motivation}

Let $X$ be a smooth projective variety over $\mathbb{C}$. Then
we can associate $X\rightsquigarrow \text{Coh}(X)$.

\begin{theorem}[Gabriel]
\label{theorem-Gabriel}
Let $X$ and $Y$ be smooth projective varieties. If $\text{Coh}(X)\cong
\text{Coh}(Y)$ then $X\cong Y$.
\end{theorem}

This means we may reconstruct $X$ from its coherent sheaf category. But it is
not such an interesting invariant. It's more interesting the derived category.
So the motivation for studying derived categories are:

\begin{enumerate}
\item (Beilinson 1979.) $\text{Coh}(\mathbb{P}^n)\hookrightarrow
D^b(\mathbb{P}^n)$ gives good information about coherent sheaves on
$\mathbb{P}^n$.
\item (Mukai.) If $\mathcal{A}$ is an abelian variety and $\hat{\mathcal{A}}$ is
the dual abelian variety, $D^b(\mathcal{A})\cong D^b(\hat{\mathcal{A}})$.
\item (Bondal, Orlov, 1990's.) Let $X$ be a smooth projective variety with an
ample or anti-ample canonical bundle $\omega_X$. Suppose that $D^b(X)\cong
D^b(Y)$ for a smooth projective variety $Y$. Then $Y \cong X$.
\end{enumerate}

\begin{remark}
\label{remark-derived-category-with-tensor-product}
If you consider not only the bounded derived category $D^b(X)$ but consider it
with the tensor product $D^b \times D^b(X) \xrightarrow{\otimes} D^b(X)$ allow
to reconstruct $X$.
\end{remark}

I hope you are now convinced this is interesting.

\section{Explicit description of localization under some assumption}
\label{section-explicit-description-of-localization}

If $S$ satisfies some properties then $\mathcal{C}[S^{-1}]$ can be explicitly
described. The conditions are:

\begin{enumerate}
\item All identities should belong to $S$ and if any two morphisms from $f,g$
and $f \circ g$ (whenever composition is defined) belong to $S$, then all three
belong to $S$.
\item For any diagram 
$$
\xymatrix{
X\ar[d]^{f}&W\ar@{.>}[l]_{\tilde{s}\in S}\ar@{.>}[d]\\
Z&\ar[l]^{s\in S}Y
}
$$
can be completed to a commutative square.
\item If for $f,g\in \text{Hom}_{\mathcal{C}}(A,B)$ and $u:B \to U$ such athat
$u\circ f=u\circ g$ there exists
$t:C \to A$ such that $ft=gt$, that is,
$$
\xymatrix{
C\ar[r]^{t}&A\ar[r]^{f,g}&B\ar[r]^{u}&U
}
$$
\end{enumerate}

The quasi-isomorphisms don't satisfy condition 2.

\begin{proposition}
\label{proposition-caracterization-of-derived-category}
Let $\mathcal{C}$ be a category and $S\subset \text{Mor}(\mathcal{C})$ satisfy
right Ore conditions. Then $\mathcal{C}[S^{-1}]$ can be described as follows:
\begin{itemize}
\item $\text{Ob}(\mathcal{C}[S^{-1}]=\text{Ob}(\mathcal{C})$.
\item $\text{Mor}_{\mathcal{C}[S^{-1}]}(A,B)=\{\text{equivalence classes of
roofs}\}$, where a roof is
$$
\xymatrix{
& C\ar[dl]^{s\in S}\ar[dr]^{f}\\
A&&B
}
$$
where $fs^{-1}\sim gt^{-1}$ if there exists a commutative diagram
$$
\xymatrix{
& & \tilde{W} \ar[ld]^{a \in S} \ar[dr]^{b \in S} & & \\
& C \ar[ld]^{s \in S} \ar[rr]^{f} & & \tilde{C} \ar[rd]^{t} & \\
A & & & & B
}
$$
and composition is defined as follows:
$$
\xymatrix{
& & \tilde{W}\ar[ld]^{a \in S}\ar[dr]^{b \in S}\\
&Z\ar[ld]^{s \in S}\ar[dr]^{f} & &  \tilde{Z}\ar[dr]\ar[dl]^{t}\\
A & &B & & C
}
$$
where
$$
(fs^{-1})\circ (gt^{-1})=(sa^{-1})\circ(gb)
$$
\end{itemize}
\end{proposition}

$\mathcal{C}[S^{-1}]$ always exists but morphisms are difficult to describe. The
proof of the proposition is an exercise:

\begin{proof}
\begin{enumerate}
\item This is an equivalence relation.
\item Composition doesn't depend on $a$ and $b$.
\item Composition doesn't depend on the choice of the roof in equivalence
classes.
\item Composition is associative.
\item $\text{Id}_X \times \text{Id}_X^{-1}$ is identity morphism.
\end{enumerate}
\end{proof}

{\bf Definition of} $Q:\mathcal{C} \to \mathcal{C}[S^{-1}]$. (Missing.)

{\bf Universal property.}

$$
\xymatrix{
\mathcal{C}\ar[r]^{F}\ar[dr]^{Q}&  \tilde{\mathcal{C}}\\
&\mathcal{C}[S^{-1}]\ar@{.>}[u]
}
$$
defines a universal property.

\begin{proposition}
\label{proposition-Ore-conditions}
Let $\mathcal{C}$ be an additive category and $S$ satisfy right Ore conditions.
Then $\mathcal{C}[S^{-1}]$ is also additive and $Q:\mathcal{C} \to
\mathcal{C}[S^{-1}]$ is additive.
\end{proposition}

\begin{proof}
Some indications, then exercise.
\end{proof}

The point is that with Ore conditions you can really define what is addition of
fractions.

\section{Ore conditions for category of complexes}
\label{section-Ore-conditions-for-category-of-complexes}

Let $\mathcal{C}=\text{Com}(\mathcal{A})$ and $S=\text{QIS}$. Then QIS {\bf does
not} satisfy Ore conditions: second and third properties fail. But we can
consider instead the homotopy category $\mathcal{H}(\mathcal{A})$. But what is 
that.

\begin{definition}
\label{definition-homotopy-category}
\begin{enumerate}
\item 
A morphism $f \in \text{Hom}_{\text{Com}\mathcal{A}}(A^{\bullet},B^{\bullet})$
is {\it homotopically equivalent to zero}, i.e. $f \sim_{\text{hom}}0$ if for
all $i\in \mathbb{Z}$, $h^i:A^i \to B^i$ such that 
$f^i=d_B^{-1}\circ h^i \circ d_A^i$.
$$
\xymatrix{
A^{i-1}\ar[r]&
A^{i}\ar[r]\ar@{.>}[dl]^{h^i}\ar[d]^{f^i}&
A^{i+1}\ar@{.>}[dl]^{h^{i+1}}\\
B^{i-1}\ar[r]& B^i\ar[r]& B^{i+1}
}
$$
\item $f\sim_{\text{hom}}g$ are {\it homotopically equivalent} if $f-g
\sim_{\text{hom}}0$.
\item The {\it homotopy category}  $\mathcal{H}(\mathcal{A})$ of $\mathcal{A}$
has objects $\text{Ob}(\mathcal{H}(\mathcal{A})=\text{Com}(\mathcal{A})$ and
morphisms
$$
\text{Hom}_{\mathcal{H}(\mathcal{A})}(A^{\bullet},B^{\bullet})
=\text{Hom}_{\text{Com}\mathcal{A}}(A,B )\Big/\{f\sim_{\text{hom}}0\}.
$$
\end{enumerate}
\end{definition}

\begin{exercise}
\label{exercise-homotopy-category}
\begin{enumerate}
\item $f\sim_{\text{hom}}0$, $H^i(f):H^i(A)\xrightarrow{0}H^i(B)$.
\item Morphisms homotopic to zero form an ideal in $\text{Com}(\mathcal{A})$,
that is, abelian subgroups w.r.t. compositions.
\item $\mathcal{H}(\mathcal{A})$ is an additive category and
$\text{Com}(\mathcal{A}) \to \mathcal{H}(\mathcal{A})$ is additive.
\end{enumerate}
\end{exercise}

{\bf Objectives.}
\begin{enumerate}
\item QIS satisfy the Ore conditions in $\mathcal{H}(\mathcal{A})$.
\item $D(\mathcal{A})=\text{Com}(\mathcal{A})[\text{QIS}^{-1}]
\cong \mathcal{H}(\mathcal{A})[\text{QIS}^{-1}]$. This will allow to describe
$D(\mathcal{A})$ explicitly as an additive category.
\item $\mathcal{H}(\mathcal{A})$ and $D(\mathcal{A})$ are not abelian, but
triangulated!
\end{enumerate}

When we lose the property of an abelian category, we lose the notion of exact
sequence.

{\bf Goal.} Now we shall prove that QIS satisfy right Ore conditions on
$\mathfrak{X}(f)$ and $D(\mathcal{A})\cong \mathcal{H}[\text{QIS}]^{-1}$.

\begin{definition}
\label{definition-shift-functor}
The {\it shift functor} is given by
\begin{align*}
[1]: \text{Com}(\mathcal{A}) &\longrightarrow \text{Com}(\mathcal{A}) \\
A^\bullet &\longmapsto (A^\bullet[1])
\end{align*}
where
$$
(A^\bullet[1])^i=A^{i+1},\qquad  d_{A[1]}=-d_A
$$
\end{definition}

\begin{definition}
\label{definition-mapping-cone}
For a morphsm $f \in
\text{Hom}_{\text{Com}(\mathcal{A})}(A^\bullet,B^{\bullet})$, its {\it mapping
cone} is the complex
$$
\text{Cone}(f):=B^{\bullet}\oplus A^{\bullet}[1]
$$
with
\begin{align*}
d_{\text{Cone}(f)}^i: A^{i+1}\oplus B^{i} &\longrightarrow A^{i+2}\oplus B^{i+1} \\
(a^{i+1},b^i) &\longmapsto (-d_Aa^{i+1},d_Bb^i+f(a^{i+1}))
\end{align*}
\end{definition}

\begin{exercise}
\label{exercise-cone-is-complex}
$\text{Cone}(f) \in \text{Com}(\mathcal{A})$.
\end{exercise}

Exact sequence in $\text{Com}(\mathcal{A})$:
$$
\xymatrix{
0\ar[r]&B\ar[r]&\text{Cone}(f)\ar[r]&A^{\bullet}[1]\ar[r]&0
}
$$
\begin{definition}
\label{definition-contractible-complex}
$A^\bullet \in \mathcal{H}(\mathcal{A})$ is {\it contractible} if $\mathcal{A}
\cong 0$ in $\mathcal{H}(\mathcal{A})$, that is,
$\text{Id}_{A^{\bullet}}\sim_{\text{hom}}0$.
\end{definition}

\begin{exercise}
\label{exercise-isomorphism-iff-cone-contractible}
$f \in \text{Hom}(A^{\bullet},B^{\bullet})$ is isomorphism in
$\mathcal{H}(\mathcal{A})$ if and only if $\text{Cone}(f)$ is contractible.
\end{exercise}

\begin{definition}
\label{definition-Hom-of-complexes}
Let  $A^{\bullet},B^{\bullet}\in \text{Com}(\mathcal{A})$. Then we define
$\underline{\text{Hom}}(A^{\bullet},B^{\bullet}) \in \text{Com}(\text{Ab})$,
$$
\text{Hom}^i(A,B)=\prod_{n \in \mathbb{Z}}\text{Hom}(A^n,B^{n+i})
$$
\begin{align*}
d^i_{\underline{\text{Hom}}}: \underline{\text{Hom}}(A,B)^i &\longrightarrow
\underline{\text{Hom}}(A,B)^{i+1} \\
f=\prod f^n &\longmapsto g
\end{align*}
and $g^n=d_Bf^n-(-1)^i f^{n+1}d_A$.
\end{definition}

\begin{exercise}
\label{exercise-Hom}
\begin{enumerate}
\item Check that $\underline{\text{Hom}}$ is a complex.
\item $\Ker
d^i_{\underline{\text{Hom}}}
=\text{Hom}_{\text{Com}(\mathcal{A})}(A^{\bullet},B^{\bullet}[i])$.
\item $\text{Im}d^{i-1}_{\text{Hom}}=\{fsim_{\text{hom}}0\}$
\end{enumerate}
\end{exercise}

Notation:
$\text{Hom}^i(A^{\bullet},B^{\bullet})=\text{Hom}(A^{\bullet},B^{\bullet}[i])$. 

\begin{lemma}
\label{lemma-cone-long-exact-sequence}
Consider the following composition
$$
\xymatrix{
A^\bullet\ar[r]_f&  B^{\bullet}\ar[r]^b&  \text{Cone}(f)\ar[r]^a &
A^{\bullet}[1]
}
$$
(which is your first {\it distinguished triangle in life}) induces the following
long exact sequence for all $C^{\bullet \in \text{Com}(\mathcal{A})}$:
$$
\xymatrix{
\cdots\ar[r]
&\text{Hom}^i_{\mathcal{H}(\mathcal{A})}(C^{\bullet},A^{\bullet})\ar[r]
&\text{Hom}_{\mathcal{H}}(\mathcal{A})^i(C^{\bullet},B^{\bullet})\ar[r]&\\
\ar[r]&\text{Hom}^i_{\mathcal{H}(\mathcal{A})}(C,\text{Cone}(f))\ar[r]
&\text{Hom}^{i+1}_{\mathcal{H}(\mathcal{A})}(C^{\bullet},A^{\bullet})\ar[r]
&\cdots
}
$$
and the same happens for
$\text{Hom}^i_{\mathcal{H}(\mathcal{A})}(\cdot,C^{\bullet})$.
\end{lemma}

It is very important that we went in $\mathcal{H}(\mathcal{A})$!

\begin{proof}[Idea of proof]
(The long exact sequence is the cohomology exact sequence associated
to a short exact sequence. Also some exercise from the past is hidden in the
details.)

The short exact sequence
$$
\xymatrix{
0\ar[r]&B^{\bullet}\ar[r]&\text{Cone}(f)\ar[r]&A[1]\ar[r]&0
}
$$
splits. This gives
$$
\xymatrix{
0\ar[r]&B^i\ar[r]&B^i\oplus A^{i+1}\ar[r]&A^{i+1}\ar[r]&0
}
$$
Which in turn gives
$$
\xymatrix{
0\ar[r]&\underline{\text{Hom}}(C^{\bullet},B^{\bullet})
\ar[r]&\underline{\text{Hom}}(C^{\bullet},\text{Cone}(f))
\ar[r]&\underline{\text{Hom}}(C^{\bullet},A[1]\ar[r]&0
}
$$
and {\it then} we take the cohomology long exact sequence.
\end{proof}

\begin{exercise}
\label{exercise-simple}
$bf=ab=fa[-1]=0$ in $\mathcal{H}(\mathcal{A})$.
\end{exercise}

\begin{exercise}
\label{exercise-right-Ore-conditions-not-in-Com}
Prove that the right Ore conditions are not satisfied in
$\text{Com}(\mathcal{A})$.
\end{exercise}

\begin{proposition}
\label{proposition-QIS-satisfy-right-Ore-conditions-in-homotopy-category}
QIS satisfy the right Ore conditions in $\mathcal{H}(\mathcal{A})$.
\end{proposition}

\begin{proof}
In lecture we proved condition 1 and part of condition 2 of the right Ore
conditions, see Subsection
\ref{subsection-explicit-description-of-localization}. Finishing the proof was
left as exercise, including Ore condition 3.
\end{proof}

\bigskip\noindent
Now we shall show that
$$
D(\mathcal{A})\cong \mathcal{H}(\mathcal{A})[\text{QIS}]
$$
$$
H:\text{Com}(\mathcal{A}\to \mathcal{H}(\mathcal{A})
$$
\begin{proposition}
\label{proposition-localization-functor}
The localization functor
$$
\xymatrix{
Q:\text{Com}(\mathcal{A})\ar[d]_H\ar[r]
& D(\mathcal{A})\cong\mathcal{H}(\mathcal{A})[\text{QIS}]^{-1}\\
\mathcal{H}(\mathcal{A})\ar[ur]_{Q'}
}
$$
can be be decomposed as $Q' \circ H$. Moreover $Q' \cong
\mathcal{H}(\mathcal{A})$, where
$Q_{\mathcal{H}(\mathcal{A})}:\mathcal{H}(\mathcal{A}) \to
\mathcal{H}(\mathcal{A})[\text{QIS}]^{-1}$.
\end{proposition}

\begin{proof}
Done in class, with some exercises for us to complete.
\end{proof}

This shows that the derived category is equivalent to the homotopy category with
localization. We have the following corollary:

\begin{lemma}
\label{lemma-}
We can specifically describe
$$
D(\mathcal{A}): \text{Ob}(D(\mathcal{A}) = \text{Com}(\mathcal{A})
$$
$$
\text{Hom}_{D(\mathcal{A})}(A^{\bullet},B^{\bullet})
=\{\text{equivelnce classes of roofs}\}
$$
Recall that a roof is
$$
\xymatrix{
& C\ar[dl]_{\text{QIS}\ni S}\ar[dr]^{f}\\
A& & B
}
$$
\end{lemma}

\bigskip\noindent
Now we shall show that $D(\mathcal{A})$ is additive. (Or have we already proved
this? ie. from the corollary?)

\bigskip\noindent
Why $\mathcal{H}(\mathcal{A})$ and $D(\mathcal{A})$ are not abelian: there are
not so many injective and surjective morphisms.

\begin{exercise}
\label{exercise-no-injective-surjective-morphisms-in-derived-categories}
$f \in \text{Hom}_{\mathcal{H}(\mathcal{A})} (A^{\bullet},B^{\bullet})$ is
injective if and only if there exists  $g: B \to A^{\bullet} \in
\text{Hom}_{\mathcal{H}(\mathcal{A})} (B^{\bullet},A^{\bullet})$ such that $g
\circ f=\text{id}_A$. Formulate the same for surjections.
\end{exercise}

\section{Some properties of the derived category}
\label{section-some-properties-of-the-derived-category}

For every $i \in \mathbb{Z}$ define
\begin{align*}
\mathcal{A} &\longrightarrow D(\mathcal{A}) \\
A &\longmapsto A[i], \quad \{0 \to \overset{i}{A} \to 0\}
\end{align*}

So $A[i]$ is {\it that} complex. It is ``concentrated in degree i''

{\bf Goal:} understand morphisms between $M,N[i]$.

\begin{definition}
\label{definition-embedding-in-Com}
Let $A \in \text{Com}(\mathcal{A})$. For all $n$, 
$$
\tau_{\leq n}(A^{\bullet})^i:=\begin{cases}
A^i\qquad &i<n \\
\Ker d^n\qquad &i=n\\
0,\quad &i>n
\end{cases}
$$ 
then 
$$
\tau_{\geq  n}(A^{\bullet})^i=\begin{cases}
0\qquad
 &i<n \\
A^n/\text{Im}d^{n-1}\qquad &i=n\\
A^i,\qquad &i>n
\end{cases}
$$
and $A^{\bullet}\to\tau_{\geq n}A^{\bullet}$.
\end{definition}

Then
$$
\tau_{\leq n}:\text{Com}(\mathcal{A}) \to \text{Com}(\mathcal{A})
$$
defines a functor $\tau_{\leq n}(\text{QIS})\subset \text{QIS}$ 
where 
$$
\tau_{\leq }(f \sim _{\text{hom}}0) \subset \sim_{\text{hom}}0
\implies
$$
\begin{align*}
\tau_{\leq n}: \mathcal{H}(\mathcal{A}) &\longrightarrow
\mathcal{H}(\mathcal{A}) \\
D(\mathcal{A}) &\longmapsto D(\mathcal{A})
\end{align*}
the same for $\tau_{\geq n}$.

\medskip
The following lemma allows us to understand how are morphisms behaved.

\begin{lemma}
\label{lemma-Homs}
Let $A,B \in \mathcal{A}$. Then
 \begin{enumerate}
\item $\text{Hom}_{D(\mathcal{A})}(A,B[-i])=0$ for all $i>0$.
\item $\text{Hom}_{\mathcal{A}}(A,B)=\text{Hom}_{D(\mathcal{A})}(A,B)$.
\end{enumerate}
In particular, $\mathcal{A} \hookrightarrow A(\mathcal{A})$ is fully-faithful.
\end{lemma}

Now we introduce a notion of Ext, which will recover the usual Ext (derived
functors).

\begin{definition}
\label{definition-Ext}
For $i>0$,
$$
\text{Ext}^i_{\mathcal{A}}(A,B)=\text{Hom}_{D(\mathcal{A})}(A,B[i]).
$$
\end{definition}
\begin{definition}[Yoneda]
\label{definition-Yoneda}
For $i>0$ let
$$
\text{Ext}_Y^i(A,B):=\{\substack{\text{equivalence classes of exact sequences}\\
B\to K^{i-1}\to K^{i-2}\to \ldots \to K^0 \to A}\}
$$
and say that two such sequences are equivalent if there is a commutative diagram
$$
\xymatrix{
B\ar[r]&  K^{i-1}\ar[r]\ar[d]&  K^{i-2}\ar[r]\ar[d]& \cdots\ar[r]\ar[d]
&K^0\ar[r]\ar[d]&A\\
B\ar[r]&  \tilde{K}^{i-1}\ar[r]&  \tilde{K}^{i-2}\ar[r]& \cdots\ar[r]
&\tilde{K}^0\ar[r]&A
}
$$
and are equivalent if there exists a sequence of equivalent elements
$$
K \sim K_0 \sim \ldots \sim \tilde{K}
$$
\end{definition}

\begin{proposition}
\label{proposition-Ext}
$$
\text{Ext}_{\mathcal{A}}^i(A,B) \cong \text{Ext}_Y^i(A,B)
$$
\end{proposition}

\begin{proof}
Sketched in class.
\end{proof}

\begin{exercise}
\label{exercise-derived-and-homotopy-categories-abelian-characterization}
$D(\mathcal{A})$ and  $\mathcal{H}(\mathcal{A})$ are abelian if and only if
$\mathcal{A}$ is semisimple, that is, all exact sequences in $\mathcal{A}$
split. If $\mathcal{A}$ is semisimple, then $D(\mathcal{A})\cong \bigoplus_{i
inn \mathbb{Z}}\mathcal{A}[-i]$.
\end{exercise}

\section{Triangulated categories}
\label{section-triangulated-categories}

\begin{definition}
\label{definition-triangulated-category}
An additive category $\mathcal{T}$ is a {\it triangulated category} if it has
the following structures:
\begin{itemize}
\item $[1]:\mathcal{T} \to \mathcal{T}$ autoequiv.
\item Class of sequences of the form
$$
\xymatrix{
A\ar[r]&B\ar[r]&C\ar[r]&A[1]
}
$$
which are called {\it distinguished triangles}, and satisfy the following
axioms:
\begin{itemize}
\item (Axiom 1.)
\begin{itemize}
\item $\xymatrix{
A\ar[r]^{\text{id}}&A \ar[r]& 0\ar[r]&A[1]
}$ is a distinguished triangle.
\item Any triangle $A \to B \to C \to A[I]$ isomorphic to a distinguished
triangle is a distinguished triangle.
\item Any morphism $(A \xrightarrow{f}B)\in \mathcal{T}$ can be completed to a
distinguished triangle $A\xrightarrow{f}B \to \mathcal{K} \to A[I]$, and
$\mathcal{K}$ is called the {\it cone} of $f$.
\end{itemize}
\item (Axiom 2.) A triangle
$$
\xymatrix{
A\ar[r]^f &  B \ar[r]^g &  C\ar[r]^h & A[1]
}
$$
is distinguished if and only if
$$
\xymatrix{
B\ar[r]^g &  C \ar[r]^g &  A[1]\ar[r]^{f[1]} & B[1]
}
$$
is distinguihsed.

\item (Axiom 3.) For any two distinguished triangles
$$
\xymatrix{
A\ar[r]^f\ar[d]_a &  B \ar[r]\ar[d]^b &  C \ar@{.>}[d]^{u}\ar[r]& 
A[1]\ar[d]^{a[1]}\\
A'\ar[r]_{f'}& B'\ar[r]& C'\ar[r]&A'[1]
}
$$
and any $a,b$ such that $bf=f'a$ there exists $u$ that makes the above diagram
commute (it is not unique!)

\item (Octahedral axiom.) Any two distinguished triangles with a common vertex 
of the form 
$$
\xymatrix{
X\ar[r]^u& Y\ar[r]^v\ar[d]& Z\ar[r]&X[1]\\
& Y'\ar[d]\\
& W\ar[d]\\
& Y[1]
}
$$
can be
completed to the following commutative diagram:
$$
\xymatrix{
X\ar[d]_{=}\ar[r]^u& Y\ar[r]^v\ar[d]&  Z\ar[r]&  X[1]\ar[d]^=\\
X\ar[r]&  Y'\ar[d]\ar[r]&  Z'\ar[d]\ar[r]& X[1]\ar[d]^{u[1]}\\
& W\ar[r]^=\ar[d]_h & W\ar[r]^h\ar[d]&  Y[1]\\
& Y[1]\ar[r]_{v[1]}&Z[1]
}
$$
\end{itemize}
\end{itemize}
\end{definition}

{\bf Properties.} Let $X\xrightarrow{f}Y\xrightarrow{g}Z\xrightarrow{h}X[1]$ be
a distinguished triangle.

\begin{lemma}
\label{lemma-properties-of-distinguished-triangles}
\begin{enumerate}
\item $gf=hg=f[1]h=0$.
\item It induces the following sequence
$$
\xymatrix{ 
\cdots\ar[r]&Z[-1]\ar[r]& X\ar[r]& Y\ar[r]& Z\ar[r]&X[1]
\ar[r]&Y[1]\ar[r]& Z[1]\ar[r]&\cdots
}
$$
such that any sequence of length $h$ (contained in it?) is a distinguished
triangle.
\end{enumerate}
\end{lemma}

\begin{proof}
$$
\xymatrix{
Y\ar[r]^g\ar[d]_=& Z\ar[d]_=\ar[r]^h& C[1]\ar[r]^{f[1]}\ar[d]&
Y[1]\ar[d]^{g[1]}\\
Z\ar[r]^{=}&Z\ar[r]& 0\ar[r]&Z[1]
}
$$
and by axiom 3 we get $gf=0$. By Axiom 2 we get the second statement.
\end{proof}

\begin{lemma}
\label{lemma-long-exact-sequences-for-distinguished-triangles}
If $X \to Y \to Z \to X[1]$ is a distinguished triangle, then for every $U \in
\mathcal{T}$ we have the following long exact sequences:
$$
\xymatrix{
\cdots\ar[r]&\text{Hom}(U,X[i])\ar[r]&\text{Hom}(U,Y[i]\ar[r]&
\text{Hom}(U,Z[i])\ar[r]&\text{Hom}(U,X[i+1]\ar[r]&\cdots
}
$$
and the same happens for $\text{Hom}(\cdot,U)$
\end{lemma}

\begin{proof}
Done in lecture, uses previous lemma and Axiom 3.
\end{proof}

We have the following corollary:

\begin{lemma}
\label{lemma-two-isomorphisms-give-an-isomorphism}
If in Axiom 3 two morphisms are isomorphisms, then the constructed one is also
an isomorphism.
\end{lemma}

\begin{proof}
Exercise. {\bf Hint.} Use Yoneda lemma.
\end{proof}

\begin{exercise}
\label{exercise-inverses-in-distinguished-triangles}
If $A\xrightarrow{f}B\xrightarrow{g}C\xrightarrow{h}A[1]$ is a distinguished
triangle, then $h=0$ if and only if $f$ has a left inverse, if and only if $g$
has a right inverse. 
\end{exercise}

As a corollary,

\begin{lemma}
\label{lemma-cone-of-morphism-is-unique}
The cone of a morphism is unique up to isomorphism.
\end{lemma}

However, the cone is not functorial, which is a problem.

\medskip\noindent
Now we want to show that the homotopy category $\mathcal{H}(\mathcal{A})$ of an
abelian category $\mathcal{A}$ has a triangulated structure. 
We need to say which are the distinguished triangles: but we already did this
for the derived category: they are
$$
\xymatrix{
A\ar[r]^f&B\ar[r]^b&\text{Cone}(f)\ar[r]^a&A[I]\ar[r]&0
}
$$

\begin{theorem}
\label{theorem-homotopy-category-is-triangulated}
$\mathcal{H}(\mathcal{A})$ is triangulated.
\end{theorem}

\begin{proof}
It looks like Axiom 1 should be immediate from our construction. Also Axiom 2
(indeed, we have a notion of $\text{Cone}(f)$ which we obviously expect to
satisfy the property of the cone $\mathcal{K}$. Axiom 3 uses an exercise using
the notion of cylinder. Proving the octahedral axiom is also an exercise with a
big hint.
\end{proof}

{\bf $D(\mathcal{A})$ is triangulated.} Let $\mathcal{C}$ be triangulated, and
$S$ a class of morphisms satisfying the right Ore conditions. We know that
$\mathcal{C}[S]^{-1}$ is an additive category, but when is it triangulated?

\begin{definition}
\label{definition-compatible-class-of-morphisms}
$S$ is {\it compatible} with a triangulated structure if
\begin{enumerate}
\item $s \in S$, $S[1]\in S$,
\item In Axiom 3, if $a,b \in S$ then we can choose $u \in S$.
\end{enumerate}
\end{definition}

\begin{proposition}
\label{proposition-compatible-class-gives-triangulated-localization}
It $\mathcal{T}$ is triangulated an $S$ satisfies the right Ore conditions and
is compatible with the triangulated structure of $\mathcal{T}$ then
$\mathcal{T}[S]^{-1}$ is triangulated.
\end{proposition}

\begin{proof}
Axioms 1-3 proved, Octahedral axiom is exercise with hint.
\end{proof}

We finally obtain:

\begin{lemma}
\label{lemma-derived-category-is-triangulated}
$D(\mathcal{A})$ is triangulated.
\end{lemma}

\begin{proof}
Two or three lines, uses 5 lemma, uses long exact sequences of $\text{Hom}$.
\end{proof}

\section{Derived functors}
\label{section-derived-functors}

\begin{definition}
\label{definition-additive-exact-functor}
An additive functor $F:\mathcal{T} \to \mathcal{T}'$ between triangulated
categories is {\it exact} if
\begin{enumerate}
\item $F \circ [1]_{\mathcal{T}}=[1]_{\mathcal{T}'}\circ F$.
\item Maps distinguished triangles to distinguished triangles.
\end{enumerate}
\end{definition}

\medskip\noindent
Here are some bounded notions:
\begin{align*}
\text{Com}^\bullet(\mathcal{A})&\\
\text{Com}^+(\mathcal{A})&=\{A\in \text{Com}(\mathcal{A})
|A^i=0,\quad i\ll 0\}\\
\text{Com}^-(\mathcal{A})&=\{A\in \text{Com}(\mathcal{A})|i \gg 0\}\\
\text{Com}^b(\mathcal{A})&=\{A \in \text{Com}(\mathcal{A})||i|\gg 0\}
\end{align*}
each of which gives a derived category $D^*$ for $*=b,+$ or $-$.

Now consider an exact functor of abelian categories $F:\mathcal{A} \to
\mathcal{B}$. We would like to obtain a functor
$D^*(\mathcal{A}) \to D^*(\mathcal{B})$.

In general functors are not exact. Thus our goal now is:
\begin{itemize}
\item From a left exact functor $F:\mathcal{A} \to \mathcal{B}$ obtain
$RF:D^+(\mathcal{A})\to D^+(\mathcal{B})$,
\item From a right exact functor $G:\mathcal{A}\to \mathcal{B}$ obtain
$LG:D^-(\mathcal{A})\to D^-(\mathcal{B})$.
\end{itemize}
Given an additive category $\mathcal{I} \subset \mathcal{A}$ of injective
objects in $\mathcal{A}$, we have  $\mathcal{H}^* (\mathcal{I})\subset 
\mathcal{H}^* (\mathcal{A}) \xrightarrow{G_{\mathcal{A}}}D^*(\mathcal{A})$.

\begin{lemma}
\label{lemma-embedding-of-injective-subcategory}
Suppose that $\mathcal{A}$ has enough injective objects, 
$\forall A^\bullet \in \mathcal{A}$ can be embedded 
$A \hookrightarrow I \in\mathcal{I}$. Then 
$i:\mathcal{H}^+(\mathcal{I}) \to D^+(\mathcal{A})$ is an equivalence.
\end{lemma}

\begin{proof}
Not proved.
\end{proof}

\section{Semiorthogonal decompositions}
\label{section-semiorthogonal-decompositions}

{\bf Idea.} To split $D^*(\mathcal{A})$ in smaller pieces that are easier.

\begin{definition}
\label{definition-strictly-full-triangulated-subcategory}
$\mathcal{A} \subset \mathcal{T}$ is a {\it strictly full triangulated
subcategory} if
\begin{itemize}
\item $\text{Hom}_{\mathcal{A}}(A,B)=\text{Hom}_{\mathcal{T}}(A,B)$ for all $A,B
\in \mathcal{A}$.
\item If  $B \in \mathcal{T}$ is isomorphic to an object from $\mathcal{A}$,
then $B \in \mathcal{A}$.
\end{itemize}
\end{definition}

\begin{definition}
\label{definition-semiorthogonal-decomposition}
Let $\mathcal{T}$ be a triangulated category and $\mathcal{A}$, $\mathcal{B}$
strictly full trinagulated subcategories in $\mathcal{T}$. Then
$\mathcal{T}=\left<\mathcal{A},\mathcal{B}\right>$ is a {\it semiorthogonal
decomposition} if
\begin{itemize}
\item $\text{Hom}_{\mathcal{T}}(B,A)=0$ for all $B \in \mathcal{B}$ and $A\in
\mathcal{A}$.
\item For all $T \in \mathcal{T}$ there exists a distinguished triangle
$$
T_B \to T \to T_{\mathcal{A}}\to T_{\mathcal{B}}[1]
$$
where $T_B \in \mathcal{B}$ and $T_\mathcal{A} \in \mathcal{A}$.
\end{itemize}
\end{definition}

\begin{proposition}[Properties of semiorthogonal decomposition]
\label{proposition-properties-of-semiorthogonal-decomposition}
\begin{enumerate}
\item Functoriality: for any $f \in T \to T'$,
$$
\xymatrix{
T_{\mathcal{B}}\ar@{.>}[d]_{\exists f_{\mathcal{B}}}\ar[r]& T\ar[d]^f\ar[r]
& T_{\mathcal{A}}\ar@{.>}[d]^{f_{\mathcal{A}}}\ar[r]& T_{\mathcal{B}}[1]\\
T_{\mathcal{B}}'\ar[r]& T'\ar[r]& T'_{\mathcal{A}}\ar[r]& T'_{\mathcal{B}}[1]
}
$$
there exist $f_{\mathcal{B}}$ and $f_{\mathcal{A}}$ as in the diagram.
\end{enumerate}
\end{proposition}

\begin{proof}
For $f_{\mathcal{A}}$,apply $\text{Hom}_{\mathcal{T}}(\cdot,T_{\mathcal{A}}')$ 
to the upper distinguished triangle, obtain a long exact sequence…
\end{proof}

Functoriality: defnote by $\alpha:\mathcal{A} \hookrightarrow  \mathcal{T}$ the
inclusion, then we have a functor $\alpha^*:\mathcal{T} \to \mathcal{A}$ given
by $T \to T_{\mathcal{A}}$, $f \mapsto  f_{\mathcal{A}}$. Similarly for
$\beta:\mathcal{B} \hookrightarrow  \mathcal{T}$ we have a functor $\beta
^*:\mathcal{T} \to \mathcal{B}$ by $T \to T_{\mathcal{B}}$, $f \mapsto
f_{\mathcal{B}}$.

\begin{lemma}
\label{lemma-alpha-and-beta-adjoint}
$\alpha^*$ is left adjoint to $\alpha$ and $\beta^*$ is right adjoint to $\beta.$
\end{lemma}

\begin{proof}
We have for any $A \in \mathcal{A}$,
$\text{Hom}_{\mathcal{A}}(\alpha^*T,A)=\text{Hom}_{\mathcal{T}}(T,\alpha_*A)$.
But that is equal to $\text{Hom}_{\mathcal{A}}(\mathcal{T}_{\mathcal{A}},A)$.
\end{proof}

\begin{exercise}
\label{exercise-orthogonal}
If $\mathcal{T}=\left<\mathcal{A},\mathcal{B}\right>$, then
$$
\mathcal{B}=\mathcal{A}^\perp=\left<B \in \mathcal{T} |
\text{Hom}(B,A), \forall A \in \mathcal{A}\right>:=\Ker \alpha^*
$$
(the $\perp$ sign should be on the left of $\mathcal{A}$ but I don't know how to
write this) and 
$$
\mathcal{A}=\mathcal{B}^\perp=\left<A \in \mathcal{T} |
\text{Hom}(B,A)=0,\forall B \in \mathcal{B}\right>:=\Ker B
$$
(here the $\perp$ is correct on the right side of $\mathcal{B}$).
\end{exercise}

This construction can be reversed.

\begin{definition}
\label{definition-left-and-right-admissible-subcategories}
A strictly full subcategory $\mathcal{A} \hookrightarrow  \mathcal{T}$ is called
\begin{itemize}
\item {\it left admissible} is there exists $\alpha^*:\mathcal{T} \to
\mathcal{A}$ left adjoint,
\item {\it right admissible} is there exists $\alpha^!:\mathcal{T} \to
\mathcal{A}$ right adjoint.
\end{itemize}
\end{definition}

\begin{lemma}
\label{lemma-admissible-implies-semiorthogonal-decomposition}
If $\mathcal{A} \hookrightarrow  \mathcal{T}$ is left admissible, then
$\mathcal{T}=\left<\mathcal{A},\mathcal{A}^\perp\right>$ is semiorthogonal
decomposition. If
$\mathcal{B} \hookrightarrow  \mathcal{T}$ is right admissible then
$\mathcal{T}=\left<\mathcal{B}^\perp, \mathcal{B}\right>$ is semiorthogonal
decomposition.
\end{lemma}

\begin{proof}
Uses a remark, Yoneda lemma.
\end{proof}

\begin{proposition}[Bondall, Van der Borgh]
\label{proposition-Bondall-Van-der-Borgh}
If $\mathcal{T}$ is smooth and proper (if $\mathcal{T}=D^b(X)$, where $X$ is
smooth and proper, we call $\mathcal{T}$ smooth and proper) then
$\mathcal{A} \subset \mathcal{T}$ is left admissible $\iff$ $\mathcal{A}$ is
right admissible $\iff$ $\mathcal{A}$ is smooth proper. If $\mathcal{A}$ is
smooth and proper and $\mathcal{A} \hookrightarrow  \mathcal{T}$ then
$\mathcal{A}$ is left and right admissible.
\end{proposition}

\medskip\noindent
If $D^b(C) \hookrightarrow  \mathcal{T}$ is fully faithful, then it is left and
right admissible.

{\bf Question.} When can we construct a full faithful embedding
 $$
D^b(\text{pt})\cong D^b(\text{mod}k) \hookrightarrow  \mathcal{T}?
$$
Let \begin{align*}
D^b(\text{mod}k) &\longrightarrow \mathcal{T} \\
k &\longmapsto E
\end{align*}
and define
$$
\varphi_E:V^\bullet \to V^\bullet \otimes E
$$
where $V^\bullet$ is a graded vector space.

When is $\varphi_E$ full faithful? Some computations show that $\varphi_E$ is
fully faithful if and only if $\text{Hom}(E,E)=k$. Recall from 
\texttt{categories.tex} that

\begin{definition}
\label{definition-faithful}
Let $F : \mathcal{A} \to \mathcal{B}$ be a functor.
\begin{enumerate}
\item We say $F$ is {\it faithful} if
for any objects $x, y \in \Ob(\mathcal{A})$ the map
$$
F : \Mor_\mathcal{A}(x, y) \to \Mor_\mathcal{B}(F(x), F(y))
$$
is injective.
\item If these maps are all bijective then $F$ is called
{\it fully faithful}.
\end{enumerate}
\end{definition}

\medskip\noindent
\begin{definition}
\label{definition-exceptional-object}
$E \in \mathcal{T}$ is exceptional if 
$$
\text{Hom}(E,E[i])=
\begin{cases}
k\qquad &i=0 \\
0\qquad &i \neq 0
\end{cases}
$$
\end{definition}

\begin{lemma}
\label{lemma-exeptional-implies-fully-faithful}
If $E$ is exceptional then $\varphi_E:D(\text{pt}) \to \mathcal{T}$ is a fully
faithful embedding. Then
$$
\mathcal{T}=\left<\varphi_E(D^b(\text{pt})^\perp,E\right>=
\left<E^\perp,\varphi_E(D^b(\text{pt}))\right>
$$
\end{lemma}

{\bf Notation.} In this case,
$\mathcal{T}=\left<E,E^\perp\right>=\left<E^\perp,E\right>$, where on the first
bracket the $\perp$ should be on the left side of $E$.

\begin{example}
\label{example-all-line-bundles-are-exceptional-on-Fano}
On  $\mathbb{P}^n$, $\mathcal{O}(i)$ is exceptional. More general, if $X$ is
Fano (i.e. smooth projective with $-K_X$ ample), then any line bundle on  $X$ is
ample.
\end{example}

\begin{proof}
$\text{Ext}^i(\mathcal{L},\mathcal{L})=H^{i}(X,\mathcal{O}_X)$,
$H^{0}(X,\mathcal{O})=k$ and $H^{>0}(X,\mathcal{O})=0$ by Kodaira
Vanishing.
\end{proof}

For Fano varieties we have nontrivial semiorthogonal decomposition.

\begin{remark}
\label{remark-CY-have-no-semiorthogonal-decomposition}
If $K_X \cong \mathcal{O}_X$ then there is no semiorthogonal decomposition of
$D^b(X)$ (for K3 or abelian).
\end{remark}

Back to Fano,
\begin{definition}
\label{definition-semiorthogonal-decomposition-for-several-categories}
Let $\mathcal{A}_1,\ldots,\mathcal{A}_n \in \mathcal{T}$ be strictly full
subcategories and $\mathcal{T}=\left<\mathcal{A}_1,\ldots,\mathcal{A}_n\right>$ 
a {\it semiorthogonal decomposition}
\begin{itemize}
\item $\text{Hom}(\mathcal{A}_j,\mathcal{A}_i)=0$ for all $j>i$.
 \item For all $T$ there is a chain of morphisms
 $$
0 = T_n \to T_{n-1}\to \ldots \to T_1 \to T_0=T
$$
such that $\text{Cone}(T_i \to T_{i-1}) \in \mathcal{A}_i$.
\end{itemize}
\end{definition}

\begin{exercise}
\label{exercise-functoriality-and-cone}
Prove functoriality of $T \to \text{Cone}(T_i \to T_j)$ for $i>j$.
\end{exercise}

\begin{definition}
\label{definition-exceptional-collection-of-objects}
A collection of objects $E_1,\ldots, E_n$ is {\it exceptional} if
 \begin{enumerate}
\item For all $i$, $E_i$ is exceptional.
\item $\text{Hom}(\left<E_i\right>,\left<E_j\right>)=0$ for all $i>j$, where
$\left<E_j\right>$ is a minimal triangulated category generated by $E_i$.
\end{enumerate}
\end{definition}

We have a corollary:

\begin{lemma}
\label{lemma-exceptional-collection-gives-semiorthogonal-decomposition}
If $E_1,\ldots,E_n$ is an exceptional collection, then
$$
\mathcal{T}=\left<E_1,\ldots,E_n,\left<E_1,\ldots,E_n\right>^\perp\right>
=\left<\left<E_1,\ldots,E_n\right>^\perp,E_1,\ldots,E_n\right>
$$
(where $\perp$ should be on the left of the bracket in the first appearence) 
is a semiorthogonal decomposition.
\end{lemma}

\begin{definition}
\label{definition-full-exceptional-collection}
If $\left<E_1,\ldots,E_n\right>^\perp=0$ (equivalently with $\perp$ on the
left).
\end{definition}

\begin{exercise}
\label{exercise-exceptional-collections-for-Fano}
Let $X$ be a Fano variety with $-K_X=mH$ for $H$ ample ($m$ is the index of
$X$). Then for every line bundle $\mathcal{L}$,
$$
(\mathcal{L}(-m+1)H,\ldots,\mathcal{L}(-H)\mathcal{L})
$$
is an exceptional collection.
\end{exercise}

\begin{proof}
Use Serre duality and Kodaira Vanishing.
\end{proof}

\medskip\noindent
Here is the promised theorem from the beginning of this minicourse:

\begin{theorem}[Beilinson]
\label{theorem-Beilinson}
$D^b(\mathbb{P}^n)=\left<\mathcal{O}(-n),\ldots,\mathcal{O}\right>$ is a full
exceptional collection.
\end{theorem}

This means that we can classify all objects of $D^b(\mathbb{P}^n)$. Because 
whenever we have a semiorthogonal decomposition we get 
these chains of morphisms, i.e. for all $D^b(\mathbb{P}^n)$ there exists a chain 
$$
0=T_n \to \ldots \to T_0=T
$$
such that $\text{Cone}(T_i \to T_{i-1}) \in \left<\mathcal{O}(-i)\right> \cong
D^b(\text{pt})$.

Probably this is the content of \cite{Beilinson-derived}.

\begin{proof}[Proof of Beilinson theorem]
Consider the Koszul complex
$$
0\to \mathcal{O}(-n-1)\to\ldots \to \Lambda^{2}V^\vee \otimes \mathcal{O}(-2)
\to V^{\vee}\otimes \mathcal{O}(-1) to \mathcal{O} \to0
$$
This implies that
$$
\mathcal{O}(n-1) \in \left< \mathcal{O}(-n),\ldots,\mathcal{O}\right>
$$
… proof completed in class.
\end{proof}
\bibliography{my}
\bibliographystyle{amsalpha}

\end{document}
