\input{preamble}
\begin{document}

\title{Geometric Invariant Theory}
\maketitle

Minicourse by Michel Brion, CIMPA school Florianópolis 2025.

Notes at 
\href{http://github.com/danimalabares/cimpa-floripa}
{github.com/danimalabares/cimpa-floripa}

\bigskip\noindent

{\bf Abstract.} Geometric Invariant Theory (GIT) studies the actions of
reductive algebraic groups on algebraic varieties and the constructions of
quotients in algebraic geometry. One of the most relevant aspect of GIT is its
application to moduli theory: we can indeed approach moduli problems trying to
construct moduli spaces as GIT quotients. In this course we will present the
basic notions of geometric invariant theory, we will illustrate its connections
to moduli and provide concrete examples of moduli spaces obtained as GIT
quotients. 

\bigskip\noindent
\tableofcontents
\bigskip\noindent

{\bf Plan.}

\begin{enumerate}
\item Quotients by finite groups.
\item Linear algebraic groups.
\item GIT (affine case).
\item GIT (general case).
\end{enumerate}

\begin{proposition}
\label{proposition-1}
$\mathcal{O}(X)^G$ is finitely generated and $\mathcal{O}(X)$ is a finite module
over $\mathcal{O}(X)^G$
\end{proposition}

Let $Y = \text{Spec}B$ and $\pi:X\to Y$ a dominant, finite morphism, which
implies it is surjective.

\begin{proposition}
\label{proposition-2}
The set-theoretic fibers of $\pi$ are the orbits $Gx$
\end{proposition}

\begin{proof}
$\pi$ is $G$-invariant, i.e. constant on orbits.
\end{proof}

\begin{proposition}
\label{proposition-2-second}
The quotient map $\pi$ is open and if $U\subset X$ is open, affine, $G$-stable, 
then the restriction $\pi|_{U}$ is the quotient map $U\to U/G$.
\end{proposition}

\begin{proposition}
\label{proposition-quotient-map-is-categorical-quotient}
$f:X\to Z$ $G$-invariant morphism, then there is a unique morphism $\varphi:Y
\to Z$ such that $f=\varphi \circ\pi$. This says that $\pi$ is the categorical
quotient.
\end{proposition}

\begin{proof}[Sketch of proof]
It is clear that $\varphi$ exists as a map of sets. It is continuous since
$\pi$ is open surjective. $\varphi^\sharp:\mathcal{O}_Z \to
\varphi_*(\mathcal{O}_Y)$.
\end{proof}

\begin{example}
\label{example-quotients}
 \begin{enumerate}
\item $G=\mu_n$ the roots of unity of order $n$ (prime to $p$). $G$ acts on
$\mathbb{A}^2$ by $g(x,y)=(gx,gy)$.  $\mathcal{O}(\mathbb{A}^2)=k[x,y]$ and
$\mathcal{O}(\mathbb{A}^2)=k[x^n,x^{n-1}y,\ldots,xy^{n-1},y^n]$.
$\mathbb{A}^2/G$ is a singular surface.
\item $G=S_n$ symmetric group, acts on $\mathbb{A}^n$ permuting the coordinates.
$\mathcal{O}(\mathbb{A}^n)=k[x_1,\ldots,x_n]$,
$\mathcal{O}(\mathbb{A}^2)^G=k[e_1,\ldots,e_n]$ where $e_i$ is the $i$-th
symmetric function. The quotient
$\pi:\mathbb{A}^n\xrightarrow{(e_1,\ldots,e_n)}\mathbb{A}^n$ is finite, flat of
degree $n!$.
\item (Symmetric products.) $X=Z^n$, $Z$ affine algebraic variety, $G=S_n$
acting by permuting copies of $Z$. Then $X/G:=Z^{(n)}$, the $n$-th symmetric
product of $Z$. Points of $Z^{(n)}=$effective $0$-cycles in $Z$ of degree $n$ =
$z_1+\ldots+z_n$, $(z_i \in Z)$ = $n_1z_1+\ldots+n_kz_k$ where $z_i$ are
distinct and $n_i$ are positive integers with some $n$.
\end{enumerate}
\end{example}

\subsection*{Local structure of quotients}
\label{subsection-local-structure-of-quotients}

See Mumford, Abelian Varieties.

\begin{definition}
\label{definition-free-locus}
Points with trivial stabilizer.
\end{definition}

The local structure of  the symmetric product variety $Z^{(n)}$ 
at the point $x$ is the stabilizer
$$
G_x=S_{n_1}\times\ldots\times S_{n_2}\subset S_n=G
$$
and we get
$$
Z^{(n_1)}\times\ldots\times Z^{n_r} = X/G_x \to Z^{(n)}=X/G
$$
which is étale at the image of $x$.

\begin{proposition}
\label{proposition-variety-smooth-then-symmetric-product-smooth}
If $Z$ is smooth then $Z^{(n)}$ is smooth.
\end{proposition}

\begin{proof}
Using power series rings.
\end{proof}

\begin{proposition}
\label{proposition-simension-geq2-then-symmetric-product-singular}
If $\dim Z \geq 2$ then $Z^{(n)}$ is singular.
\end{proposition}

\begin{proof}
Done in lecture.
\end{proof}

\medskip\noindent
Then we have the {\it Hilbert scheme characteristic morphism}
\begin{align*}
\gamma: \text{Hilb}_n(Z) &\longrightarrow Z^{(n)} \\
W\subset Z &\longmapsto \sum_{z \in Z}\dim (\mathcal{O}_{W,z}z
\end{align*}
where $W$ must be {\bf finite of length }$n$. See Binger, Linear determinants
(older) or J. Bertin, The punctual Hilbert Scheme.

If $Z$ is smooth of dimension 1 then $\gamma$ is an isomorphism. If $Z$ is of
dimension 2 then $\gamma$ is a resolution. In general $\gamma$ is an isomorphism
above $Z_{f_2}^{(n)}$.

\subsection*{Linear algebraic groups}
\label{subsection-linear-algebraic-groups}
\begin{definition}
\label{definition-algebraic-group}
An {\it algebraic group} is a variety $G$ equipped with morphisms
\begin{align*}
m: G\times G &\longrightarrow G \\
(g,h) &\longmapsto gh
\end{align*}
and
\begin{align*}
i: G &\longrightarrow G \\
g &\longmapsto g^{-1}
\end{align*}
with an element $e \in G$ which satisfies the group axioms.
\end{definition}

(In fact this is the same as a group scheme of finite type over $k$.)

\begin{example}
\label{example-algebraic-groups}
\begin{enumerate}
\item Finite groups.
\item $\mathbb{G}_a=$additive group = $(\mathbb{A}^1,+)$ and $\mathbb{G}_m$
=multiplicative group = $(\mathbb{A}^1\setminus\{0\},\times)$.
\item $\text{GL}_n$ = general linear group. This is a principal open subset
($\det \neq 0$), an affine variety.
\item Classical groups $\text{SL}_n,\text{O}_n,\text{Sp}_{2n}$ closed in 
$\text{GL}_n$. Also 
$\text{PGL}_n=\text{GL}_n/\mathbb{G}_n:I_n\hookrightarrow\text{GL}_{n^2}$.
\item $(E,0)$ elliptic curve; there is a unique algebraic group law + with
neutral element $0$. (Abelian varieties.)
\end{enumerate}
\end{example}

\begin{proposition}
\label{proposition-connected-components-of-algebraic-group}
$G$ algebraic group, $G^0$ connected component of $e$: $G^0$ is a closed normal
subgroup of $G$ and the connected components of $G$ are the $gG^0$ for $g \in
G$. Moreover, $G^0$ is irreducible and $G$ is smooth (which is obvious because
it is smooth somewhere and we may just translate).
\end{proposition}
$$
\xymatrix{
1\ar[r]&G^0\ar[r]&G\ar[r]&\pi_0(G)\ar[r]&1
}
$$
\begin{definition}
\label{definition-action-of-algebraic-group}
An {\it action} of an algebraic group $G$ on a variety $X$ is a morphism $a:G
\times X \to X$ which satisfies the axioms of an action:
$$
a(g,x)=g\cdot x,\qquad  g\cdot(h\cdot x)=gh\cdot x,\qquad  e\cdot x=x
$$
The {\it orbit} of $x \in X$ is $G\cdot x=\{g\cdot x|g \in G\}$. The {\it
stabilizer} $G_x$ is $\{g \in G|g\cdot x=x\}$, which is a closed subgroup of
$G$.
\end{definition}

\begin{proposition}
\label{proposition-properties-of-orbit}
\begin{enumerate}
\item Each orgbit $G\cdot x$ is locally closed, smooth of dimension
$\dim(G)-\dim(G_x)$.
\item The orbit map $G \to G\cdot x$, $g \mapsto  g\cdot x$ is faithfully flat.
\item $\overline{G\cdot x}\setminus G\cdot x$ is a union of orbits of smaller
dimension.
\item Closed orbits exist.
\end{enumerate}
\end{proposition}

\begin{proof}
Should be easy.
For example, to see it's faithfully flat you just notice it is flat somewhere
(if it is connected, is it connected?) and then use homogeneity again.
\end{proof}

\begin{proposition}
\label{proposition-bound-on-dimension-of-orbit-gives-open-set}
For any $n \geq 1$, the set $\{x \in X|\dim(G\cdot x)\geq n\}$ is open. That is,
$\{x \in X|\dim(G_x)\leq n\}$ is open.

In particular, the points with finite stabilizer form an open subset.
\end{proposition}

\begin{proof}
Use
\begin{align*}
\gamma: G \times X &\longrightarrow X\times X \\
(g,x) &\longmapsto (g\cdot x,x)
\end{align*}
\end{proof}

\begin{lemma}
\label{lemma-homomorphism-of-algebraic-groups}
Let $f:G \to H$ be an homomorphism of algebraic groups. Then $\Ker (f)$ is a
normal subgroup of $G$. $\text{Im}(f)$ is a
closed subgroup of $H$ of dimension $\dim(G)-\dim \Ker f$.
\end{lemma}

\begin{example}
\label{example-SL2-acts-on-homogeneous-polynomials}
$G=\text{SL}_2$ acts on $k[x,y]_n:=V_n$ (these are called the {\it binary forms 
of degree $n$}) by linear change of variables.
\end{example}

\begin{definition}
\label{definition-linear-algebraic-group}
An algebraic group is {\it linear} if $G \hookrightarrow \text{GL}_n$ for some
$n$.
\end{definition}

\begin{example}
\label{example-multiplicative-group-is-linear}
$\mathbb{G}_a=\left\{\begin{pmatrix}
1&t\\ 
0&q
\end{pmatrix}\right\}$ is linear.
\end{example}

\begin{proposition}
\label{proposition-co-action}
$G$ linear algebraic group,  $X$ a $G$-variety, $G$ acts on
$\mathcal{O}(X)=\Gamma(X,\mathcal{O}_X)$ via $(g\cdot f)(x)=f(g^{-1}\cdot x)$
then $\mathcal{O}(X)$ is a union of f.d. $G$-stable subspaces on which $G$ acts
algebraically.
\end{proposition}

\begin{proof}
From the action $a:G \times X \to X$ we get the {\it co-action} 
\begin{align*}
a^\sharp: \mathcal{O}(X) &\longrightarrow \mathcal{O}(G \times X) \\
f &\longmapsto \sum_{i=1}^n(\varphi_i \otimes \psi_i)
\end{align*}
\end{proof}

\begin{definition}
\label{definition-G-module}
A finite dimensional vector space $V$ is called a {\it $G$-module} if $V$ is equipped
with an action of $G$ via a homomorphism of algebraic groups 
$G \to \text{GL}(V)$.

A  {\it $G$-module} is a vector space $V$ equipped with a linear action og the
group $G$ such that $V=$ union of finite dimensional $G$-submodules.
\end{definition}

\begin{proposition}
\label{proposition-G-modules}
 \begin{enumerate}
\item $X$ affine $G$-variety, $G$ linear algebraic group, then $X
\hookrightarrow  V$ finite dimensional $G$-module as a closed $G$-stable
subvariety.
\item Every affine algebraic group is linear.
\end{enumerate}
\end{proposition}

Let $X$ be an affine $G$-variety. Then
\begin{align*}
\mathcal{O}(X) &\longrightarrow \mathcal{O}(G\times X)
=\mathcal{O}(X)[t,t^{-1}] \\
f &\longmapsto \sum_{n \in \mathbb{Z}}f_n t^n
\end{align*}

In particular, every $\mathbb{G}_m$-module is semisimple: every nonzero
$G$-module contains a nonzero fixed point. Indeed, let $V$ be a
finite-dimensional $G$-module. Let $v \in V$ be a $\mathbb{G}_a$-eigenvector.
Then $g\cdot v=\gamma(g)v$ for ? $X \in
\mathcal{O}(\mathbb{G}_a)^{\times}=k^{\times}$ taking $g=0$ yields $X=1$.

$G$ connected algebraic group, then $G$ is {\it unipotent} if it is an iterated
extension of copies of $\mathbb{G}_a$. Then every nonzero $G$-module contains a
nonzero fixed point.

As an example consider $G=\text{SL}_2$. (We saw that it is not linearly
reductive in the sense described in the next section.)

\subsection*{Quotients by linearly reductive groups}
\label{subsection-quotients-by-linearly-reductive-groups}

\begin{definition}
\label{definition-linearly-reductive-group}
A linear algebraic group $G$ is {\it linearly reductive} if every $G$-module is
semi-simple.
\end{definition}

\begin{exercise}
\label{exercise-linearly-reductive-characterization}
$G$ is linearly reductive if and only if the functor 
$(\text{$G$-modules})\to(\text{vector spaces})$, $V \mapsto V^G$ is exact.
\end{exercise}

\begin{theorem}
\label{theorem-linearly-reductive-normal-unipotent-connected-subgroups-are-
trivial}
 If $G$ is linearly reductive then every connected normal unipotent subgroup $U$ 
of $G$ is trivial. The converse is true in characteristic zero.
\end{theorem}

\begin{proof}
($\implies$) Let $V$ be a $G$-module, $V\neq 0$ and $U$ as in the statement.
Then $V^U\neq 0$ is a $G$-module. Let $W$ be a $G$-module. 
Then $V=V^U \oplus W$. [Missing]

($\impliedby$) We show this for $k= \mathbb{C}$. View $G \supset K$ compact
subgroup, Zariski dense (eg. $C^* \supset S^1$ or 
$\text{GL}_n(\mathbb{C})\supset \text{U}_n$.) [Missing]
\end{proof}

\begin{definition}
\label{definition-reductive-linear-algebraic-group}
A linear algebraic group $G$ is {\it reductive} if it has a nontrivial connected
unipotent normal subgroup.
\end{definition}

The following definition is from StacksProject, \texttt{brauer.tex}:

\begin{definition}
\label{definition-simple}
Let $A$ be a $k$-algebra.
We say an $A$-module $M$ is {\it simple} if it is nonzero and
the only $A$-submodules are $0$ and $M$.
We say $A$ is {\it simple} if the only two-sided ideals of $A$ are
$0$ and $A$.
\end{definition}

\begin{lemma}
\label{lemma-Reynolds-operator}
$G$ linearly reductive, $V$ a $G$-module. Then
\begin{enumerate}
\item There is a unique projection of $G$-modules $R_V:V \to V^G$ called the
{\it Reynolds operator}.
\item 
$$
\xymatrix{
V\ar[d]_{R_V}\ar[r]^f&  W\ar[d]_{R_V}\\
V^G\ar[r]^{f^G}&W^G
}
$$
commutes.
\item $A$ a $G$-algebra, $B:=A^G$. Then $R_A(ab)=bR_A(a)$ for all $a \in A$, $b
\in B$.
\end{enumerate}
\end{lemma}

\begin{proof}
\begin{enumerate}
\item There is a unique decomposition $V=V^G \oplus V'$ where $V'$ is the sum of
nontrivial simple $G$-submodules.
\item Implied by the decomposition $V=V^G \oplus V'$.
\item 
$$
\xymatrix{
V\ar[d]_{R_A}\ar[r]^b&  W\ar[d]_{R_A}\\
V^G\ar[r]^{b}&W^G
}
$$
commutes.
\end{enumerate}
\end{proof}

\begin{theorem}[Hilbert,Nagata]
\label{theorem-linearly-reductive-action-
implies-invariant-sections-finitely-generated}
If $G$ is a linearly reductive group acting on an affine variety $X$, then
$\mathcal{O}(X)^G$ is finitely generated.
\end{theorem}

\begin{proof}
Done in class, uses $R_A$ and graded Nakayama lemma.
\end{proof}

\medskip\noindent	
$\mathcal{O}(X)^G \subset \mathcal{O}(X)$ yields $\pi:X\to Y$, a $G$-invariant
morphism of affine varieties, giving a good quotient.

\begin{proposition}
\label{proposition-projection}
\begin{enumerate}
\item $\pi$ is regular.
\item If  $Z \subset X$ is closed and $G$-stable, then $\pi|_{Z}$ if the good
quotient of $Z$.
\item If $Z,Z'\subset X$ are closed $G$-stable and disjoint, then $\pi(Z)$,
$\pi_i(Z')$ are disjoint.
\item Every fiber of $\pi$ contains a unique closed $\mathbb{G}$-orbit.
\end{enumerate}
But if $U\subset X$ is open affine $G$-stable, then $\pi|_{U}$ is not the good
quotient.
\end{proposition}

\begin{example}
\label{example-Hopf-surface?}
$G=\mathbb{G}_n$ acts on $\mathbb{A}^n$ by scalar, i.e. 
$g(x_1,\ldots,x_n)=(gx_1,\ldots,gx_n)$. Then $\mathcal{O}(\mathbb{A}^n)^G=k$,
$\pi:\mathbb{A}^n \to \bullet$,
but $f_nU=D(f)$, where $f$ is a nonzero linear form,
$$
\mathcal{O}(U)^{\mathbb{G}_m}=\bigcup_{d \geq 0}\frac{k[x_1,\ldots,x_n]}{f^d}
$$
\end{example}

{\bf Notation:} $Y=X//G$, the space of closed orbits.

\begin{definition}
\label{definition-stable-point}
A point $x \in X$ is called {\it stable} if its orbit $G\cdot x \subset X$ is
closed and the stabilizer  $G_x$ is finite.
\end{definition}

This definition is different from that in Mumford's book on GIT (our definition
there is called {\it properly} stable point.)

\begin{proposition}
\label{proposition-stable-points}
The set of stable points $X^s$ is open, $G$-stable and $\pi(X^s)=Y^s$ is open.
Moreover, $X^s=\pi^{-1}(Y^s)$ and $X^s \to Y^s$ is a geometric quotient.
\end{proposition}

Warning: there might not be any stable points at all! (But it looks like there
are some ways to get around this.)

\begin{proof}
The first trick very elementary; just playing with orbits. 
Let $x_0 \in X$ be stable. Then
$$
G\cdot x_0 \subset \pi^{-1}\pi(x_0)=\{ x \in X 
: \overline{G\cdot x}\supset G\cdot x\}=G\cdot x_0
$$
since $\dim (G\cdot x_0)=\dim G$.

Then, roughly, remove higher dimensional orbits, and since the remaining orbits
have all the same dimension, then they are all closed.

The final assertion $X^s \to Y^s$ is similar to what we did. No deep arguments.
\end{proof}

\begin{example}
\label{example-action-of-SL2-on-forms}
$G=\text{SL}_n$ acts on $X=k[x_1,\ldots,x_n]_d$ by linear change of variables.
Let's consider some values. If $d=1$, then the quotient is  $X//G=\text{pt}$.
For $d=2$, the quadractic forms, we have the discriminant $\Delta$ as an 
invariant, and in fact $\mathcal{O}(\Delta)=k[\Delta]$ {\bf (A nice exercise!)}. 
Actually it looks like $\pi=\Delta:X \to \mathbb{A}^1$, and the stabilizer of 
$f$ is $\text{SO}(f)$, no stable points.

For $d \geq 3$, and $n \geq 5$, then $\mathcal{O}(X)^G$ is not known! But we
also have the discriminant $\Delta \in \mathcal{O}(X)^G$ and in fact (Jordan) if
$\Delta(f) \neq 0$ then $f$ is stable. Let us prove this fact (we don't know how
did Jordan proved this): we show that $G_f$ is finit and stable, i.e.
$\text{Lie}(G_f)=0$. The Lie algebra of $\text{SL}_2$, the traceless matrices,
acts on $X$ by
$$
A\cdot f= \sum_{i=1}^n \ell_i \frac{\partial f}{\partial x_i}
$$
where $\ell_i = \sum_{j=1}^n a_{ij}x_j$. Then $A\cdot f=0$. Now 
$\Delta(f) \neq 0$ iff $\frac{\partial f}{\partial x_j}$ have no nontrivial
common zero, iff, $\frac{\partial f}{\partial x_j}$ form a regular sequence.
This says that the ring Cohen-Macaulay. Then
$$
\ell_i \frac{\partial f}{\partial x_i}
=\sum_{i \neq  j}\ell_j \frac{\partial f}{\partial x_j}
$$
which in turn implies that
$$
\ell_i \in \left(\frac{\partial f}{\partial x_j}, \; j \neq  i\right)
\iff \ell_i=0 \forall  i
$$
… which says {\it what?}
\end{example}

\medskip\noindent
Let $X$ be a projective $G$-variety, $G$ linearly reductive, $L \to X$ an ample
line bundle
$$
R(X,L)=\bigoplus_{n=0}^{\infty} \Gamma(X,L^{\otimes n})
$$
which is a graded, finitely generated algebra, then $X=\text{Proj}R(X,L)$. If
$G$ acts on $R(X,L)$ ``compatibly'', then $R(X,L)^G$ is finitely generated:
$$
\text{Proj}R(X,L)^G=X//G
$$
and we get a birational map
$$
\xymatrix{
X\ar[r]^{bir}& X//G\\
U\ar[u]_{hook}\ar[ur]^{\pi}
}
$$
where $U$ is the domain of definition.

If $G$ is finite, then  $X= \mathbb{P}(V)$, $V$ finite dimensional $G$-module. 
$L = \mathcal{O}_X(1)$: $G$ acts on $L$ and hence on $R(X,L)$ (``{\it I'll explain
details soon}'')

We have a finite ring extension $R(X,L)^G = R(X,L)$, and we get
$$
X=\text{Proj}R(X,L) \to \text{Proj}R(X,L)^G=Y
$$

\medskip\noindent
Now let $G$ be an algebraic group, $X$ a $G$-variety, $f:L \to X$ a line
bundle.

\begin{definition}
\label{definition-G-linearization}
A  {\it $G$-liearization} of $L$ is a $G$-action on $L$ such that $f$ is
equivariant and $g:L_x \to L_{g\cdot x}$ is linear for all $g \in G, x \in X$.
That is, the $G$-action on $L$ commutes with the $\mathbb{G}_m$-action by
multiplication on fibers.
\end{definition}

Then we get an action on the section ring: $G$ acts on 
$L^{\times}=L\setminus \{\text{zero section}\}$ and hence
$\mathcal{O}(L^{\times})$ is a $G$-module. As we said before,
 $\mathcal{O}(L^{\times})=\bigoplus_{n \in \mathbb{Z}}\Gamma(X,L^{\otimes n})$
is a graded module, and the grading corresponds to the $G$-action! That is, 
this is a $G$-module structure on $\Gamma(X,L^{\otimes n})$. And $R(X,L)$ is a
graded $G$-algebra.

\begin{definition}
\label{definition-semi-stable-line-bundle}
$X$ a $G$-variety, $L$ a $G$-linearized line bundle on $X$ and $x \in X$. We say
$x$ is {\it semi-stable} w.r.t. $L$ if there exists $n \geq 1$ and 
$\sigma \in \Gamma(X,L^{\otimes n})^G$ and that $x \in X_\sigma$ and 
$X_\sigma$ is affine.

$x$ is  {\it stable} if in addition $G\cdot x$ is closed in $X_\sigma$ 
and $G_x$ is finite.
\end{definition}

\begin{theorem}[Mumford]
\label{theorem-Mumford}
The set of semis-stable points $X^{ss}(L)\subset X$ is open, $G$-stable and has 
a good quotient $\pi:X^{ss}(L) \to Y(L)$. (Notice everything depends on the 
line bundle.) The set of stable poins $X^s(L)$ is open $G$-stable and equal to 
$\pi^{-1}\pi(X^s(L))$ and $\pi|_{X^s(L)}$ is a geometric quotient.
\end{theorem}

\begin{proposition}
\label{proposition-on-projectivization-of-V}
Let $x \in \mathbb{P}(V)$ and $x \in [v]$.
\begin{enumerate}
\item $x \in \mathbb{P}(V)^{ss}$ $\iff$ $0 \not \in \overline{G\cdot v}$.
\item $x\in \mathbb{P}(V)^s$ $\iff$ $v \in V^s$ i.e. $G\cdot v$ is closed in $V$
and $G_v$ is finite.
\end{enumerate}
\end{proposition}

\begin{proof}
Done in class.
\end{proof}

The following is our first example of moduli space.

\begin{example}[Smooth hypersurfaces of degree $d$ in $\mathbb{P}^n$]
\label{example-smooth-hypersurfaces}
The isomorphism classes of such surfaces are given by: take degree $d$
polynomials, projectivize, take away diagonal ( this gives an open set) and
finally quoetient by $\text{SL}_{n+1}$:
\begin{align*}
\{\text{isom. classes}\}&=\mathbb{P}(k[x_0,\ldots,x_n]_d)_{\Delta}
/\text{SL}_{n+1}\overset{\text{open}}{\hookrightarrow }\\
\mathbb{P}(\cdot)^1/\text{SL}_{n+1}& \hookrightarrow  
\mathbb{P}(\cdot)^n//\text{SL}_{n+1}
\end{align*}
a projective variety, normal, very singular.
\end{example}

\begin{remark}
\label{remark-categorial-quotient-yields-coarse-moduli-space}
$G \mathbb{y} X$ categorical quotient $\pi:X \to Y$, then $\pi$ yields a coarse
moduli space for
$$
\xymatrix{
X\ar[r]\ar[rd]& [X/G]\ar@{.>}[d]\\
&Y
}
$$
\end{remark}

\begin{proof}
$\text{Hom}(S,[X/G])$ is a category, not a set as usual. Its objects are
$$
\xymatrix{
P\ar[d]_{\substack{\text{principal} \\ \mathbb{G}\text{-bundle}}}
\ar[r]^{\substack{G\text{-equivariant} \\ \text{morphism }b}}&X\\
S
}
$$
and morphisms are
$$
\xymatrix{
&X\\
P\ar[ur]^{b'}\ar[rr]^\varphi \ar[dr]&  &  P\ar[ul]^b\ar[dl]\\
&S
}
$$
(continues…)
\end{proof}

\subsection{Hilbert-Mumford criterion}
\label{subsection-Hilbert-Mumford-criterion}

\begin{definition}
\label{definition-one-parameter-subgroup}
A {\it one-parameter subgroup} of  $G$ is a homomorphism
$\lambda:\mathbb{G}_m \to G$. If $G \mathbb{y} X$ and $x \in X$, we say that
$$
\lim_{t \to 0}\lambda(t) 
$$
exists and equals $y \in X$ if $\mathbb{A}^1\setminus\{0\}=\mathbb{G}_m \to X$,
$t \mapsto \lambda(t)\cdot x$ extends to $\mathbb{A}^1 \to X$, $0 \mapsto y$.
\end{definition}

\begin{theorem}[Hilbert-Mumford criterion]
\label{theorem-Hilbert-Mumford-criterion}
$V$ finite-dimensional $G$-module, $G$ linearly reductive, $v \in V$,
$\overline{G\cdot v}\supset G\cdot v$ closed orbit. There exists a one-paremeter
subgroup $\lambda$ such that $\lim_{t \to 0} \lambda(t)\cdot v \in G\cdot w$.
\end{theorem}

We will not prove this theorem. Here are some corollaries.

\begin{lemma}
\label{lemma-unstable}
$v \not \in V^{ss} \iff \exists  \lambda$ such that 
$\lim_{t\to_0}\lambda(t)\cdot v =0$. Then $v$ is called {\it unstable} and 
$$
\{\text{unstable points}\}=\text{{\it nilcone}} \mathcal{N}(V)=V.
$$
\end{lemma}

\begin{lemma}
\label{lemma-orbit-is-closed-iff-limit-lies-in-orbit}
$G\cdot v$ is closed $\iff$ $\lim_{t \to 0} \lambda(t)\cdot v$ exists, then it
lies in $G\cdot v$.
\end{lemma}

\begin{proposition}
\label{proposition-orbit-is-affine-iff-stabilizer-is-linearly-reductive}
$G\cdot v$ is affine $\iff$ $G_v$ is linearly reductive.
\end{proposition}

\begin{proof}
($\implies$) $G_v = H \iff G\cdot v G/H=G//H$.

($\impliedby$) We show that the functor
\begin{align*}
(\text{left $H$-modules}) &\longrightarrow (\text{f.d. vector spces}) \\
V &\longmapsto V^H
\end{align*}
(continues…)
\end{proof}

We can characterize stable points using this proposition.

\begin{lemma}
\label{lemma-stable-point-iff-orbit-is-closed-and-not-fixed-by-nontrivial-1PS}
$v \in V^s$ $\iff$ $G\cdot v$ is closed and $v$ is fixed by no nontrivial
one-parameter subgroup.
\end{lemma}

\begin{proof}
The reason is that if $G_v=H$ is linearly reductive then $H^0$ is generated by
images of a one-parameter subgroup.
\end{proof}

\begin{example}[Stable points for $SL_2$ acting on degree $d$ polynomials]
\label{example-stable-points-for-SL2-acting-on-degree-d-polynomials}
If $G=\text{SL}_2$ in characterstic zero, then
$$
\lambda(t)=\begin{pmatrix}
t&0\\ 
0&t^{-1}
\end{pmatrix}
$$
every nontrivial one-parameter subgroup of $G$ is conjugate to $\lambda(t)^n$
for a unique $n \geq 1$. $V=V_d=k[x,y]_d\ni x^e y^{d-e}$.
$$
\lambda(t) x^e y^{d-e}=t^{d-2e}x^ey^{d-e}
$$
$f \in V$ is unstable $\iff$ $\exists g \in G$ such that $gb \in \left<
x^{d/2-t}y^{d/2+1},\ldots, y^d\right>$ if dimension is even, and $
\left<x^{(d-1)/2}y^{(d+1)/2},\ldots, y^d\right>$ if dimension is odd, $\iff$
$f$ has no root of multiplicity $d/2$. $f \not \in V^s$ $\iff$ $f$ has no root
of multiplicity $\geq d/2$.

If $d$ is odd then $V^s=V^{ss}$. If $d$ is even then $x^{d/2}y^{d/2}$ has a
closed orbit and is fixed by $\\text{Im}(\lambda)$.
\end{example}

\subsection*{Hyperelliptic curves}
\label{subsection-hyperelliptic-curves}

Let $C \to \mathbb{P}^1$ be a smooth projective curve of degree $2$, genus $g$,
branched in $2g+2$ distinct points, uniquely determined by the branch divisor.
\begin{gather*}
\{\text{isomorphism classes of hyperellipetic curves of genus $g$}\}\\
=\mathbb{P}(V_{2g+2})_\Delta/\text{SL}_2 \hookrightarrow
\mathbb{P}(V_{2g+2})^{ss}//\text{SL}_2
\end{gather*}
projective normal of dimension $2g-3$.

\subsection*{Local structure of quotients}
\label{subsection-local-structure-of-quotients}
$X$ affine $G$-variety, $G$ reductive, $x\in X$ such that $G\cdot x$ is closed,
$H=G_x$ (reductive).

\begin{theorem}
\label{theorem-}
There exists an affine $H$-stable (locally ?) subvariety $S \subset X$ such that
the sequence
$$
\xymatrix{
(g,s)H\ar[r]^{\text{mapsto}}&g\cdot s\\
(G\times S)/H\ar[d]\ar[r]&  X\ar[d]\\
S//H\ar[r]&  X//G\\
x\ar[r]_{mapsto} &\overline{x}
}
$$
$h(g,s)=(gh^{-1},hs)$ is cartesian with étale horizontal arrows. If $x \in X$ is
smooth, then there is an $H$-equivariant morphism.
\begin{align*}
S &\longrightarrow T_x(X)/T_x(G\cdot x)=N_{G\cdot x/X,x} \\
x &\longmapsto 0
\end{align*}
\end{theorem}

This finishes the statement of the theorem. The last map may be interpreted as
the inverse of the exponential map from differential geometry.

A corollary:

\begin{lemma}
\label{lemma-}
If $G_x=e$ then $X \to X//G$ is a principal $G$-bundle in a $G$-stable
neighbourhood of $x$.
\end{lemma}

\begin{exercise}
\label{exercise-also-works-for-G-finite-and-S-open}
$G$ finite, works with $S\subset X$ open.
\end{exercise}

\begin{exercise}
\label{exercise-}
$G=\text{SL}_2$, $V=V_G$, $x^3y^3 \in V_G$, stabilizer
$=\mathbb{G}_m=\text{Im}(\lambda)$. Prove that $V_G//\text{SL}_2$ is singular at
the image of $x^3y^3$. ($S=\mathbb{G}_m$-module with weights $-6,-4,0,4,6$.)
{\bf Hint.} To
check for non-smoothness you can check that $\mathcal{O}(S)^{\mathbb{G}_m}$ is
not a polynomial ring; this is equivalent to non-smoothness for the quotient,
and it should be easy to find: you just find that there are too many generators.
\end{exercise}



\end{document}
